\section{Partition Coordinate Theory}
\label{sec:partition_coordinates}

\subsection{Categorical States in Bounded Systems}

\begin{definition}[Categorical State]
\label{def:categorical_state}
A categorical state $\mathcal{C}$ is a discrete, distinguishable configuration of a bounded system. Two states $\mathcal{C}_i$ and $\mathcal{C}_j$ are categorically distinct if they can be distinguished by at least one measurement operation.
\end{definition}

For a bounded system with finite phase space volume $\mu(M) < \infty$ observed by agents with finite resolution $\delta x$, the number of distinguishable categorical states is finite. The categorical state space $\Cspace$ is the set of all such distinguishable configurations.

\begin{axiom}[Finite Categorical Capacity]
\label{ax:finite_capacity}
Any bounded physical system admits finite categorical capacity: $|\Cspace| < \infty$.
\end{axiom}

\subsection{Partition Coordinates}

\begin{definition}[Partition Coordinates]
\label{def:partition_coordinates}
The partition coordinates $(n, \ell, m, s)$ characterize a categorical state through:
\begin{itemize}
\item $n \in \mathbb{N}$: partition depth (principal quantum number analog)
\item $\ell \in \{0, 1, \ldots, n-1\}$: angular complexity (azimuthal quantum number analog)
\item $m \in \{-\ell, -\ell+1, \ldots, +\ell\}$: orientation (magnetic quantum number analog)
\item $s \in \{-1/2, +1/2\}$: chirality (spin quantum number analog)
\end{itemize}
\end{definition}

\begin{theorem}[Capacity Formula]
\label{thm:capacity_formula}
The number of accessible categorical states at partition depth $n$ is
\begin{equation}
C(n) = 2n^2
\end{equation}
\end{theorem}

\begin{proof}
For fixed $n$, angular complexity $\ell$ ranges from $0$ to $n-1$. For each $\ell$, orientation $m$ ranges from $-\ell$ to $+\ell$, giving $2\ell + 1$ values. Chirality $s$ provides factor 2. Total count:
\begin{align}
C(n) &= 2 \sum_{\ell=0}^{n-1} (2\ell + 1) \\
&= 2 \sum_{\ell=0}^{n-1} (2\ell + 1) \\
&= 2 \left[ 2 \sum_{\ell=0}^{n-1} \ell + \sum_{\ell=0}^{n-1} 1 \right] \\
&= 2 \left[ 2 \cdot \frac{(n-1)n}{2} + n \right] \\
&= 2[n(n-1) + n] \\
&= 2n^2
\end{align}
\end{proof}

\begin{corollary}[Polynomial Growth]
Categorical capacity grows polynomially with partition depth: $C(n) \sim \mathcal{O}(n^2)$.
\end{corollary}

\subsection{Commutation Relations}

\begin{theorem}[Partition Coordinate Commutation]
\label{thm:partition_commutation}
The partition coordinate operators satisfy
\begin{equation}
[\hat{n}, \hat{\ell}] = [\hat{\ell}, \hat{m}] = [\hat{m}, \hat{s}] = 0
\end{equation}
\end{theorem}

\begin{proof}
Partition coordinates are discrete labels, not continuous observables. Measuring $n$ determines which partition level; measuring $\ell$ determines which angular manifold within that level; measuring $m$ determines which orientation within that manifold; measuring $s$ determines chirality. These are sequential refinements, not competing measurements. Each measurement narrows the categorical state space without perturbing previously determined coordinates.

Formally, let $|\psi\rangle = |n, \ell, m, s\rangle$ be an eigenstate. Then:
\begin{align}
\hat{n} \hat{\ell} |\psi\rangle &= \hat{n} (\ell |\psi\rangle) = \ell n |\psi\rangle \\
\hat{\ell} \hat{n} |\psi\rangle &= \hat{\ell} (n |\psi\rangle) = n \ell |\psi\rangle
\end{align}
Since $\ell n = n \ell$, we have $[\hat{n}, \hat{\ell}] = 0$. Similar arguments apply to other pairs.
\end{proof}

\begin{corollary}[Simultaneous Measurability]
All four partition coordinates can be measured simultaneously with no uncertainty trade-off.
\end{corollary}

\subsection{Completeness of Partition Coordinates}

\begin{theorem}[Partition Coordinate Completeness]
\label{thm:partition_completeness}
The four partition coordinates $(n, \ell, m, s)$ provide complete characterization of categorical states in bounded systems.
\end{theorem}

\begin{proof}
Completeness requires that specification of $(n, \ell, m, s)$ uniquely determines the categorical state. We prove by construction.

\textbf{Step 1}: Partition depth $n$ determines the energy scale and number of accessible states $C(n) = 2n^2$.

\textbf{Step 2}: Angular complexity $\ell \in \{0, \ldots, n-1\}$ determines the topological structure. For $\ell = 0$: spherically symmetric. For $\ell > 0$: $\ell$ angular nodes.

\textbf{Step 3}: Orientation $m \in \{-\ell, \ldots, +\ell\}$ determines spatial orientation of angular structure. This provides $2\ell + 1$ distinct orientations.

\textbf{Step 4}: Chirality $s \in \{-1/2, +1/2\}$ determines handedness, distinguishing enantiomers.

Total specification: $(n, \ell, m, s)$ selects one state from $C(n) = 2n^2$ possibilities. Since capacity formula counts all accessible states, the coordinates are complete.
\end{proof}

\subsection{Mapping to Physical Observables}

\begin{proposition}[Energy Correspondence]
\label{prop:energy_correspondence}
Partition depth $n$ corresponds to energy through
\begin{equation}
E_n = -\frac{E_0}{n^2}
\end{equation}
where $E_0$ is the ground state binding energy.
\end{proposition}

\begin{proof}
This is the Rydberg formula for hydrogen-like systems. For molecular systems, $E_0$ represents the dissociation energy, and $n$ counts partition levels from ground state.
\end{proof}

\begin{proposition}[Angular Momentum Correspondence]
\label{prop:angular_correspondence}
Angular complexity $\ell$ corresponds to angular momentum through
\begin{equation}
L = \hbar \sqrt{\ell(\ell + 1)}
\end{equation}
\end{proposition}

\begin{proposition}[Magnetic Moment Correspondence]
\label{prop:magnetic_correspondence}
Orientation $m$ corresponds to magnetic moment projection through
\begin{equation}
\mu_z = m \mu_B
\end{equation}
where $\mu_B$ is the Bohr magneton.
\end{proposition}

\subsection{Partition Operations}

\begin{definition}[Partition Operation]
\label{def:partition_operation}
A partition operation $\Pi: \Cspace \to \Cspace \times \Cspace$ maps a categorical state to a pair of daughter states, conserving total quantum numbers:
\begin{equation}
\Pi(\mathcal{C}) = (\mathcal{C}_1, \mathcal{C}_2), \quad n_1 + n_2 = n, \quad \ell_1 + \ell_2 = \ell
\end{equation}
\end{definition}

\begin{definition}[Partition Lag]
\label{def:partition_lag}
The partition lag $\taulag$ is the time required for a partition operation to complete. During this interval, the categorical state is undetermined.
\end{definition}

\begin{theorem}[Partition Lag Bounds]
\label{thm:partition_lag_bounds}
The partition lag satisfies
\begin{equation}
\frac{\hbar}{E_n - E_{n-1}} \leq \taulag \leq \frac{1}{\Gamma_n}
\end{equation}
where $E_n$ is energy at level $n$ and $\Gamma_n$ is decay rate.
\end{theorem}

\begin{proof}
Lower bound: Energy-time uncertainty $\Delta E \cdot \Delta t \geq \hbar$ with $\Delta E = E_n - E_{n-1}$ gives $\taulag \geq \hbar/\Delta E$.

Upper bound: Partition cannot complete faster than decay rate $\Gamma_n$, giving $\taulag \leq 1/\Gamma_n$.
\end{proof}

\subsection{Categorical Distance}

\begin{definition}[Categorical Distance]
\label{def:categorical_distance}
The categorical distance between states $\mathcal{C}_i = (n_i, \ell_i, m_i, s_i)$ and $\mathcal{C}_j = (n_j, \ell_j, m_j, s_j)$ is
\begin{equation}
\dcat(\mathcal{C}_i, \mathcal{C}_j) = |n_i - n_j| + |\ell_i - \ell_j| + |m_i - m_j| + |s_i - s_j|
\end{equation}
\end{definition}

\begin{proposition}[Metric Properties]
\label{prop:metric_properties}
Categorical distance $\dcat$ satisfies metric axioms:
\begin{enumerate}
\item Non-negativity: $\dcat(\mathcal{C}_i, \mathcal{C}_j) \geq 0$
\item Identity: $\dcat(\mathcal{C}_i, \mathcal{C}_j) = 0 \iff \mathcal{C}_i = \mathcal{C}_j$
\item Symmetry: $\dcat(\mathcal{C}_i, \mathcal{C}_j) = \dcat(\mathcal{C}_j, \mathcal{C}_i)$
\item Triangle inequality: $\dcat(\mathcal{C}_i, \mathcal{C}_k) \leq \dcat(\mathcal{C}_i, \mathcal{C}_j) + \dcat(\mathcal{C}_j, \mathcal{C}_k)$
\end{enumerate}
\end{proposition}

\begin{proof}
Properties (1)-(3) follow immediately from definition. For triangle inequality:
\begin{align}
\dcat(\mathcal{C}_i, \mathcal{C}_k) &= |n_i - n_k| + |\ell_i - \ell_k| + |m_i - m_k| + |s_i - s_k| \\
&\leq |n_i - n_j| + |n_j - n_k| + |\ell_i - \ell_j| + |\ell_j - \ell_k| \\
&\quad + |m_i - m_j| + |m_j - m_k| + |s_i - s_j| + |s_j - s_k| \\
&= \dcat(\mathcal{C}_i, \mathcal{C}_j) + \dcat(\mathcal{C}_j, \mathcal{C}_k)
\end{align}
\end{proof}

\subsection{Selection Rules}

\begin{theorem}[Partition Selection Rules]
\label{thm:selection_rules}
Allowed partition transitions satisfy:
\begin{align}
\Delta n &= \pm 1, \pm 2, \ldots \\
\Delta \ell &= \pm 1 \\
\Delta m &= 0, \pm 1 \\
\Delta s &= 0
\end{align}
\end{theorem}

\begin{proof}
These are conservation laws for partition operations:

\textbf{$\Delta n$ unrestricted}: Energy can be redistributed arbitrarily among fragments.

\textbf{$\Delta \ell = \pm 1$}: Angular momentum conservation requires emitted photon/particle carries $\ell = 1$, giving $\Delta \ell = \pm 1$.

\textbf{$\Delta m = 0, \pm 1$}: Projection conservation allows $\Delta m = 0$ (parallel transition) or $\Delta m = \pm 1$ (perpendicular transition).

\textbf{$\Delta s = 0$}: Chirality is conserved in partition operations (no parity violation).
\end{proof}
