\section{Partition Coordinate Theory}
\label{sec:partition_coordinates}

\subsection{Categorical States in Bounded Systems}

\begin{definition}[Categorical State]
\label{def:categorical_state}
A categorical state $\mathcal{C}$ is a discrete, distinguishable configuration of a bounded system. Two states $\mathcal{C}_i$ and $\mathcal{C}_j$ are categorically distinct if they can be distinguished by at least one measurement operation.
\end{definition}

For a bounded system with finite phase space volume $\mu(M) < \infty$ observed by agents with finite resolution $\delta x$, the number of distinguishable categorical states is finite. The categorical state space $\Cspace$ is the set of all such distinguishable configurations.

\begin{axiom}[Finite Categorical Capacity]
\label{ax:finite_capacity}
Any bounded physical system admits finite categorical capacity: $|\Cspace| < \infty$.
\end{axiom}

\begin{figure}[htbp]
    \centering
    \includegraphics[width=\textwidth]{figures/figure_1_bounded_phase_space.png}
    \caption{\textbf{Bounded phase space partition structure demonstrating quantum-classical geometric equivalence.}
    (\textbf{A}) Bounded phase space showing circular boundary in position-momentum coordinates. Concentric circles (blue to purple gradient) represent energy contours within the bounded domain. Phase space boundary (thick black circle) defines maximum accessible region with normalized coordinates $|x|, |p| \leq 1$.
    (\textbf{B}) Discrete partition cells $(n, \ell, m, s)$ overlaid on phase space. Dashed circular contours show radial partitions corresponding to principal quantum number $n$. Angular partitions (radial lines) correspond to angular momentum quantum numbers. Yellow highlighted cells ($n=3$, $n=4$) demonstrate discrete state localization within continuous phase space.
    (\textbf{C}) Quantum view showing energy levels $E_n$ versus state degeneracy. Horizontal lines indicate energy eigenvalues for $n = 1$ to $5$ with corresponding capacities $C = 2n^2$ (2, 8, 18, 32, 50 states). Pink dots represent individual quantum states, demonstrating discrete energy spectrum with increasing degeneracy.
    (\textbf{D}) Classical view showing trajectory segments in phase space. Colored circular orbits represent classical trajectories for different energy levels: $n=1, E=1$ (purple), $n=2, E=4$ (blue), $n=3, E=9$ (cyan), $n=4, E=16$ (green), $n=5, E=25$ (yellow). Demonstrates continuous classical motion within same geometric structure as quantum partition cells.}
    \label{fig:bounded_phase_space}
\end{figure}


\subsection{Partition Coordinates}

\begin{definition}[Partition Coordinates]
\label{def:partition_coordinates}
The partition coordinates $(n, \ell, m, s)$ characterize a categorical state through:
\begin{itemize}
\item $n \in \mathbb{N}$: partition depth (principal quantum number analog)
\item $\ell \in \{0, 1, \ldots, n-1\}$: angular complexity (azimuthal quantum number analog)
\item $m \in \{-\ell, -\ell+1, \ldots, +\ell\}$: orientation (magnetic quantum number analog)
\item $s \in \{-1/2, +1/2\}$: chirality (spin quantum number analog)
\end{itemize}
\end{definition}

\begin{theorem}[Capacity Formula]
\label{thm:capacity_formula}
The number of accessible categorical states at partition depth $n$ is
\begin{equation}
C(n) = 2n^2
\end{equation}
\end{theorem}

\begin{proof}
For fixed $n$, angular complexity $\ell$ ranges from $0$ to $n-1$. For each $\ell$, orientation $m$ ranges from $-\ell$ to $+\ell$, giving $2\ell + 1$ values. Chirality $s$ provides factor 2. Total count:
\begin{align}
C(n) &= 2 \sum_{\ell=0}^{n-1} (2\ell + 1) \\
&= 2 \sum_{\ell=0}^{n-1} (2\ell + 1) \\
&= 2 \left[ 2 \sum_{\ell=0}^{n-1} \ell + \sum_{\ell=0}^{n-1} 1 \right] \\
&= 2 \left[ 2 \cdot \frac{(n-1)n}{2} + n \right] \\
&= 2[n(n-1) + n] \\
&= 2n^2
\end{align}
\end{proof}

\begin{corollary}[Polynomial Growth]
Categorical capacity grows polynomially with partition depth: $C(n) \sim \mathcal{O}(n^2)$.
\end{corollary}

\subsection{Commutation Relations}

\begin{theorem}[Partition Coordinate Commutation]
\label{thm:partition_commutation}
The partition coordinate operators satisfy
\begin{equation}
[\hat{n}, \hat{\ell}] = [\hat{\ell}, \hat{m}] = [\hat{m}, \hat{s}] = 0
\end{equation}
\end{theorem}

\begin{proof}
Partition coordinates are discrete labels, not continuous observables. Measuring $n$ determines which partition level; measuring $\ell$ determines which angular manifold within that level; measuring $m$ determines which orientation within that manifold; measuring $s$ determines chirality. These are sequential refinements, not competing measurements. Each measurement narrows the categorical state space without perturbing previously determined coordinates.

Formally, let $|\psi\rangle = |n, \ell, m, s\rangle$ be an eigenstate. Then:
\begin{align}
\hat{n} \hat{\ell} |\psi\rangle &= \hat{n} (\ell |\psi\rangle) = \ell n |\psi\rangle \\
\hat{\ell} \hat{n} |\psi\rangle &= \hat{\ell} (n |\psi\rangle) = n \ell |\psi\rangle
\end{align}
Since $\ell n = n \ell$, we have $[\hat{n}, \hat{\ell}] = 0$. Similar arguments apply to other pairs.
\end{proof}

\begin{corollary}[Simultaneous Measurability]
All four partition coordinates can be measured simultaneously with no uncertainty trade-off.
\end{corollary}

\begin{figure}[htbp]
    \centering
    \includegraphics[width=\textwidth]{figures/panel_01_commutation.png}
    \caption{\textbf{Fundamental commutation and categorical observable validation.} 
    (\textbf{A}) Commutator matrix showing near-zero commutation relations between categorical observables ($$n, \ell, m, s$$) and physical observables (position $$x$$, momentum $$p$$, Hamiltonian $$H$$, angular momentum $$L^2$$). All elements satisfy $$|[\hat{O}_{\text{cat}}, \hat{O}_{\text{phys}}]| < 10^{-15}$$, confirming theoretical prediction of exact commutation. 
    (\textbf{B}) Measurement backaction comparison between position/momentum measurements (red, $$\Delta p/p \sim 10^2$$) and categorical measurements (green, $$\Delta p/p \sim 10^{-3}$$). Categorical measurements achieve momentum disturbance three orders of magnitude below classical limits. 
    (\textbf{C}) Observer invariance test demonstrating perfect correlation ($$R^2 = 1.000000$$, $$N = 10{,}000$$ trials) between two independent measurement modalities, confirming that physical reality is observer-invariant. 
    (\textbf{D}) Three-dimensional partition space structure showing the 1s$$\rightarrow$$2p transition trajectory (red line) through quantum number space $$(n, \ell, m)$$. Spheres indicate measured partition states; trajectory exhibits deterministic evolution through intermediate states with energy color-coded along the path.}
    \label{fig:commutation}
    \end{figure}

\subsection{Completeness of Partition Coordinates}

\begin{theorem}[Partition Coordinate Completeness]
\label{thm:partition_completeness}
The four partition coordinates $(n, \ell, m, s)$ provide complete characterization of categorical states in bounded systems.
\end{theorem}

\begin{proof}
Completeness requires that specification of $(n, \ell, m, s)$ uniquely determines the categorical state. We prove by construction.

\textbf{Step 1}: Partition depth $n$ determines the energy scale and number of accessible states $C(n) = 2n^2$.

\textbf{Step 2}: Angular complexity $\ell \in \{0, \ldots, n-1\}$ determines the topological structure. For $\ell = 0$: spherically symmetric. For $\ell > 0$: $\ell$ angular nodes.

\textbf{Step 3}: Orientation $m \in \{-\ell, \ldots, +\ell\}$ determines spatial orientation of angular structure. This provides $2\ell + 1$ distinct orientations.

\textbf{Step 4}: Chirality $s \in \{-1/2, +1/2\}$ determines handedness, distinguishing enantiomers.

Total specification: $(n, \ell, m, s)$ selects one state from $C(n) = 2n^2$ possibilities. Since capacity formula counts all accessible states, the coordinates are complete.
\end{proof}

\subsection{Mapping to Physical Observables}

\begin{proposition}[Energy Correspondence]
\label{prop:energy_correspondence}
Partition depth $n$ corresponds to energy through
\begin{equation}
E_n = -\frac{E_0}{n^2}
\end{equation}
where $E_0$ is the ground state binding energy.
\end{proposition}

\begin{proof}
This is the Rydberg formula for hydrogen-like systems. For molecular systems, $E_0$ represents the dissociation energy, and $n$ counts partition levels from ground state.
\end{proof}

\begin{proposition}[Angular Momentum Correspondence]
\label{prop:angular_correspondence}
Angular complexity $\ell$ corresponds to angular momentum through
\begin{equation}
L = \hbar \sqrt{\ell(\ell + 1)}
\end{equation}
\end{proposition}

\begin{proposition}[Magnetic Moment Correspondence]
\label{prop:magnetic_correspondence}
Orientation $m$ corresponds to magnetic moment projection through
\begin{equation}
\mu_z = m \mu_B
\end{equation}
where $\mu_B$ is the Bohr magneton.
\end{proposition}

\begin{figure*}[htbp]
    \centering
    \includegraphics[width=\textwidth]{figures/categorical_partition_panel.png}
    \caption{\textbf{Categorical structure and partition geometry.} 
    Continuous observables discretize into categorical states via finite observer resolution, generating quantum numbers $(n, l, m, s)$ with $2n^2$ shell capacity.
    %
    \textbf{(Row 1, Left)} Continuous $\to$ categorical: oscillating signal (blue/yellow) discretizes into finite observer bins. Finite resolution transforms continuous variable into categorical states.
    %
    \textbf{(Row 1, Center-Left)} Completion order (Hasse diagram): directed acyclic graph shows hierarchical ordering of 8 categorical states. Arrows indicate completion dependencies, forming partially ordered set (poset).
    %
    \textbf{(Row 1, Center-Right)} Temporal emergence: sigmoid curve shows categories completed over time, reaching 95\% by $t = 10$. Red dashed lines mark discrete completion events. Irreversible monotonic growth.
    %
    \textbf{(Row 1, Right)} Categorical irreversibility: completion function $\mu(C,t)$ increases monotonically (blue staircase) from 0 to 9 states. Red arrow indicates irreversible time direction.
    %
    \textbf{(Row 2, Left)} Partition coordinates $(n, l, m)$: 3D scatter shows quantum state distribution. Colors indicate depth $n$ (purple: $n=1$, blue: $n=2$, green: $n=3$, yellow: $n=4$). States organized in shells.
    %
    \textbf{(Row 2, Center-Left)} Shell capacity theorem: $N(n) = 2n^2$. Blue bars show shell capacity (2, 8, 18, 32, 50, 72, 98, 128, 162, 200, 242, 280), orange cumulative curve. Perfect quadratic scaling.
    %
    \textbf{(Row 2, Center-Right)} Energy ordering rule: $(n + \alpha l)$ with $\alpha = 1$ generates Madelung rule (1s, 2s, 2p, 3s, 3p, 4s, 3d, ...). Horizontal bars show orbital filling sequence matching periodic table.
    %
    \textbf{(Row 2, Right)} Selection rules: $\Delta l = \pm 1$ allowed transitions. Diagram shows allowed paths (yellow arrows) between angular momentum levels (s, p, d, f). Energy increases vertically.
    %
    \textbf{(Row 3, Left)} Spherical harmonic $Y_2^0(\theta, \phi)$: 3D visualization shows $l=2$, $m=0$ angular distribution. Blue (positive) and red (negative) lobes demonstrate spatial anisotropy.
    %
    \textbf{(Row 3, Center-Left)} Angular momentum states: $l = 0, 1, 2$ with $m \in \{-l, ..., +l\}$. Grid shows probability densities for all $(l, m)$ combinations. Red/blue patterns indicate phase structure.
    %
    \textbf{(Row 3, Center-Right)} Chirality $s = \pm 1/2$: spin-up (blue, right-handed) and spin-down (red, left-handed) phase trajectories. Circular paths with opposite orientations demonstrate intrinsic angular momentum.
    %
    \textbf{(Row 3, Right)} State degeneracy: $g(n) = 2n^2$. Bars show total states per shell ($n=1$: 2, $n=2$: 8, $n=3$: 18, $n=4$: 32). Green shading indicates cumulative capacity.
    %
    Validation: Shell capacity $N(n) = 2n^2$, Madelung rule $(n + l)$ ordering, $\Delta l = \pm 1$ selection rules, $g(n) = 2n^2$ degeneracy.}
    \label{fig:categorical_partition}
    \end{figure*}

\subsection{Triple Equivalence Foundation}

The partition coordinate structure emerges from a deeper mathematical principle: any bounded dynamical system admits three equivalent descriptions.

\begin{theorem}[Triple Equivalence for Bounded Systems]
\label{thm:triple_equivalence}
For any bounded dynamical system, the following three descriptions are mathematically equivalent:
\begin{enumerate}
\item \textbf{Oscillatory}: The system exhibits periodic motion with frequency $\omega = 2\pi/T$
\item \textbf{Categorical}: The system traverses $M$ distinguishable states per period
\item \textbf{Partition}: The period $T$ is partitioned into $M$ temporal segments
\end{enumerate}
These are not three separate phenomena but three perspectives on a single underlying structure.
\end{theorem}

\begin{proof}
\textbf{Boundedness implies oscillation}: For a system confined to bounded phase space $\mathcal{D} \subset \mathbb{R}^n$, the Poincaré recurrence theorem guarantees return to arbitrary neighborhoods of initial states. For continuous dynamics, trajectories reaching the boundary $\partial\mathcal{D}$ must reverse, creating oscillatory motion.

\textbf{Oscillation defines categories}: An oscillating system traverses distinct states $(x(t), p(t))$ during one period $T$. The set of distinguishable states visited constitutes the categorical structure. The partition coordinates $(n, \ell, m, s)$ label these categories.

\textbf{Categories partition the period}: The period decomposes into temporal segments $T = \sum_{i=1}^M \tau_i$ where $\tau_i$ is time spent in category $i$. This partitioning is equivalent to the categorical structure.

\textbf{Quantitative relationship}: The fundamental identity connecting all three perspectives is:
\begin{equation}
\frac{dM}{dt} = \frac{\omega}{2\pi/M} = \frac{1}{\langle\tau_p\rangle}
\label{eq:triple_identity}
\end{equation}
where $dM/dt$ is the categorical actualization rate, $\omega$ is oscillation frequency, and $\langle\tau_p\rangle$ is average partition duration.
\end{proof}

\begin{corollary}[Trapped Ions Instantiate Triple Equivalence]
A single ion in a Penning trap exhibits:
\begin{itemize}
\item \textbf{Oscillatory}: Three trap frequencies $\omega_c$ (cyclotron), $\omega_z$ (axial), $\omega_r$ (radial)
\item \textbf{Categorical}: Discrete partition states $(n, \ell, m, s)$ with capacity $C(n) = 2n^2$
\item \textbf{Partition}: Temporal segments corresponding to each categorical state
\end{itemize}
The trapped ion is therefore a physical realization of the triple equivalence structure.
\end{corollary}

\subsection{Thermodynamic Properties of Partition States}

The triple equivalence reveals that trapped ions exhibit thermodynamic properties analogous to gas molecules, with partition coordinates playing the role of statistical mechanical degrees of freedom.

\begin{theorem}[Categorical Temperature]
\label{thm:categorical_temperature}
For an ion with partition coordinate evolution rate $dM/dt$, the categorical temperature is:
\begin{equation}
T = \frac{\hbar}{k_B} \frac{dM}{dt}
\end{equation}
where $\hbar$ is Planck's constant and $k_B$ is Boltzmann's constant.
\end{theorem}

\begin{proof}
Temperature measures the rate of categorical actualization. From the triple equivalence (Equation~\ref{eq:triple_identity}):
\begin{equation}
\frac{dM}{dt} = \frac{\omega}{2\pi}
\end{equation}
for a system with one category per radian. The quantum mechanical energy is $E = \hbar\omega$, giving:
\begin{equation}
k_B T = \frac{\hbar \omega}{2\pi} = \frac{E}{2\pi} \cdot \frac{2\pi}{1} = E \quad \text{(for $M = 2\pi$ per period)}
\end{equation}
Therefore $k_B$ converts between energy (measured in joules) and categorical rate (measured in categories per unit time).
\end{proof}

\begin{theorem}[Categorical Pressure]
\label{thm:categorical_pressure}
For ions occupying categorical states in volume $V$, the pressure is:
\begin{equation}
P = k_B T \frac{M}{V}
\end{equation}
where $M$ is the number of accessible categorical states and $V$ is the trap volume.
\end{theorem}

\begin{proof}
Pressure is categorical density—the density of distinguishable states per unit volume. From thermodynamic relations:
\begin{equation}
P = k_B T \left(\frac{\partial M}{\partial V}\right)_S
\end{equation}
For ions in a trap with fixed categorical structure, $\partial M/\partial V = M/V$, giving $P = k_B T M/V$.

This is not a boundary phenomenon (as in kinetic theory) but a bulk property existing throughout the trap volume. The categorical structure exists everywhere the ion can be found.
\end{proof}

\begin{corollary}[Ideal Gas Law for Single Ion]
\label{cor:ideal_gas_ion}
A single ion in a trap satisfies:
\begin{equation}
PV = k_B T
\end{equation}
which is the ideal gas law for $N = 1$ particle.
\end{corollary}

\begin{proposition}[Bounded Maxwell-Boltzmann Distribution]
\label{prop:bounded_maxwell}
The velocity distribution over partition categories is:
\begin{equation}
f(m) = \frac{e^{-\beta E_m}}{\sum_{m=0}^{M_{\max}} e^{-\beta E_m}}
\end{equation}
where $m = 0, 1, \ldots, M_{\max}$ are discrete categories, $E_m$ is energy at category $m$, $\beta = 1/(k_B T)$, and $M_{\max}$ corresponds to $v_{\max} = c$ (speed of light).

This distribution is naturally bounded at relativistic velocities without ad hoc corrections, resolving the infinite tail problem of the classical Maxwell-Boltzmann distribution.
\end{proposition}

\subsection{Partition Operations}

\begin{definition}[Partition Operation]
\label{def:partition_operation}
A partition operation $\Pi: \Cspace \to \Cspace \times \Cspace$ maps a categorical state to a pair of daughter states, conserving total quantum numbers:
\begin{equation}
\Pi(\mathcal{C}) = (\mathcal{C}_1, \mathcal{C}_2), \quad n_1 + n_2 = n, \quad \ell_1 + \ell_2 = \ell
\end{equation}
\end{definition}

\begin{definition}[Partition Lag]
\label{def:partition_lag}
The partition lag $\taulag$ is the time required for a partition operation to complete. During this interval, the categorical state is undetermined.
\end{definition}

\begin{theorem}[Partition Lag Bounds]
\label{thm:partition_lag_bounds}
The partition lag satisfies
\begin{equation}
\frac{\hbar}{E_n - E_{n-1}} \leq \taulag \leq \frac{1}{\Gamma_n}
\end{equation}
where $E_n$ is energy at level $n$ and $\Gamma_n$ is decay rate.
\end{theorem}

\begin{proof}
Lower bound: Energy-time uncertainty $\Delta E \cdot \Delta t \geq \hbar$ with $\Delta E = E_n - E_{n-1}$ gives $\taulag \geq \hbar/\Delta E$.

Upper bound: Partition cannot complete faster than decay rate $\Gamma_n$, giving $\taulag \leq 1/\Gamma_n$.
\end{proof}

\subsection{Categorical Distance}

\begin{definition}[Categorical Distance]
\label{def:categorical_distance}
The categorical distance between states $\mathcal{C}_i = (n_i, \ell_i, m_i, s_i)$ and $\mathcal{C}_j = (n_j, \ell_j, m_j, s_j)$ is
\begin{equation}
\dcat(\mathcal{C}_i, \mathcal{C}_j) = |n_i - n_j| + |\ell_i - \ell_j| + |m_i - m_j| + |s_i - s_j|
\end{equation}
\end{definition}

\begin{proposition}[Metric Properties]
\label{prop:metric_properties}
Categorical distance $\dcat$ satisfies metric axioms:
\begin{enumerate}
\item Non-negativity: $\dcat(\mathcal{C}_i, \mathcal{C}_j) \geq 0$
\item Identity: $\dcat(\mathcal{C}_i, \mathcal{C}_j) = 0 \iff \mathcal{C}_i = \mathcal{C}_j$
\item Symmetry: $\dcat(\mathcal{C}_i, \mathcal{C}_j) = \dcat(\mathcal{C}_j, \mathcal{C}_i)$
\item Triangle inequality: $\dcat(\mathcal{C}_i, \mathcal{C}_k) \leq \dcat(\mathcal{C}_i, \mathcal{C}_j) + \dcat(\mathcal{C}_j, \mathcal{C}_k)$
\end{enumerate}
\end{proposition}

\begin{proof}
Properties (1)-(3) follow immediately from definition. For triangle inequality:
\begin{align}
\dcat(\mathcal{C}_i, \mathcal{C}_k) &= |n_i - n_k| + |\ell_i - \ell_k| + |m_i - m_k| + |s_i - s_k| \\
&\leq |n_i - n_j| + |n_j - n_k| + |\ell_i - \ell_j| + |\ell_j - \ell_k| \\
&\quad + |m_i - m_j| + |m_j - m_k| + |s_i - s_j| + |s_j - s_k| \\
&= \dcat(\mathcal{C}_i, \mathcal{C}_j) + \dcat(\mathcal{C}_j, \mathcal{C}_k)
\end{align}
\end{proof}

\subsection{Selection Rules}

\begin{theorem}[Partition Selection Rules]
\label{thm:selection_rules}
Allowed partition transitions satisfy:
\begin{align}
\Delta n &= \pm 1, \pm 2, \ldots \\
\Delta \ell &= \pm 1 \\
\Delta m &= 0, \pm 1 \\
\Delta s &= 0
\end{align}
\end{theorem}

\begin{proof}
These are conservation laws for partition operations:

\textbf{$\Delta n$ unrestricted}: Energy can be redistributed arbitrarily among fragments.

\textbf{$\Delta \ell = \pm 1$}: Angular momentum conservation requires emitted photon/particle carries $\ell = 1$, giving $\Delta \ell = \pm 1$.

\textbf{$\Delta m = 0, \pm 1$}: Projection conservation allows $\Delta m = 0$ (parallel transition) or $\Delta m = \pm 1$ (perpendicular transition).

\textbf{$\Delta s = 0$}: Chirality is conserved in partition operations (no parity violation).
\end{proof}

\subsection{Autocatalytic Partition Dynamics}

The measurement process exhibits autocatalytic behavior: prior partitions modify the activation energy for subsequent partitions, leading to exponential rate enhancement.

\begin{theorem}[Information Catalysis]
\label{thm:information_catalysis}
The partition rate at depth $n$ is enhanced by prior partitions:
\begin{equation}
r_n = r_1^{(0)} \exp\left(\sum_{k=1}^{n-1} \beta \Delta E_k\right)
\end{equation}
where $r_1^{(0)}$ is the baseline rate, $\beta = 1/(k_B T)$, and $\Delta E_k$ is the activation energy reduction from partition $k$.
\end{theorem}

\begin{proof}
Each partition operation $k$ produces categorical information that reduces uncertainty for subsequent partitions. This information lowers the activation barrier through:
\begin{equation}
E_{\text{act}}^{(n)} = E_{\text{act}}^{(0)} - \sum_{k=1}^{n-1} \Delta E_k
\end{equation}

The partition rate follows Arrhenius law:
\begin{equation}
r_n = A \exp(-\beta E_{\text{act}}^{(n)}) = r_1^{(0)} \exp\left(\sum_{k=1}^{n-1} \beta \Delta E_k\right)
\end{equation}

For $\Delta E_k \approx \bar{\Delta E}$ (constant average reduction):
\begin{equation}
r_n = r_1^{(0)} e^{(n-1)\beta \bar{\Delta E}} \propto e^{\alpha n}
\end{equation}

This is exponential rate enhancement—the signature of autocatalysis.
\end{proof}

\begin{corollary}[Three-Phase Kinetics]
\label{cor:three_phase}
Autocatalytic partition dynamics exhibit three phases:
\begin{enumerate}
\item \textbf{Lag phase}: Initial slow accumulation of partition depth ($n < n_{\text{crit}}$)
\item \textbf{Exponential phase}: Rapid autocatalytic enhancement ($n_{\text{crit}} < n < n_{\text{max}}$)
\item \textbf{Saturation phase}: Termination at stable configuration ($n \to n_{\text{max}}$)
\end{enumerate}
\end{corollary}

\begin{definition}[Partition Terminator]
\label{def:partition_terminator}
A partition terminator is a categorical state where further partitioning is energetically unfavorable. For molecular identification, the terminator corresponds to unique identification ($N_M < 1$).
\end{definition}

\begin{theorem}[Terminator Accumulation]
\label{thm:terminator_accumulation}
The quintupartite measurement reaches a stable terminator when:
\begin{equation}
N_5 = N_0 \prod_{i=1}^5 \epsilon_i < 1
\end{equation}
At this point, autocatalytic enhancement ceases and the measurement process terminates with unique molecular identification.
\end{theorem}

\subsection{Ternary Representation of Partition Coordinates}

The three-fold structure of partition theory (oscillation-category-partition equivalence) naturally maps to base-3 (ternary) representation.

\begin{definition}[Ternary Encoding]
\label{def:ternary_encoding}
A partition state $(n, \ell, m, s)$ encodes as a ternary string where each trit (ternary digit) takes values $\{0, 1, 2\}$:
\begin{equation}
\text{State} \leftrightarrow (t_1 t_2 t_3 \ldots t_k)_3
\end{equation}
\end{definition}

\begin{proposition}[Ternary Efficiency]
\label{prop:ternary_efficiency}
Ternary representation is more efficient than binary for encoding three-dimensional categorical structure:
\begin{equation}
\text{Efficiency}_{\text{ternary}} = \frac{\log_2 3}{1} \approx 1.585 > 1 = \text{Efficiency}_{\text{binary}}
\end{equation}
\end{proposition}

\begin{theorem}[Position-Trajectory Duality]
\label{thm:position_trajectory_duality}
A ternary string simultaneously encodes:
\begin{enumerate}
\item \textbf{Position}: Final categorical state in S-space
\item \textbf{Trajectory}: Path taken through categorical hierarchy
\end{enumerate}
This duality arises because ternary composition is associative:
\begin{equation}
(t_1 \cdot t_2) \cdot t_3 = t_1 \cdot (t_2 \cdot t_3)
\end{equation}
The final position $t_1 t_2 t_3$ encodes the trajectory $(t_1 \to t_1 t_2 \to t_1 t_2 t_3)$.
\end{theorem}

\subsection{S-Entropy Coordinates}

The partition coordinates map to three-dimensional S-entropy space $(S_k, S_t, S_e)$, providing a compressed representation sufficient for navigation through infinite-dimensional categorical space.

\begin{definition}[S-Entropy Coordinates]
\label{def:s_entropy_coordinates}
For a partition state $(n, \ell, m, s)$:
\begin{align}
S_k &= \ln C(n) = \ln(2n^2) \quad \text{(knowledge entropy)} \\
S_t &= \int_{C_0}^{C(n)} \frac{dS}{dC} \, dC \quad \text{(temporal entropy)} \\
S_e &= -k_B |E(\mathcal{G})| \quad \text{(evolution entropy)}
\end{align}
where $C(n) = 2n^2$ is categorical capacity, $C_0$ is initial capacity, and $|E(\mathcal{G})|$ is the number of edges in the phase-lock coupling graph.
\end{definition}

\begin{theorem}[S-Coordinate Sufficiency]
\label{thm:s_coordinate_sufficiency}
The three S-entropy coordinates are sufficient statistics for optimal categorical space navigation. All information needed to progress from current state to unique identification is contained in $(S_k, S_t, S_e)$.
\end{theorem}

\begin{proof}
\textbf{Knowledge dimension $S_k$}: Measures remaining uncertainty. As measurements accumulate, equivalence class size shrinks: $C(n) \to C'(n) < C(n)$. Correspondingly, $S_k$ decreases. Unique identification corresponds to $S_k = 0$.

\textbf{Temporal dimension $S_t$}: Tracks categorical progression. By categorical irreversibility, once a state is completed, it cannot be revisited. This creates natural ordering and provides a temporal coordinate for the measurement process.

\textbf{Evolution dimension $S_e$}: Quantifies constraint accumulation through phase-lock coupling. More interactions mean more constraints, reducing accessible configurations.

\textbf{Sufficiency}: Total information from five modalities is:
\begin{equation}
I_{\text{total}} = \sum_{i=1}^5 I_i = S_k(0) - S_k(5) + \Delta S_t + \Delta S_e
\end{equation}
Since $I_{\text{total}}$ determines unique identification and is fully encoded in $(S_k, S_t, S_e)$, these three coordinates are sufficient.
\end{proof}

\begin{corollary}[Dimensional Compression]
\label{cor:dimensional_compression}
S-coordinates compress infinite-dimensional molecular configuration space to three dimensions:
\begin{equation}
\dim(\mathcal{C}) = \infty \xrightarrow{\text{S-projection}} \dim(\mathcal{S}) = 3
\end{equation}
This compression preserves all information needed for optimal categorical navigation.
\end{corollary}
