\section{Transport Dynamics and Partition Extinction}
\label{sec:transport_dynamics}

\subsection{Universal Transport Formula}

\begin{definition}[Transport Coefficient]
\label{def:transport_coefficient}
A transport coefficient $\Xi$ quantifies the response of a system to an applied gradient, relating flux $J$ to driving force $X$ through $J = \Xi^{-1} X$.
\end{definition}

For electrical transport: $\Xi = \rho$ (resistivity), $J = $ current density, $X = $ electric field.
For viscous transport: $\Xi = \mu$ (viscosity), $J = $ momentum flux, $X = $ velocity gradient.
For diffusive transport: $\Xi = D^{-1}$ (inverse diffusivity), $J = $ particle flux, $X = $ concentration gradient.
For thermal transport: $\Xi = \kappa^{-1}$ (inverse conductivity), $J = $ heat flux, $X = $ temperature gradient.

\begin{theorem}[Universal Transport Formula]
\label{thm:universal_transport}
All transport coefficients admit the universal form
\begin{equation}
\Xi = \mathcal{N}^{-1} \sum_{i,j} \taulag_{ij} g_{ij}
\end{equation}
where $\taulag_{ij}$ is the partition lag between carriers $i$ and $j$, $g_{ij}$ is the coupling strength, and $\mathcal{N}$ is a normalization factor.
\end{theorem}

\begin{proof}
Transport arises from partition operations between carriers. Each partition event with lag $\taulag_{ij}$ contributes to dissipation proportional to coupling strength $g_{ij}$. The transport coefficient measures total dissipation per unit flux.

\textbf{Step 1}: Partition rate between carriers $i$ and $j$ is $\Gamma_{ij} = \taulag_{ij}^{-1}$.

\textbf{Step 2}: Each partition generates entropy $\Delta S_{ij} = \kB \ln n_{\text{res},ij}$ where $n_{\text{res},ij}$ is undetermined residue count.

\textbf{Step 3}: Entropy production rate: $\dot{S} = \sum_{ij} \Gamma_{ij} \Delta S_{ij} = \sum_{ij} \taulag_{ij}^{-1} \kB \ln n_{\text{res},ij}$.

\textbf{Step 4}: Dissipation power: $P = T \dot{S} = T \sum_{ij} \taulag_{ij}^{-1} \kB \ln n_{\text{res},ij}$.

\textbf{Step 5}: For flux $J$, dissipation per unit flux: $\Xi = P/J^2 \propto \sum_{ij} \taulag_{ij} g_{ij}$ where $g_{ij} = \ln n_{\text{res},ij}$ is coupling strength.

Normalization factor $\mathcal{N}$ depends on carrier properties (density, charge, mass).
\end{proof}

\subsection{Partition Lag Temperature Dependence}

\begin{proposition}[Phonon-Limited Partition Lag]
\label{prop:phonon_partition_lag}
For phonon-limited processes, partition lag satisfies
\begin{equation}
\taulag(T) = \taulag_0 T^{-1}
\end{equation}
due to increasing phonon populations with temperature.
\end{proposition}

\begin{proof}
Phonon number density: $n_{\text{ph}} \sim T$ (classical limit). Partition rate proportional to phonon scattering rate: $\Gamma \sim n_{\text{ph}} \sim T$. Therefore $\taulag = \Gamma^{-1} \sim T^{-1}$.
\end{proof}

\begin{proposition}[Activated Partition Lag]
\label{prop:activated_partition_lag}
For activated processes, partition lag satisfies
\begin{equation}
\taulag(T) = \taulag_0 \exp(\Delta/\kB T)
\end{equation}
where $\Delta$ is activation energy barrier.
\end{proposition}

\begin{proof}
Arrhenius law: rate $\Gamma = \Gamma_0 \exp(-\Delta/\kB T)$. Therefore $\taulag = \Gamma^{-1} = \taulag_0 \exp(\Delta/\kB T)$.
\end{proof}

\subsection{Partition Extinction}

\begin{definition}[Phase-Locking]
\label{def:phase_locking}
Carriers $i$ and $j$ are phase-locked if they occupy the same categorical state: $\mathcal{C}_i = \mathcal{C}_j$.
\end{definition}

\begin{theorem}[Partition Extinction Theorem]
\label{thm:partition_extinction}
When carriers become phase-locked, partition operations between them become undefined. The partition lag undergoes discontinuous transition:
\begin{equation}
\taulag_{ij}(T) = \begin{cases}
\taulag_0(T) & T > T_c \\
0 & T < T_c
\end{cases}
\end{equation}
at critical temperature $T_c$, causing transport coefficient to vanish:
\begin{equation}
\Xi(T < T_c) = 0
\end{equation}
\end{theorem}

\begin{proof}
\textbf{Physical mechanism}: Above $T_c$, carriers occupy different categorical states. Partition operations between them are well-defined: determine which carrier belongs to which partition outcome. Below $T_c$, carriers occupy the same categorical state. Partition operations between them are undefined: cannot distinguish which carrier belongs to which outcome.

\textbf{Mathematical formulation}: Partition operation requires categorical distinction. For carriers $i$ and $j$ in states $\mathcal{C}_i$ and $\mathcal{C}_j$:
\begin{equation}
\Pi(\mathcal{C}_i, \mathcal{C}_j) = \begin{cases}
(\mathcal{C}_i', \mathcal{C}_j') & \text{if } \mathcal{C}_i \neq \mathcal{C}_j \\
\text{undefined} & \text{if } \mathcal{C}_i = \mathcal{C}_j
\end{cases}
\end{equation}

When $\mathcal{C}_i = \mathcal{C}_j$ (phase-locked), partition operation is undefined, giving $\taulag_{ij} = 0$.

\textbf{Discontinuity}: Categorical distinction is discrete—carriers are either distinguishable ($\mathcal{C}_i \neq \mathcal{C}_j$) or indistinguishable ($\mathcal{C}_i = \mathcal{C}_j$). No intermediate state exists. Therefore transition is discontinuous.

\textbf{Transport coefficient}: From universal formula (Theorem~\ref{thm:universal_transport}):
\begin{equation}
\Xi = \mathcal{N}^{-1} \sum_{i,j} \taulag_{ij} g_{ij}
\end{equation}
When all carriers phase-lock, $\taulag_{ij} = 0$ for all $i,j$, giving $\Xi = 0$.
\end{proof}

\begin{corollary}[Dissipationless Transport]
\label{cor:dissipationless_transport}
Phase-locked systems exhibit dissipationless transport: zero resistance, zero viscosity, or zero thermal resistance depending on transport mode.
\end{corollary}

\subsection{Critical Temperature}

\begin{theorem}[Phase-Locking Critical Temperature]
\label{thm:critical_temperature}
The critical temperature for phase-locking satisfies
\begin{equation}
T_c = \frac{\Delta_{\text{lock}}}{\kB}
\end{equation}
where $\Delta_{\text{lock}}$ is the phase-locking energy.
\end{theorem}

\begin{proof}
Phase-locking occurs when thermal energy $\kB T$ falls below energy required to maintain categorical distinction $\Delta_{\text{lock}}$. At $T = T_c$, these energies balance:
\begin{equation}
\kB T_c = \Delta_{\text{lock}}
\end{equation}
giving $T_c = \Delta_{\text{lock}}/\kB$.
\end{proof}

\begin{example}[Superconductivity]
For BCS superconductors, phase-locking energy is the gap energy $\Delta_{\text{lock}} = \Delta_{\text{BCS}}$. The critical temperature satisfies
\begin{equation}
\Delta_{\text{BCS}} = 1.76 \kB T_c
\end{equation}
This is the BCS gap relation, derived here from partition extinction.
\end{example}

\begin{example}[Superfluidity]
For helium-4, phase-locking occurs when thermal de Broglie wavelength equals interatomic spacing:
\begin{equation}
\lambda_{\text{th}} = \frac{h}{\sqrt{2\pi m \kB T_\lambda}} = a
\end{equation}
where $a \approx 3.6$ Å is interatomic spacing. This gives $T_\lambda = 2.17$ K, the observed $\lambda$-transition temperature.
\end{example}

\subsection{Undetermined Residue}

\begin{definition}[Undetermined Residue]
\label{def:undetermined_residue}
During partition lag $\taulag$, certain states cannot be assigned to either partition outcome. These states constitute undetermined residue with count $n_{\text{res}}$.
\end{definition}

\begin{proposition}[Residue Entropy]
\label{prop:residue_entropy}
Undetermined residue generates entropy
\begin{equation}
\Delta S_{\text{res}} = \kB \ln n_{\text{res}}
\end{equation}
\end{proposition}

\begin{proof}
Entropy measures number of accessible microstates. Undetermined residue represents $n_{\text{res}}$ states that could not be categorically assigned. Boltzmann formula: $S = \kB \ln \Omega$ with $\Omega = n_{\text{res}}$.
\end{proof}

\begin{theorem}[Dissipation-Residue Relation]
\label{thm:dissipation_residue}
Dissipation power equals temperature times residue entropy production rate:
\begin{equation}
P = T \dot{S}_{\text{res}} = T \sum_{ij} \Gamma_{ij} \kB \ln n_{\text{res},ij}
\end{equation}
\end{theorem}

\begin{proof}
Second law: dissipation converts work to heat through entropy production. Heat generation rate: $\dot{Q} = T \dot{S}$. For partition operations, entropy production comes from undetermined residue: $\dot{S} = \sum_{ij} \Gamma_{ij} \Delta S_{\text{res},ij}$. Substituting $\Delta S_{\text{res},ij} = \kB \ln n_{\text{res},ij}$ gives result.
\end{proof}

\subsection{Coupling Strength}

\begin{definition}[Phase-Lock Coupling]
\label{def:phase_lock_coupling}
The coupling strength $g_{ij}$ measures the degree to which carriers $i$ and $j$ are correlated:
\begin{equation}
g_{ij} = \langle \delta \mathcal{C}_i \delta \mathcal{C}_j \rangle
\end{equation}
where $\delta \mathcal{C}_i = \mathcal{C}_i - \langle \mathcal{C}_i \rangle$ is categorical state fluctuation.
\end{definition}

\begin{proposition}[Coupling Bounds]
\label{prop:coupling_bounds}
Coupling strength satisfies $0 \leq g_{ij} \leq 1$ with:
\begin{itemize}
\item $g_{ij} = 0$: uncorrelated (independent carriers)
\item $g_{ij} = 1$: fully correlated (phase-locked)
\end{itemize}
\end{proposition}

\begin{proof}
Correlation coefficient bounds: $-1 \leq \text{Corr}(X, Y) \leq 1$. For categorical states, negative correlation is unphysical (cannot anti-correlate discrete states), giving $0 \leq g_{ij} \leq 1$.
\end{proof}

\subsection{Transport Coefficient Scaling}

\begin{theorem}[Temperature Scaling]
\label{thm:temperature_scaling}
For phonon-limited transport, transport coefficient scales as
\begin{equation}
\Xi(T) = \Xi_0 T^{-1}
\end{equation}
in the high-temperature regime.
\end{theorem}

\begin{proof}
From Proposition~\ref{prop:phonon_partition_lag}: $\taulag \sim T^{-1}$. Coupling strength $g_{ij}$ is temperature-independent in classical regime. Universal formula: $\Xi \sim \sum_{ij} \taulag_{ij} g_{ij} \sim T^{-1}$.
\end{proof}

\begin{corollary}[Electrical Resistivity]
For metals in high-temperature regime: $\rho(T) \propto T$.
\end{corollary}

\begin{corollary}[Viscosity]
For gases in high-temperature regime: $\mu(T) \propto \sqrt{T}$ (Chapman-Enskog theory).
\end{corollary}

\subsection{Partition Extinction in Different Systems}

\begin{theorem}[Universality of Partition Extinction]
\label{thm:partition_extinction_universality}
Partition extinction occurs in any system where carriers can phase-lock:
\begin{enumerate}
\item Superconductors: Cooper pairs phase-lock $\to$ $\rho = 0$
\item Superfluids: Bosons condense $\to$ $\mu = 0$
\item Bose-Einstein condensates: Atoms occupy ground state $\to$ macroscopic coherence
\end{enumerate}
\end{theorem}

\begin{proof}
All three phenomena share common mechanism:

\textbf{Superconductivity}: Electrons form Cooper pairs (bosons) that phase-lock into single categorical state. Partition operations between pairs become undefined: $\taulag \to 0$. Resistivity vanishes: $\rho = 0$.

\textbf{Superfluidity}: Helium-4 atoms (bosons) condense into ground state, forming phase-locked network. Partition operations between atoms in superfluid component become undefined: $\taulag \to 0$. Viscosity vanishes: $\mu = 0$.

\textbf{Bose-Einstein Condensation}: Dilute atomic gases undergo macroscopic occupation of ground state. Partition operations between condensed atoms become undefined: $\taulag \to 0$. Macroscopic wavefunction emerges.

In all cases, categorical unification ($\mathcal{C}_i = \mathcal{C}_j$ for all $i,j$) causes partition extinction ($\taulag \to 0$), yielding dissipationless transport ($\Xi \to 0$).
\end{proof}

\subsection{Connection to Measurement}

\begin{proposition}[Measurement as Partition Operation]
\label{prop:measurement_partition}
Measurement is a partition operation that determines categorical state.
\end{proposition}

\begin{proof}
Measurement divides state space into "measured value = x" and "measured value $\neq$ x". This is partition operation $\Pi: \Cspace \to \Cspace_x \times \Cspace_{\bar{x}}$. Measurement time equals partition lag: $t_{\text{meas}} = \taulag$.
\end{proof}

\begin{corollary}[Fast Measurement]
\label{cor:fast_measurement}
Measurement with small partition lag ($\taulag \to 0$) is dissipationless.
\end{corollary}

\begin{proof}
From Theorem~\ref{thm:dissipation_residue}, dissipation $P \propto \taulag$. As $\taulag \to 0$, dissipation $P \to 0$.
\end{proof}

This establishes theoretical foundation for quantum non-demolition measurement (Section~\ref{sec:qnd_measurement}).
