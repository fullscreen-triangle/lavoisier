\section{Ternary Representation and Geometric Continuity}
\label{sec:ternary_representation}

\subsection{Base-3 Encoding of Partition Coordinates}

\begin{definition}[Ternary Digit (Trit)]
\label{def:trit}
A ternary digit (trit) takes values in $\{0, 1, 2\}$, representing three possible states.
\end{definition}

\begin{definition}[Balanced Ternary]
\label{def:balanced_ternary}
Balanced ternary uses digits $\{-1, 0, +1\}$ instead of $\{0, 1, 2\}$, providing symmetric representation around zero.
\end{definition}

\begin{theorem}[Natural Ternary Encoding]
\label{thm:natural_ternary}
Three-dimensional S-entropy coordinates $(n_x, n_y, n_z)$ naturally encode as ternary numbers.
\end{theorem}

\begin{proof}
\textbf{Coordinate structure}: Each S-entropy coordinate $n_i \in \{0, 1, 2, \ldots\}$ counts partitions along axis $i \in \{x, y, z\}$.

\textbf{Ternary decomposition}: Any integer $n$ decomposes uniquely in base 3:
\begin{equation}
n = \sum_{k=0}^{\infty} t_k 3^k
\end{equation}
where $t_k \in \{0, 1, 2\}$ are trits.

\textbf{Three-dimensional encoding}: Coordinates $(n_x, n_y, n_z)$ encode as three ternary strings:
\begin{align}
n_x &= \sum_{k} t_k^{(x)} 3^k \\
n_y &= \sum_{k} t_k^{(y)} 3^k \\
n_z &= \sum_{k} t_k^{(z)} 3^k
\end{align}

\textbf{Natural correspondence}: Three spatial dimensions $\leftrightarrow$ three trit values. Each trit position $k$ represents partition depth. Trit value represents partition outcome along each axis.
\end{proof}

\subsection{Position as Trajectory}

\begin{theorem}[Position-Trajectory Identity]
\label{thm:position_trajectory}
In ternary representation, position is identical to trajectory.
\end{theorem}

\begin{proof}
\textbf{Trajectory construction}: Start at origin. At partition depth $k$, move in direction determined by trits $(t_k^{(x)}, t_k^{(y)}, t_k^{(z)})$:
\begin{itemize}
\item $t_k^{(i)} = 0$: no movement along axis $i$
\item $t_k^{(i)} = 1$: move positive along axis $i$
\item $t_k^{(i)} = 2$: move negative along axis $i$ (or use balanced ternary: $-1$)
\end{itemize}

Step size at depth $k$: $\Delta_k = 3^{-k}$ (decreasing geometrically).

\textbf{Position after $N$ steps}:
\begin{equation}
\mathbf{r}_N = \sum_{k=0}^{N-1} \Delta_k \hat{\mathbf{d}}_k = \sum_{k=0}^{N-1} 3^{-k} (t_k^{(x)} \hat{\mathbf{x}} + t_k^{(y)} \hat{\mathbf{y}} + t_k^{(z)} \hat{\mathbf{z}})
\end{equation}

\textbf{Ternary representation}:
\begin{equation}
\mathbf{r}_N = \left(\sum_{k=0}^{N-1} t_k^{(x)} 3^{-k}\right) \hat{\mathbf{x}} + \left(\sum_{k=0}^{N-1} t_k^{(y)} 3^{-k}\right) \hat{\mathbf{y}} + \left(\sum_{k=0}^{N-1} t_k^{(z)} 3^{-k}\right) \hat{\mathbf{z}}
\end{equation}

This is ternary expansion of position coordinates. Position is encoded by sequence of trits, which is also the trajectory.

\textbf{Identity}: Position $\mathbf{r}$ and trajectory $\{\mathbf{d}_k\}$ contain identical information. Knowing position determines trajectory uniquely, and vice versa.
\end{proof}

\subsection{Continuity from Discrete Trits}

\begin{theorem}[Ternary Continuity Theorem]
\label{thm:ternary_continuity}
Continuous space emerges from discrete ternary representation in the limit of infinite partition depth.
\end{theorem}

\begin{proof}
\textbf{Discrete representation}: At finite depth $N$, position is discrete:
\begin{equation}
\mathbf{r}_N = \sum_{k=0}^{N-1} t_k^{(i)} 3^{-k} \hat{\mathbf{e}}_i
\end{equation}
with spacing $\Delta_N = 3^{-N}$ between adjacent points.

\textbf{Continuum limit}: As $N \to \infty$, spacing vanishes: $\Delta_N \to 0$. Position becomes continuous:
\begin{equation}
\mathbf{r} = \lim_{N \to \infty} \mathbf{r}_N = \sum_{k=0}^{\infty} t_k^{(i)} 3^{-k} \hat{\mathbf{e}}_i
\end{equation}

\textbf{Completeness}: Every real number $x \in [0, 1]$ has unique ternary expansion:
\begin{equation}
x = \sum_{k=1}^{\infty} t_k 3^{-k}, \quad t_k \in \{0, 1, 2\}
\end{equation}
(except for countable set of endpoints, which have two representations).

\textbf{Topological continuity}: For any $\epsilon > 0$, choose $N$ such that $3^{-N} < \epsilon$. Then positions differing only in trits $k \geq N$ are within $\epsilon$ of each other. This is $\epsilon$-$\delta$ definition of continuity.

\textbf{Conclusion}: Continuous space is limit of discrete ternary representation. Continuity emerges from infinite partition depth.
\end{proof}

\subsection{Geometric Interpretation}

\begin{proposition}[Ternary Partition Geometry]
\label{prop:ternary_geometry}
Each partition step divides space into $3^3 = 27$ cubic cells.
\end{proposition}

\begin{proof}
Three-dimensional space with three partition outcomes per axis: $3 \times 3 \times 3 = 27$ cells. Each cell labeled by trit triple $(t_x, t_y, t_z) \in \{0,1,2\}^3$.
\end{proof}

\begin{theorem}[Self-Similar Structure]
\label{thm:self_similar}
Ternary partition generates self-similar fractal structure.
\end{theorem}

\begin{proof}
\textbf{Scaling symmetry}: At depth $k$, cell size is $\ell_k = 3^{-k}$. At depth $k+1$, each cell subdivides into 27 smaller cells of size $\ell_{k+1} = 3^{-(k+1)} = \ell_k / 3$.

\textbf{Self-similarity}: Structure at depth $k+1$ is identical to structure at depth $k$ under rescaling by factor 3. This is definition of self-similarity.

\textbf{Fractal dimension}: Scaling relation: $N(\ell) = (\ell_0/\ell)^D$ where $N(\ell)$ is number of cells of size $\ell$. For ternary partition: $N(3^{-k}) = 27^k = (3^3)^k = 3^{3k}$. Therefore $D = 3$: fractal dimension equals spatial dimension.

\textbf{Space-filling}: Fractal with dimension equal to embedding dimension is space-filling. Ternary partition fills entire space in limit $k \to \infty$.
\end{proof}

\subsection{Balanced Ternary for Signed Coordinates}

\begin{theorem}[Balanced Ternary Symmetry]
\label{thm:balanced_ternary_symmetry}
Balanced ternary $\{-1, 0, +1\}$ provides natural representation for signed partition coordinates.
\end{theorem}

\begin{proof}
\textbf{Physical interpretation}: Partition outcomes relative to reference point:
\begin{itemize}
\item $t_k = -1$: move in negative direction
\item $t_k = 0$: no movement
\item $t_k = +1$: move in positive direction
\end{itemize}

\textbf{Symmetry}: Balanced ternary is symmetric around zero. Negation simply flips signs: $-n \leftrightarrow$ flip all trits. Standard ternary lacks this symmetry.

\textbf{Signed coordinates}: Partition coordinates can be positive or negative (e.g., magnetic quantum number $m \in \{-\ell, \ldots, +\ell\}$). Balanced ternary naturally represents signed values without separate sign bit.

\textbf{Arithmetic}: Addition in balanced ternary is simpler than standard ternary (no carries beyond immediate neighbors).
\end{proof}

\subsection{Trit Operations}

\begin{definition}[Trit Addition]
\label{def:trit_addition}
Balanced ternary addition follows rules:
\begin{align}
0 + 0 &= 0 \\
0 + 1 &= 1 \\
1 + 1 &= \overline{1}1 \quad \text{(carry)} \\
1 + (-1) &= 0
\end{align}
\end{definition}

\begin{definition}[Trit Multiplication]
\label{def:trit_multiplication}
Balanced ternary multiplication follows rules:
\begin{align}
0 \times a &= 0 \\
1 \times a &= a \\
(-1) \times a &= -a
\end{align}
\end{definition}

\begin{proposition}[Trit Computation Efficiency]
\label{prop:trit_efficiency}
Ternary computation requires fewer digits than binary for same numeric range.
\end{proposition}

\begin{proof}
\textbf{Digit count}: To represent number $N$:
\begin{itemize}
\item Binary: $\lceil \log_2 N \rceil$ bits
\item Ternary: $\lceil \log_3 N \rceil$ trits
\end{itemize}

\textbf{Ratio}: $\frac{\log_3 N}{\log_2 N} = \frac{\ln N / \ln 3}{\ln N / \ln 2} = \frac{\ln 2}{\ln 3} \approx 0.631$.

Ternary uses $\sim$63\% as many digits as binary.

\textbf{Information content}: Each bit carries $\log_2 2 = 1$ bit of information. Each trit carries $\log_2 3 \approx 1.585$ bits of information. Ternary is more information-dense.
\end{proof}

\subsection{Coordinate Transformation}

\begin{theorem}[Ternary-Cartesian Transformation]
\label{thm:ternary_cartesian}
Ternary partition coordinates transform to Cartesian coordinates via:
\begin{equation}
\mathbf{r} = \sum_{k=0}^{\infty} 3^{-k} \mathbf{t}_k
\end{equation}
where $\mathbf{t}_k = (t_k^{(x)}, t_k^{(y)}, t_k^{(z)})$ is trit vector at depth $k$.
\end{theorem}

\begin{proof}
Direct consequence of Theorem~\ref{thm:position_trajectory}. Ternary representation is partition coordinate system. Cartesian coordinates obtained by summing partition steps.
\end{proof}

\begin{corollary}[Inverse Transformation]
\label{cor:inverse_transformation}
Cartesian coordinates transform to ternary via successive partition:
\begin{equation}
t_k^{(i)} = \lfloor 3^{k+1} r_i \rfloor \bmod 3
\end{equation}
\end{corollary}

\begin{proof}
Extract $k$-th trit by scaling coordinate to appropriate magnitude, taking integer part, and reducing modulo 3.
\end{proof}

\subsection{Velocity and Momentum in Ternary}

\begin{theorem}[Ternary Velocity]
\label{thm:ternary_velocity}
Velocity in ternary representation is:
\begin{equation}
\mathbf{v} = \frac{d\mathbf{r}}{dt} = \sum_{k=0}^{\infty} 3^{-k} \frac{d\mathbf{t}_k}{dt}
\end{equation}
\end{theorem}

\begin{proof}
Differentiate ternary position (Theorem~\ref{thm:ternary_cartesian}) with respect to time. Trits $\mathbf{t}_k$ are time-dependent for dynamical systems.
\end{proof}

\begin{proposition}[Trit Flip Rate]
\label{prop:trit_flip_rate}
Trit flip rate at depth $k$ is $\Gamma_k = 3^k \Gamma_0$ where $\Gamma_0$ is fundamental partition rate.
\end{proposition}

\begin{proof}
Partition rate increases with depth due to smaller cell size. Cell size: $\ell_k = 3^{-k}$. Crossing time: $\tau_k = \ell_k / v \sim 3^{-k}$. Flip rate: $\Gamma_k = \tau_k^{-1} \sim 3^k$.
\end{proof}

\begin{theorem}[Momentum Quantization]
\label{thm:momentum_quantization}
Momentum in ternary representation is quantized:
\begin{equation}
\mathbf{p} = \hbar \sum_{k=0}^{\infty} 3^k \mathbf{t}_k
\end{equation}
\end{theorem}

\begin{proof}
\textbf{De Broglie relation}: $\mathbf{p} = \hbar \mathbf{k}$ where $\mathbf{k}$ is wave vector.

\textbf{Partition wave vector}: At depth $k$, cell size is $\ell_k = 3^{-k}$. Associated wave vector: $k_k = 2\pi / \ell_k = 2\pi \cdot 3^k$.

\textbf{Trit contribution}: Each trit $t_k$ contributes wave vector $\mathbf{k}_k = 3^k \mathbf{t}_k$ (in units of $2\pi$).

\textbf{Total momentum}: $\mathbf{p} = \hbar \sum_k \mathbf{k}_k = \hbar \sum_k 3^k \mathbf{t}_k$.

This is ternary representation of momentum, dual to ternary position.
\end{proof}

\subsection{Uncertainty Relation}

\begin{theorem}[Ternary Uncertainty Principle]
\label{thm:ternary_uncertainty}
Position and momentum uncertainties in ternary representation satisfy:
\begin{equation}
\Delta x \cdot \Delta p \geq \frac{\hbar}{2}
\end{equation}
\end{theorem}

\begin{proof}
\textbf{Position uncertainty}: Truncating ternary expansion at depth $N$ gives uncertainty $\Delta x \sim 3^{-N}$.

\textbf{Momentum uncertainty}: Truncating momentum expansion at depth $N$ gives uncertainty $\Delta p \sim \hbar \cdot 3^N$.

\textbf{Product}: $\Delta x \cdot \Delta p \sim 3^{-N} \cdot \hbar \cdot 3^N = \hbar$.

Exact coefficient depends on trit distribution, giving $\Delta x \cdot \Delta p \geq \hbar/2$ (Heisenberg uncertainty principle).

\textbf{Interpretation}: Cannot simultaneously specify all position trits (small $\Delta x$) and all momentum trits (small $\Delta p$). Specifying position trits to depth $N$ leaves momentum trits at depth $> N$ uncertain.
\end{proof}

\subsection{Connection to Quantum Mechanics}

\begin{theorem}[Ternary Quantum States]
\label{thm:ternary_quantum}
Quantum wavefunctions are ternary superpositions:
\begin{equation}
|\psi\rangle = \sum_{\{\mathbf{t}_k\}} c_{\{\mathbf{t}_k\}} |\{\mathbf{t}_k\}\rangle
\end{equation}
where $|\{\mathbf{t}_k\}\rangle$ are ternary basis states.
\end{equation}
\end{theorem}

\begin{proof}
\textbf{Position basis}: $|\mathbf{r}\rangle$ with $\mathbf{r} = \sum_k 3^{-k} \mathbf{t}_k$ (Theorem~\ref{thm:ternary_cartesian}).

\textbf{Ternary basis}: $|\{\mathbf{t}_k\}\rangle$ labels state by sequence of trits. This is complete basis for position space.

\textbf{Wavefunction}: $\psi(\mathbf{r}) = \langle \mathbf{r} | \psi \rangle = \sum_{\{\mathbf{t}_k\}} c_{\{\mathbf{t}_k\}} \delta(\mathbf{r} - \mathbf{r}_{\{\mathbf{t}_k\}})$ where $\mathbf{r}_{\{\mathbf{t}_k\}} = \sum_k 3^{-k} \mathbf{t}_k$.

\textbf{Superposition}: General state is superposition over all trit sequences. Coefficients $c_{\{\mathbf{t}_k\}}$ are complex amplitudes.
\end{proof}

\subsection{Computational Advantages}

\begin{proposition}[Ternary Quantum Computing]
\label{prop:ternary_quantum_computing}
Ternary representation enables efficient quantum simulation.
\end{proposition}

\begin{proof}
\textbf{Qubit vs qutrit}: Standard quantum computing uses qubits (2-level systems). Ternary quantum computing uses qutrits (3-level systems).

\textbf{Efficiency}: Qutrit carries $\log_2 3 \approx 1.585$ bits of information vs 1 bit for qubit. Speedup factor: $\sim$1.585.

\textbf{Natural encoding}: Partition coordinates naturally encode as qutrits. No conversion needed between physical system and computational representation.

\textbf{Gate operations}: Ternary gates (Hadamard, phase, CNOT) directly implement partition operations. One-to-one correspondence between physics and computation.
\end{proof}

This establishes ternary representation as natural mathematical framework unifying discrete partition operations and continuous geometry.
