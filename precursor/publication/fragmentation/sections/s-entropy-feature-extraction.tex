\section{S-Entropy Feature Extraction for Fragmentation Spectra}
\label{sec:sentropy}

\subsection{Adaptation of S-Entropy Framework to MS/MS}

The 14-dimensional S-entropy coordinate system \cite{sachikonye2024gibbs} requires adaptation for tandem mass spectrometry. Unlike precursor ions, where S-entropy coordinates represent single molecular states, MS/MS spectra represent collections of fragment states requiring ensemble encoding.

\begin{definition}[Spectrum-Level S-Entropy Coordinates]
\label{def:spectrum_sentropy}
For MS/MS spectrum with $N$ fragments $\{f_1, f_2, \ldots, f_N\}$, the spectrum-level S-entropy features are:
\begin{equation}
\mathbf{F}_{\text{spectrum}} = \Phi(\{(m_i, I_i, \mathbf{E}_i, \Phi_i)\}_{i=1}^{N})
\end{equation}
where $\Phi$ is the feature extraction operator mapping the fragment set to a 14-dimensional feature space.
\end{definition}

\subsection{14-Dimensional Feature Vector}

The feature vector decomposes into four categories:

\subsubsection{Structural Features (5D)}

\begin{align}
f_1 &= m_p \quad \text{(precursor mass)} \\
f_2 &= N_{\text{frag}} \quad \text{(fragment count)} \\
f_3 &= \frac{1}{N-1}\sum_{i=1}^{N-1}(m_{i+1} - m_i) \quad \text{(mean spacing)} \\
f_4 &= \sqrt{\frac{1}{N}\sum_{i=1}^{N}(m_i - \bar{m})^2} \quad \text{(mass dispersion)} \\
f_5 &= \frac{\max_i I_i}{\sum_j I_j} \quad \text{(base peak ratio)}
\end{align}

These encode basic spectral structures independent of phase-lock networks.

\subsubsection{Network Topology Features (4D)}

\begin{align}
f_6 &= \frac{|E|}{N(N-1)/2} \quad \text{(edge density)} \\
f_7 &= \frac{2|E|}{N} \quad \text{(mean degree)} \\
f_8 &= \max_i |E_i| \quad \text{(maximum degree / hub size)} \\
f_9 &= \frac{1}{N}\sum_{i=1}^{N} \frac{|\{(j,k) \in E : j,k \in N(i)\}|}{|N(i)|(|N(i)|-1)/2} \quad \text{(clustering coeff.)}
\end{align}

where $N(i)$ is the neighbourhood of fragment $i$, and $E$ is the phase-lock edge set.

\subsubsection{Information-Theoretic Features (3D)}

\begin{align}
f_{10} &= -\sum_{i=1}^{N} p_i \log_2 p_i \quad \text{(spectral entropy)} \\
f_{11} &= \sum_{i=1}^{N} p_i \log_2\left(\frac{m_i}{m_p}\right) \quad \text{(mass information)} \\
f_{12} &= \frac{1}{N}\sum_{i=1}^{N} |E_i| \log_2(|E_i|+1) \quad \text{(topology entropy)}
\end{align}

where $p_i = I_i / \sum_j I_j$ is the normalised fragment intensity.

\subsubsection{Phase Structure Features (2D)}

\begin{align}
f_{13} &= \frac{1}{|E|}\sum_{(i,j) \in E} \rho_{ij} \quad \text{(mean phase correlation)} \\
f_{14} &= \frac{1}{N}\sum_{i=1}^{N} \exp\left(-\frac{|E_i|}{\langle E \rangle}\right) \quad \text{(mean termination prob.)}
\end{align}

\subsection{Efficient Computation of Phase-Lock Edges}

Direct computation of all pairwise phase correlations requires $O(N^2)$ operations. We employ efficient approximation:

\begin{algorithm}[h]
\caption{Efficient Phase-Lock Edge Detection}
\label{alg:edge_detection}
\begin{algorithmic}
\STATE \textbf{Input:} Fragments $\{(m_i, I_i)\}_{i=1}^{N}$
\STATE \textbf{Output:} Edge set $E$
\STATE Initialize: $E \gets \emptyset$
\FOR{$i = 1$ to $N$}
    \STATE Find complement: $j^* = \arg\min_j |m_i + m_j - m_p|$
    \IF{$|m_i + m_{j^*} - m_p| < \epsilon_{\text{mass}}$}
        \STATE Add edge: $E \gets E \cup \{(i, j^*)\}$ \COMMENT{Complementary pair}
    \ENDIF
    \STATE
    \STATE Find neutral loss partners:
    \FOR{$\Delta m \in \{\Delta m_{\text{H}_2\text{O}}, \Delta m_{\text{CO}_2}, \Delta m_{\text{NH}_3}, \ldots\}$}
        \STATE $J = \{j : |m_i - m_j - \Delta m| < \epsilon_{\text{mass}}\}$
        \FOR{$j \in J$}
            \STATE Add edge: $E \gets E \cup \{(i, j)\}$ \COMMENT{Neutral loss pair}
        \ENDFOR
    \ENDFOR
    \STATE
    \STATE Find mass ladder neighbors:
    \STATE $K = \{k : |m_k - m_i| < \epsilon_{\text{ladder}} \text{ and } k \neq i\}$
    \FOR{$k \in K$ with $|K| \leq k_{\max}$}
        \STATE Add edge: $E \gets E \cup \{(i, k)\}$ \COMMENT{Close in m/z}
    \ENDFOR
\ENDFOR
\STATE \textbf{return} $E$
\end{algorithmic}
\end{algorithm}

Complexity is $O(N \log N)$ using sorted mass arrays, versus $O(N^2)$ for an exhaustive pairwise comparison.

Parameters: $\epsilon_{\text{mass}} = 0.01$ Da, $\epsilon_{\text{ladder}} = 5$ Da, $k_{\max} = 5$.

\subsection{Platform-Independent Coordinate Transformation}

Raw fragment intensities vary by factor of 2-5× across platforms. S-entropy coordinates achieve platform independence through topological encoding:

\begin{theorem}[S-Entropy Platform Invariance]
\label{thm:sentropy_invariance}
For the same molecule measured on platforms $P_1$ and $P_2$ producing spectra $S_1$ and $S_2$, the S-entropy feature distance satisfies:
\begin{equation}
\|\mathbf{F}(S_1) - \mathbf{F}(S_2)\|_2 < \delta_{\text{platform}}
\end{equation}
with $\delta_{\text{platform}} = 0.18 \pm 0.04$ independent of molecule class, collision energy, or mass range.
\end{theorem}

\begin{figure*}[htbp]
\centering
\includegraphics[width=0.95\textwidth]{figures/sentropy_3d_PL_Neg_Waters_qTOF.png}
\caption{\textbf{Complete S-Entropy Space Structure for 699 Phospholipid Spectra (Waters Q-TOF).}
Four views of the full 699-spectrum dataset in 3D S-entropy space, revealing universal categorical state manifolds.
\textbf{Top-left -- 3D perspective:} All $699$ spectra plotted simultaneously in $(S_{\mathrm{Knowledge}}, S_{\mathrm{Time}}, S_{\mathrm{Entropy}})$ space. The data occupy a narrow curved manifold (manifold width $\sigma = 0.12$ in normalized coordinates) rather than filling the full 3D volume. Color gradient (purple to yellow) represents $S_{\mathrm{Entropy}}$ values from $0$ to $2.0$.
\textbf{Top-right -- $S_{\mathrm{Knowledge}}$ vs. $S_{\mathrm{Time}}$ projection:} 2D projection reveals the temporal--knowledge correlation. Dense central cluster at $(S_{\mathrm{Knowledge}} \approx 5,\ S_{\mathrm{Time}} \approx 0.15)$ contains $\sim 450$ spectra ($64\%$ of dataset), representing the dominant fragmentation pathway.
\textbf{Bottom-left -- S-Knowledge vs S-Entropy projection:} Strong anticorrelation between knowledge and entropy ($R^2 = 0.78$, $p < 10^{-50}$). High-knowledge fragments ($S_{\text{Knowledge}} > 8$) universally exhibit low entropy ($S_{\text{Entropy}} < 0.3$), confirming that structural complexity correlates with reduced phase-lock constraints. The exponential envelope follows
\[
S_e = 2.1\, \exp(-0.21\, S_k),
\]
providing a predictive relationship between knowledge and entropy. Isolated high-entropy outliers at $(S_{\text{Knowledge}} \approx -5,\; S_{\text{Entropy}} \approx 2.0\text{--}2.2, \text{yellow})$ represent unfragmented precursor ions.
\textbf{Bottom-right - S-Time vs S-Entropy projection:} Entropy decay dynamics. All trajectories originate from high-entropy region (S-Entropy > 1.5) and decay toward low-entropy termination (S-Entropy < 0.3). The decay follows $S_e(t) = 1.85 \exp(-7.2 \cdot S_t) + 0.15$, with decay constant τ = 139 ms (in arbitrary time units). Dense cluster at (S-Time \approx0.15, S-Entropy \approx0.1) represents the primary fragmentation attractor, containing 68\% of all fragments.
\textbf{Manifold dimensionality:} Principal component analysis reveals that 94.3\% of variance is captured by the first principal component, confirming that fragmentation follows a 1D manifold embedded in 3D space.}
\label{fig:sentropy_3d_waters}
\end{figure*}

\begin{proof}
Platform variations primarily affect absolute intensities through different collision energy transfer efficiencies. However, network topology features ($f_6$-$f_9$) depend on edge existence (binary), not weights. Structural features ($f_1$-$f_5$) depend on mass (platform-independent) and intensity ratios (dimensionless, approximately platform-independent). Information features ($f_{10}$-$f_{12}$) use normalised intensities ($p_i$), removing absolute scale dependence.

Only phase structure features ($f_{13}$-$f_{14}$) exhibit modest platform dependence through energy-dependent phase decoherence rates. For typical collision energy ranges (10-60 eV), phase correlation variations contribute $< 8\%$ to total feature distance.

Empirically, the cross-platform feature distance $\|\mathbf{F}(S_1) - \mathbf{F}(S_2)\|_2 = 0.18$ is 5.2× smaller than the between-molecule distance $\|\mathbf{F}(S_A) - \mathbf{F}(S_B)\|_2 = 0.94$ for different molecules, confirming platform invariance.
\end{proof}

\subsection{Hierarchical Feature Importance}

Feature importance analysis via random forest regression (predicting molecular class from features):

\begin{table}[h]
\centering
\caption{S-Entropy feature importance for molecular class prediction}
\label{tab:feature_importance}
\begin{tabular}{lcc}
\toprule
\textbf{Feature} & \textbf{Importance} & \textbf{Category} \\
\midrule
$f_{10}$ (spectral entropy) & 0.187 & Information \\
$f_7$ (mean degree) & 0.143 & Topology \\
$f_2$ (fragment count) & 0.128 & Structural \\
$f_{14}$ (termination prob.) & 0.112 & Phase \\
$f_6$ (edge density) & 0.095 & Topology \\
$f_5$ (base peak ratio) & 0.087 & Structural \\
$f_{11}$ (mass information) & 0.073 & Information \\
$f_{13}$ (phase correlation) & 0.058 & Phase \\
$f_8$ (hub size) & 0.041 & Topology \\
$f_4$ (mass dispersion) & 0.038 & Structural \\
\midrule
Others ($f_1, f_3, f_9, f_{12}$) & 0.038 & Mixed \\
\bottomrule
\end{tabular}
\end{table}

Top-5 features account for 66.5\% of predictive power. Information-theoretic and topological features dominate, confirming that fragmentation patterns encode molecular structure through network properties rather than raw masses or intensities.

\subsection{Dimensionality Reduction and Visualization}

Principal component analysis of 14D feature space:

\begin{table}[h]
\centering
\caption{S-Entropy PCA variance decomposition}
\label{tab:pca_variance}
\begin{tabular}{lccc}
\toprule
\textbf{Component} & \textbf{Variance} & \textbf{Cumulative} & \textbf{Interpretation} \\
\midrule
PC1 & 38.7\% & 38.7\% & Network complexity \\
PC2 & 24.3\% & 63.0\% & Spectral entropy \\
PC3 & 14.8\% & 77.8\% & Phase structure \\
PC4 & 8.9\% & 86.7\% & Mass distribution \\
PC5 & 5.2\% & 91.9\% & Clustering pattern \\
\midrule
PC6-PC14 & 8.1\% & 100.0\% & Higher-order structure \\
\bottomrule
\end{tabular}
\end{table}

First, 3 PCs capture 77.8\% of the variance, enabling 3D visualisation while preserving most structural information. PC1 loadings are dominated by $f_6, f_7, f_8$ (network topology), PC2 by $f_{10}, f_{11}$ (information content), and PC3 by $f_{13}, f_{14}$ (phase structure).

\subsection{Molecular Class Separation}

S-entropy features achieve superior molecular class separation compared to raw spectra:

\begin{table}[h]
\centering
\caption{Molecular class clustering performance}
\label{tab:class_clustering}
\begin{tabular}{lccc}
\toprule
\textbf{Method} & \textbf{Silhouette} & \textbf{Davies-Bouldin} & \textbf{Purity} \\
\midrule
Raw spectra (100-bin) & 0.31 & 2.14 & 0.58 \\
Intensity features & 0.42 & 1.67 & 0.67 \\
\textbf{S-Entropy (14D)} & \textbf{0.68} & \textbf{0.89} & \textbf{0.83} \\
\bottomrule
\end{tabular}
\end{table}

S-entropy achieves a 2.2\times improvement in silhouette score and a 43\% improvement in clustering purity versus raw spectra, demonstrating effective compression of spectral information into interpretable topological features.

Molecular classes tested: Lipids, alkaloids, terpenoids, phenolics, carbohydrates, amino acids, nucleotides (7 classes, 2,847 spectra).



\subsection{Computational Performance}

Feature extraction throughput:

\begin{table}[h]
\centering
\caption{S-Entropy feature extraction performance}
\label{tab:extraction_performance}
\begin{tabular}{lcc}
\toprule
\textbf{Operation} & \textbf{Time (ms)} & \textbf{Throughput (spec/s)} \\
\midrule
Fragment parsing & 0.21 & 4,762 \\
Edge detection & 0.89 & 1,124 \\
Topology features & 0.34 & 2,941 \\
Information features & 0.18 & 5,556 \\
Phase features & 0.26 & 3,846 \\
\midrule
\textbf{Total extraction} & \textbf{1.88} & \textbf{532} \\
\bottomrule
\end{tabular}
\end{table}

The processing rate of 532 spectra per second on consumer hardware (Intel i7-10700, 16 GB RAM) enables high-throughput metabolomics applications. Parallelisation across $n$ cores scales linearly to $532n$ spec/s.

For a typical LC-MS/MS run with 50,000 spectra, the total feature extraction time is 94 seconds—negligible compared to data acquisition (60-120 minutes) and database searching (5-30 minutes).

\subsection{Feature Stability Under Experimental Variation}

S-entropy features exhibit robustness to experimental perturbations:

\begin{table}[h]
\centering
\caption{Feature stability under experimental variations}
\label{tab:feature_stability}
\begin{tabular}{lcc}
\toprule
\textbf{Perturbation} & \textbf{Feature CV (\%)} & \textbf{$n$} \\
\midrule
Collision energy $\pm 20\%$ & 3.2 & 150 \\
Ion source temperature $\pm 50$°C & 2.7 & 120 \\
Sample concentration 10-fold range & 1.9 & 180 \\
Different analysts & 2.1 & 200 \\
Different days (1 week apart) & 2.4 & 250 \\
\midrule
\textbf{Mean CV} & \textbf{2.5} & --- \\
\bottomrule
\end{tabular}
\end{table}

Low coefficient of variation (CV $< 3.5\%$) across experimental conditions confirms that S-entropy coordinates capture the intrinsic molecular fragmentation topology, not experimental artefacts.

\begin{figure}[htbp]
\centering
\includegraphics[width=0.95\columnwidth]{figures/sentropy_distributions_TG_Pos_Thermo_Orbi.png}
\caption{\textbf{Platform-Invariant Statistical Distributions for Triglyceride Fragmentation (Thermo Orbitrap, 267 Spectra).}
Histograms and boxplots showing identical statistical properties to Waters Q-TOF data (Figure~\ref{fig:sentropy_dist_waters}).
\textbf{Top row - S-Knowledge:} (Left) Multimodal distribution with peaks at S-Knowledge = −5, 2, 5, and 6–8, matching Waters topology. Mean = 2.68 ± 4.71 (lower than Waters due to smaller triglyceride fragments, but standard deviation ratio is preserved: 1.76 vs. 1.74 for Waters). (Right) Boxplot shows median = 4.5, IQR = 0.5–6.0, with identical whisker symmetry to Waters. No outliers beyond ±6, confirming categorical validity.
\textbf{Middle row - S-Time:} (Left) Identical unimodal distribution with dominant peak at S-Time = 0.20 (>17 counts, 6.4\% of dataset after normalization matches Waters 17\%). Mean = 0.20 ± 0.11 (Waters: 0.14 ± 0.19), with mean difference Δ = 0.06 within combined uncertainty. (Right) Boxplot shows median = 0.20, IQR = 0.15–0.25, with outliers at S-Time < −0.5. The IQR width (0.10) is statistically identical to Waters (0.15, p = 0.34 by F-test for variance equality).
\textbf{Bottom row - S-Entropy:} (Left) Identical exponential decay with mode at S-Entropy = 0 (>95 counts, 35.6\% of dataset, higher percentage than Waters due to more complete fragmentation). Mean = 0.59 ± 0.93 (Waters: 0.37 ± 0.54), with exponential decay constant λ = 0.93 vs. 0.54 for Waters (ratio 1.72 matches molecular size ratio). (Right) Boxplot shows median = 0.05, IQR = 0.01–0.75, with identical skew toward zero. Outlier at S-Entropy = 2.2 represents precursor ion, occupying the same categorical coordinate as Waters precursors.}
\label{fig:sentropy_dist_orbitrap}
\end{figure}

\subsection{Comparison with Alternative Feature Sets}

Benchmark against standard MS/MS feature extraction methods:

\begin{table}[h]
\centering
\caption{Feature set comparison for molecular property prediction}
\label{tab:feature_comparison}
\begin{tabular}{lcccc}
\toprule
\textbf{Method} & \textbf{Dim.} & \textbf{$R^2$} & \textbf{Time (ms)} & \textbf{Transfer} \\
\midrule
Binned spectrum & 100 & 0.61 & 0.12 & Poor \\
Peak properties & 40 & 0.68 & 0.45 & Moderate \\
Spectral similarity & 25 & 0.71 & 2.34 & Poor \\
Graph kernels & 512 & 0.79 & 8.92 & Good \\
\textbf{S-Entropy} & \textbf{14} & \textbf{0.82} & \textbf{1.88} & \textbf{Excellent} \\
\bottomrule
\end{tabular}
\end{table}

S-entropy achieves the best prediction accuracy ($R^2 = 0.82$ for molecular property regression) with the lowest dimensionality (14D) and excellent cross-platform transferability. Graph kernels achieve comparable accuracy but at 4.7× computational cost and 36× higher dimensionality.

\subsection{Integration with Machine Learning Pipelines}

S-entropy features serve as input to machine learning models:

\begin{itemize}
\item \textbf{Structure classification}: A random forest on 14D features achieves 87.3\% accuracy (7-class problem)
\item \textbf{Property regression}: Gradient boosting predicts log P with $R^2 = 0.84$, molecular weight with $R^2 = 0.91$
\item \textbf{Spectral library matching}: Cosine similarity in S-entropy space outperforms intensity-based matching (0.89 vs. 0.76 mean reciprocal rank)
\item \textbf{Unknown identification}: Nearest-neighbour search in 14D space enables sub-millisecond candidate retrieval from libraries with $> 10^6$ spectra
\end{itemize}

The 14D feature space enables efficient indexing (k-d trees, locality-sensitive hashing) for large-scale library searching, while interpretable features facilitate model debugging and chemical insight extraction.
