\section{Poincar\'{e} Computing}
\label{sec:poincare-computing}

\subsection{Poincar\'{e} Recurrence}

\begin{theorem}[Poincar\'{e} Recurrence Theorem]
In a bounded phase space with volume-preserving dynamics, almost every trajectory returns arbitrarily close to its initial point.
\end{theorem}

This classical result from dynamical systems theory provides the foundation for interpreting compound identification as trajectory completion.

\subsection{Identification as Recurrence}

\begin{definition}[Partition Trajectory]
A partition trajectory is a sequence of partition coordinates:
\begin{equation}
    \mathcal{T} = \{(n_0, l_0, m_0, s_0), (n_1, l_1, m_1, s_1), \ldots, (n_k, l_k, m_k, s_k)\}
\end{equation}
where consecutive entries satisfy selection rules.
\end{definition}

\begin{definition}[Trajectory Closure]
A trajectory closes if there exists a chemically consistent endpoint:
\begin{equation}
    \sum_i m(n_i, l_i) = M_{\text{precursor}}
\end{equation}
where $m(n, l)$ is the mass associated with coordinates $(n, l)$.
\end{definition}

\begin{theorem}[Identification as Trajectory Closure]
\label{thm:id-closure}
Compound identification is equivalent to finding a closed trajectory through measured partition coordinates.
\end{theorem}

\begin{proof}
A compound is characterized by its precursor mass and fragmentation pattern. The fragmentation pattern defines a set of partition coordinates. A valid identification corresponds to a trajectory through these coordinates that:
\begin{enumerate}
    \item Starts at the precursor
    \item Passes through all measured fragments
    \item Closes (mass is conserved)
\end{enumerate}
This is exactly the condition for trajectory closure.
\end{proof}

\subsection{Poincar\'{e} Machine}

\begin{definition}[Poincar\'{e} Machine]
A Poincar\'{e} machine is a computational device that solves problems by finding closed trajectories in phase space.
\end{definition}

\begin{proposition}[Mass Spectrometer as Poincar\'{e} Machine]
A mass spectrometer implementing the virtual instrument ensemble is a Poincar\'{e} machine: it solves identification by trajectory closure.
\end{proposition}

\subsection{Poincar\'{e} Complexity}

\begin{definition}[Poincar\'{e} Complexity]
The Poincar\'{e} complexity $\Pi(C)$ of a compound $C$ is the minimum number of measurements required for trajectory closure:
\begin{equation}
    \Pi(C) = \min \{k : \exists \text{ unique closed trajectory through } k \text{ measurements}\}
\end{equation}
\end{definition}

\begin{proposition}[Complexity Bounds]
For a compound with $F$ fragments at maximum depth $n$:
\begin{equation}
    \lceil \log_2 C(n) \rceil \leq \Pi(C) \leq F
\end{equation}
\end{proposition}

\begin{proof}
Lower bound: At least $\log_2 C(n)$ bits are required to specify a state among $C(n) = 2n^2$ possibilities.
Upper bound: Measuring all fragments guarantees closure.
\end{proof}

\subsection{Information Routing}

\begin{definition}[Information Gain]
The information gain from measurement $M_i$ is
\begin{equation}
    I(M_i) = H(\mathcal{T}_{\text{before}}) - H(\mathcal{T}_{\text{after}})
\end{equation}
where $H(\mathcal{T})$ is the entropy over trajectory space.
\end{definition}

\begin{algorithm}
\caption{Optimal Measurement Routing}
\label{alg:routing}
\begin{algorithmic}[1]
\Require Current trajectory distribution $P(\mathcal{T})$, available measurements $\{M_i\}$
\Ensure Next measurement $M^*$
\For{each measurement $M_i$}
    \State Compute expected posterior: $P(\mathcal{T} | M_i)$
    \State Compute expected information gain: $I_i = \mathbb{E}[H_{\text{before}} - H_{\text{after}}]$
\EndFor
\State $M^* \gets \arg\max_i I_i$
\State \Return $M^*$
\end{algorithmic}
\end{algorithm}

\subsection{Convergence Dynamics}

\begin{theorem}[Convergence Rate]
With optimal measurement routing, the expected number of measurements to trajectory closure is
\begin{equation}
    \mathbb{E}[k] = O(\Pi(C) \cdot \log \Pi(C))
\end{equation}
\end{theorem}

\begin{proof}
Each optimally chosen measurement reduces trajectory entropy by at least a constant factor. The number of halvings to reach certainty from $C(n)$ initial states is $O(\log C(n)) = O(\log n)$. Combined with the minimum $\Pi(C)$ measurements, total is $O(\Pi(C) \log \Pi(C))$.
\end{proof}

\subsection{Multi-Modal Fusion}

\begin{definition}[Modal Projection]
Each analytical modality (MS, NMR, IR, etc.) implements a projection $\pi_\alpha$ onto a subspace of partition coordinates.
\end{definition}

\begin{theorem}[Optimal Fusion]
Multi-modal fusion is optimal when modalities project onto orthogonal subspaces:
\begin{equation}
    \pi_\alpha \perp \pi_\beta \implies I(\alpha, \beta) = I(\alpha) + I(\beta)
\end{equation}
\end{theorem}

\begin{proof}
Orthogonal projections provide independent information. Information is additive for independent sources.
\end{proof}

\begin{corollary}[Fusion Strategy]
For maximum information gain, select modalities whose projections span partition coordinate space with minimum overlap.
\end{corollary}

\subsection{De Novo Identification}

\begin{algorithm}
\caption{De Novo Identification via Poincar\'{e} Computing}
\label{alg:denovo}
\begin{algorithmic}[1]
\Require Measurements $\{M_1, \ldots, M_k\}$, molecular formula $F$
\Ensure Candidate structures $\{C_1, \ldots, C_m\}$
\State Extract partition coordinates from each measurement
\State Build trajectory graph $G$: nodes are coordinates, edges satisfy selection rules
\State Find all closed paths in $G$ consistent with $F$
\State For each closed path $P$:
    \State \quad Reconstruct molecular graph from path topology
    \State \quad Check chemical validity (valence, ring strain, etc.)
    \State \quad Score by trajectory likelihood
\State \Return top-scoring valid structures
\end{algorithmic}
\end{algorithm}

\subsection{Computational Interpretation}

The Poincar\'{e} computing framework provides a new perspective on analytical chemistry:

\begin{itemize}
    \item \textbf{Measurement} = Projection onto partition coordinate subspace
    \item \textbf{Identification} = Trajectory closure in partition space
    \item \textbf{Structure elucidation} = Trajectory topology determination
    \item \textbf{Method development} = Projection optimization
\end{itemize}

This interpretation unifies diverse analytical techniques under a common mathematical framework, enabling systematic optimization and integration.

