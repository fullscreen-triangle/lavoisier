\section{Hardware Oscillatory Mechanics}
\label{sec:hardware-oscillatory-mechanics}

\subsection{Oscillator Classification}

\begin{definition}[Hardware Oscillator]
A hardware oscillator is a physical system exhibiting periodic dynamics with characteristic frequency $\nu$, quality factor $Q$, and phase $\phi$. The state at time $t$ is
\begin{equation}
    x(t) = A \cos(2\pi\nu t + \phi)
\end{equation}
where $A$ is the amplitude.
\end{definition}

Hardware oscillators in analytical chemistry include:

\begin{table}[h]
\centering
\caption{Hardware oscillators in analytical instrumentation}
\begin{tabular}{llll}
\toprule
Oscillator Type & Frequency Range & $Q$ (typical) & Application \\
\midrule
RF field (quadrupole) & 0.5--3 MHz & $10^2$--$10^3$ & Ion confinement \\
Ion secular motion & 50--500 kHz & $10^3$--$10^4$ & Mass analysis \\
Orbitrap axial & 100--400 kHz & $10^4$--$10^5$ & High-resolution MS \\
NMR precession & 60--1000 MHz & $10^6$--$10^8$ & Structure determination \\
IR molecular vibration & $10^{13}$--$10^{14}$ Hz & $10^2$--$10^3$ & Functional group ID \\
\bottomrule
\end{tabular}
\end{table}

\subsection{Oscillation Hierarchy}

\begin{definition}[Oscillation Hierarchy]
An oscillation hierarchy $\mathcal{H} = \{\nu_1, \nu_2, \ldots, \nu_k\}$ is an ordered sequence of frequencies with $\nu_1 > \nu_2 > \cdots > \nu_k$, each corresponding to a distinct physical process.
\end{definition}

\begin{example}[Quadrupole Ion Trap Hierarchy]
\begin{align}
    \nu_1 &= 1 \text{ MHz (RF drive)} \\
    \nu_2 &= 100 \text{ kHz (axial secular)} \\
    \nu_3 &= 50 \text{ kHz (radial secular)} \\
    \nu_4 &= 10 \text{ kHz (micromotion beat)}
\end{align}
\end{example}

\begin{proposition}[Hierarchy Separability]
For a separable hierarchy with $\nu_i / \nu_{i+1} > Q_i$, each oscillation level can be analyzed independently.
\end{proposition}

\begin{proof}
The frequency resolution of oscillator $i$ is $\delta\nu_i = \nu_i / Q_i$. Independence requires $\nu_i - \nu_{i+1} > \delta\nu_i$, giving $\nu_i / \nu_{i+1} > Q_i / (Q_i - 1) \approx 1 + 1/Q_i$.
\end{proof}

\subsection{Timing Precision}

\begin{definition}[Timing Precision]
The timing precision of an oscillator is
\begin{equation}
    \delta t = \frac{1}{Q \nu}
\end{equation}
the minimum distinguishable time interval.
\end{definition}

\begin{theorem}[Timing-Information Relation]
The information extractable from an oscillator in time $T$ is
\begin{equation}
    I = Q \nu T \cdot \ln 2 \text{ bits}
\end{equation}
\end{theorem}

\begin{proof}
In time $T$, the oscillator completes $N = \nu T$ cycles. Each cycle distinguishes between $Q$ levels. Total distinguishable states: $Q^N$. Information: $\log_2(Q^N) = N \log_2 Q \approx Q\nu T \ln 2 / \ln 2$.
\end{proof}

\subsection{Phase Space Coverage}

\begin{definition}[Oscillator Phase Space]
The phase space of an oscillator with frequency $\nu$ and energy $E$ is the ellipse
\begin{equation}
    \frac{p^2}{2mE} + \frac{m\omega^2 x^2}{2E} = 1
\end{equation}
with area $\Gamma = E/\nu = 2\pi E/\omega$.
\end{definition}

\begin{proposition}[Minimum Phase Space Cell]
The minimum resolvable phase space cell has area
\begin{equation}
    \Gamma_{\min} = \frac{h}{Q}
\end{equation}
\end{proposition}

Higher quality factors enable finer phase space resolution, corresponding to more precise partition coordinate extraction.

\subsection{Oscillator Coupling}

\begin{definition}[Coupled Oscillators]
Oscillators $i$ and $j$ are coupled if energy transfer occurs:
\begin{equation}
    H = H_i + H_j + V_{ij}
\end{equation}
where $V_{ij}$ is the coupling potential.
\end{definition}

\begin{proposition}[Coupling and Partition Depth]
Coupling between oscillators at levels $i$ and $j$ in the hierarchy enables transitions in partition coordinates:
\begin{equation}
    \Delta n = |i - j|
\end{equation}
Direct coupling between adjacent levels ($|i-j| = 1$) enables $\Delta n = 1$ transitions; indirect coupling through intermediate levels is required for larger $\Delta n$.
\end{proposition}

This explains the selection rule $\Delta l = \pm 1$: direct coupling between adjacent angular modes dominates.

