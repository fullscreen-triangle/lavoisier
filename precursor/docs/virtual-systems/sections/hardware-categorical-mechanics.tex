\section{Hardware Categorical Mechanics}
\label{sec:hardware-categorical-mechanics}

\subsection{Categorical State Creation}

\begin{definition}[Categorical State]
A categorical state is a discrete, distinguishable configuration of a system. Two states are categorically distinct if they can be distinguished by at least one measurement.
\end{definition}

\begin{theorem}[Oscillation Creates Categories]
\label{thm:oscillation-creates}
Any oscillatory measurement establishes categorical states. Prior to measurement, the system exists in a superposition; after measurement, it occupies a definite category.
\end{theorem}

\begin{proof}
An oscillator with frequency $\nu$ distinguishes states separated by energy $\Delta E \geq h\nu$. States within $\Delta E < h\nu$ are indistinguishable. The measurement thus partitions the continuous energy axis into discrete bins---categories.
\end{proof}

\subsection{Category-Oscillator Correspondence}

\begin{proposition}[One Oscillator, One Coordinate]
Each independent oscillator in the hierarchy extracts one partition coordinate:
\begin{align}
    \text{Oscillator 1} &\to n \quad \text{(partition depth)} \\
    \text{Oscillator 2} &\to l \quad \text{(angular complexity)} \\
    \text{Oscillator 3} &\to m \quad \text{(orientation)} \\
    \text{Oscillator 4} &\to s \quad \text{(chirality)}
\end{align}
\end{proposition}

\begin{proof}
Independence of oscillators ensures orthogonality of the extracted information. Four independent oscillators extract four independent coordinates, matching the dimensionality of partition space.
\end{proof}

\subsection{Categorical Morphisms}

\begin{definition}[Categorical Transition Morphism]
A transition between categorical states $(n_i, l_i, m_i, s_i) \to (n_f, l_f, m_f, s_f)$ is a morphism in the category of partition states if it satisfies selection rules.
\end{definition}

\begin{proposition}[Morphism Composition]
Morphisms compose: if $\alpha: A \to B$ and $\beta: B \to C$ are valid transitions, then $\beta \circ \alpha: A \to C$ is valid.
\end{proposition}

This composition law enables construction of multi-step fragmentation pathways from elementary transitions.

\subsection{Categorical Completeness}

\begin{definition}[Categorical Completeness]
A measurement is categorically complete if it uniquely determines all partition coordinates $(n, l, m, s)$.
\end{definition}

\begin{theorem}[Completeness from Hierarchy]
A measurement using oscillation hierarchy $\mathcal{H}$ with $|\mathcal{H}| \geq 4$ independent oscillators is categorically complete.
\end{theorem}

\begin{proof}
Four independent oscillators extract four coordinates. By Theorem~\ref{thm:completeness} (main text), four coordinates suffice for complete specification.
\end{proof}

\begin{corollary}[Minimum Hardware]
Categorically complete measurement requires at least four independent oscillation modes.
\end{corollary}

This explains why single-stage mass spectrometry (one dominant oscillation mode) provides incomplete structural information: it extracts only $n$, leaving $(l, m, s)$ undetermined.

\subsection{Categorical Entropy}

\begin{definition}[Categorical Entropy]
The categorical entropy of a measurement is
\begin{equation}
    S_{\text{cat}} = -k_B \sum_i p_i \ln p_i
\end{equation}
where $p_i$ is the probability of observing category $i$.
\end{definition}

\begin{proposition}[Maximum Entropy]
For $C(n) = 2n^2$ categories at depth $n$, maximum entropy is
\begin{equation}
    S_{\max} = k_B \ln(2n^2) = k_B(2\ln n + \ln 2)
\end{equation}
achieved when all categories are equally probable.
\end{proposition}

\begin{theorem}[Entropy Increase Under Measurement]
Sequential measurements monotonically increase categorical entropy:
\begin{equation}
    S(M_1, M_2, \ldots, M_k) \geq S(M_1, M_2, \ldots, M_{k-1})
\end{equation}
with equality only when $M_k$ provides no new information.
\end{theorem}

\begin{proof}
Each measurement establishes new categorical distinctions. The number of distinguishable states can only increase (or stay constant), hence entropy is non-decreasing.
\end{proof}

\subsection{Hardware Substrate Independence}

\begin{theorem}[Substrate Independence]
\label{thm:substrate-independence}
The categorical states created by oscillatory measurement depend only on the oscillation hierarchy, not on the physical substrate.
\end{theorem}

\begin{proof}
The categorical state $(n, l, m, s)$ is determined by timing relationships between oscillation levels (Algorithm~\ref{alg:timing}, main text). These timing relationships are independent of whether the oscillators are mechanical, electromagnetic, or acoustic.
\end{proof}

This theorem establishes the universality of the virtual instrument framework: any hardware implementing the required oscillation hierarchy extracts identical partition coordinates.

