\section{Virtual Detector Ensemble}
\label{sec:virtual-detector-ensemble}

\subsection{Virtual Instrument Definition}

\begin{definition}[Virtual Instrument]
A virtual instrument is a software function $V: \mathcal{D} \to \mathcal{C}$ that maps raw detector data $\mathcal{D}$ to partition coordinates $\mathcal{C} = (n, l, m, s)$.
\end{definition}

Virtual instruments are ``virtual'' in that they do not correspond to specific hardware components but rather to information extraction protocols applied to hardware output.

\subsection{The Four Canonical Virtual Instruments}

\subsubsection{Shell Resonator}

\begin{definition}[Shell Resonator]
The Shell Resonator extracts partition depth $n$ from the primary oscillation frequency:
\begin{equation}
    n = V_{\text{shell}}(\nu_1, Q_1, t) = \left\lfloor \frac{\nu_1 \cdot t}{Q_1 \cdot \tau_{\text{ref}}} \right\rfloor
\end{equation}
where $\tau_{\text{ref}}$ is a reference time constant.
\end{definition}

Physical interpretation: $n$ counts the number of complete oscillation cycles that fit within the bounded region. For molecular fragmentation, $n$ corresponds to the fragmentation generation.

\subsubsection{Angular Analyser}

\begin{definition}[Angular Analyser]
The Angular Analyser extracts angular complexity $l$ from the secondary oscillation frequency:
\begin{equation}
    l = V_{\text{angular}}(\nu_2, Q_2, n) = \left\lfloor \frac{\nu_2}{\nu_1} \cdot n \right\rfloor \mod n
\end{equation}
\end{definition}

Physical interpretation: $l$ counts the number of angular nodes in the secular motion. For fragmentation, $l$ corresponds to the pathway complexity---the number of distinct bond cleavages.

\subsubsection{Orientation Mapper}

\begin{definition}[Orientation Mapper]
The Orientation Mapper extracts orientation $m$ from phase relationships:
\begin{equation}
    m = V_{\text{orient}}(\phi_1, \phi_2, l) = \text{round}\left( \frac{\phi_1 - \phi_2}{\pi} \cdot l \right)
\end{equation}
with $|m| \leq l$.
\end{definition}

Physical interpretation: $m$ identifies which specific fragmentation pathway was taken among pathways of equal complexity $l$.

\subsubsection{Chirality Discriminator}

\begin{definition}[Chirality Discriminator]
The Chirality Discriminator extracts chirality $s$ from polarization or rotation:
\begin{equation}
    s = V_{\text{chiral}}(\phi_L, \phi_R) = \frac{1}{2} \text{sign}(\phi_L - \phi_R)
\end{equation}
where $\phi_L, \phi_R$ are phases measured with left/right circular polarization.
\end{definition}

Physical interpretation: $s = \pm 1/2$ encodes the handedness of the molecular configuration.

\subsection{Instrument Independence}

\begin{theorem}[Virtual Instrument Independence]
The four virtual instruments extract orthogonal information:
\begin{equation}
    \text{Cov}(V_i, V_j) = 0 \quad \text{for } i \neq j
\end{equation}
\end{theorem}

\begin{proof}
Each virtual instrument $V_i$ depends on a distinct oscillator in the hierarchy. Independence of oscillators (by definition of hierarchy) implies independence of extracted coordinates.
\end{proof}

\subsection{Hardware Implementation Requirements}

\begin{table}[h]
\centering
\caption{Hardware requirements for virtual instruments}
\begin{tabular}{llc}
\toprule
Virtual Instrument & Required Hardware & Min. $Q$ \\
\midrule
Shell Resonator & Primary oscillator & $n_{\max}$ \\
Angular Analyser & Secondary oscillator & $n_{\max}^2$ \\
Orientation Mapper & Phase detector & $n_{\max}^2$ \\
Chirality Discriminator & Polarization selector & $2$ \\
\bottomrule
\end{tabular}
\end{table}

The quality factor requirements scale with desired partition depth. For $n_{\max} = 10$: Shell Resonator requires $Q \geq 10$, Angular Analyser requires $Q \geq 100$.

\subsection{Virtual Instrument Composition}

\begin{definition}[Instrument Ensemble]
The instrument ensemble $\mathcal{E} = (V_{\text{shell}}, V_{\text{angular}}, V_{\text{orient}}, V_{\text{chiral}})$ is the product of four virtual instruments:
\begin{equation}
    \mathcal{E}: \mathcal{D} \to (n, l, m, s)
\end{equation}
\end{definition}

\begin{proposition}[Ensemble Completeness]
The instrument ensemble extracts all partition coordinates. No additional virtual instruments can provide independent information.
\end{proposition}

\begin{proof}
By Theorem~\ref{thm:completeness} (main text), four coordinates suffice. By independence (previous theorem), the four instruments provide exactly four independent measurements.
\end{proof}

\subsection{Partial Ensembles}

Not all hardware platforms support the full ensemble. Partial ensembles extract subsets of coordinates:

\begin{itemize}
    \item \textbf{Mass-only} ($V_{\text{shell}}$): Extracts $n$ only. Typical of unit-resolution MS.
    \item \textbf{MS/MS} ($V_{\text{shell}}, V_{\text{angular}}$): Extracts $(n, l)$. Typical of tandem MS.
    \item \textbf{MS + IMS} ($V_{\text{shell}}, V_{\text{angular}}, V_{\text{orient}}$): Extracts $(n, l, m)$.
    \item \textbf{Full ensemble}: Extracts $(n, l, m, s)$. Requires chiral selection.
\end{itemize}

\begin{proposition}[Information Hierarchy]
The information content of partial ensembles is strictly ordered:
\begin{equation}
    I(V_{\text{shell}}) < I(V_{\text{shell}}, V_{\text{angular}}) < I(V_{\text{shell}}, V_{\text{angular}}, V_{\text{orient}}) < I(\mathcal{E})
\end{equation}
\end{proposition}

Each additional virtual instrument provides non-redundant information, improving identification confidence.

