\section{Geometric Element Partitioning}
\label{sec:geometric-element-partitioning}

\subsection{Phase Space Geometry}

\begin{definition}[Bounded Phase Space]
A bounded phase space $\Omega \subset \mathbb{R}^{2d}$ is a compact region with symplectic structure $\omega$ and finite volume $V = \int_\Omega \omega^d$.
\end{definition}

The geometry of $\Omega$ determines the partition coordinate structure. We consider the generic case of a centrally symmetric bounded region.

\subsection{Radial Partitioning}

\begin{definition}[Radial Partition]
The radial partition of depth $n$ divides $\Omega$ into $n$ concentric shells:
\begin{equation}
    \Omega = \bigcup_{k=1}^n S_k, \quad S_k = \{x : r_{k-1} < |x| \leq r_k\}
\end{equation}
where $r_0 = 0$ and $r_n = R$ (boundary radius).
\end{definition}

\begin{proposition}[Shell Volume]
For uniform radial spacing $r_k = kR/n$, the volume of shell $S_k$ scales as
\begin{equation}
    V(S_k) \propto k^{d-1}
\end{equation}
in $d$ dimensions.
\end{proposition}

In 3D ($d=3$), outer shells have larger volume, accommodating more categorical states.

\subsection{Angular Partitioning}

\begin{definition}[Angular Partition]
Within shell $S_k$, the angular partition of complexity $l$ divides the shell into zones by polar angle:
\begin{equation}
    S_k = \bigcup_{j=0}^{l} Z_{k,j}, \quad Z_{k,j} = \{x \in S_k : \theta_j < \theta \leq \theta_{j+1}\}
\end{equation}
\end{definition}

\begin{proposition}[Angular Constraint]
The maximum angular complexity at depth $n$ is
\begin{equation}
    l_{\max} = n - 1
\end{equation}
\end{proposition}

\begin{proof}
Angular modes with wavelength $\lambda_\theta$ require radial extent $r \geq \lambda_\theta$. At depth $n$ with radial extent $R/n$, the minimum angular wavelength is $\lambda_{\theta,\min} = R/n$. The number of angular nodes is thus bounded by $l < n$.
\end{proof}

\subsection{Orientation Partitioning}

\begin{definition}[Orientation Partition]
Within angular zone $Z_{k,j}$, the orientation partition divides by azimuthal angle:
\begin{equation}
    Z_{k,j} = \bigcup_{m=-l}^{l} O_{k,j,m}
\end{equation}
with $2l+1$ orientation sectors.
\end{definition}

\begin{proposition}[Orientation Count]
At angular complexity $l$, there are $2l+1$ distinguishable orientations.
\end{proposition}

\begin{proof}
The azimuthal angle $\phi \in [0, 2\pi)$ admits $2l+1$ distinguishable sectors when the angular wavelength is $2\pi/(2l+1)$.
\end{proof}

\subsection{Chirality Partitioning}

\begin{definition}[Chirality Partition]
Each orientation sector admits a binary chirality distinction:
\begin{equation}
    O_{k,j,m} = O_{k,j,m}^+ \cup O_{k,j,m}^-
\end{equation}
corresponding to $s = +1/2$ and $s = -1/2$.
\end{definition}

\begin{proposition}[Chirality as Double Cover]
The chirality partition corresponds to the double cover of $SO(3)$ by $SU(2)$. Two orientations related by $2\pi$ rotation are distinguishable by chirality.
\end{proposition}

\subsection{Capacity Derivation}

\begin{theorem}[Partition Capacity]
The number of distinct partition elements at depth $n$ is
\begin{equation}
    C(n) = 2n^2
\end{equation}
\end{theorem}

\begin{proof}
Count partition elements:
\begin{align}
    C(n) &= \sum_{l=0}^{n-1} \sum_{m=-l}^{l} 2 \quad \text{(sum over } l, m, s \text{)} \\
    &= 2 \sum_{l=0}^{n-1} (2l+1) \\
    &= 2n^2
\end{align}
\end{proof}

\subsection{Element Labeling}

\begin{definition}[Partition Element]
A partition element $E_{n,l,m,s}$ is the region of phase space with coordinates $(n, l, m, s)$. Elements are non-overlapping and exhaust $\Omega$:
\begin{equation}
    \Omega = \bigsqcup_{n,l,m,s} E_{n,l,m,s}
\end{equation}
\end{definition}

\begin{proposition}[Element Uniqueness]
No two partition elements share identical coordinates. Each point $x \in \Omega$ belongs to exactly one element.
\end{proposition}

This uniqueness is the partition analogue of the Pauli exclusion principle: no two ``categorical particles'' can occupy the same partition element.

\subsection{Periodic Table Correspondence}

The partition capacity formula $C(n) = 2n^2$ reproduces the shell structure of the periodic table:

\begin{center}
\begin{tabular}{ccc}
\toprule
Shell $n$ & $C(n) = 2n^2$ & Elements in Period \\
\midrule
1 & 2 & H, He \\
2 & 8 & Li--Ne \\
3 & 18 & Na--Ar (partial: 8) \\
4 & 32 & K--Kr (partial: 18) \\
\bottomrule
\end{tabular}
\end{center}

The discrepancy between $C(n)$ and period length arises from energy ordering effects (Section~6 of companion paper): elements fill in $(n+l)$ order, not $n$ order.

This correspondence is not coincidental. Atomic electron configurations occupy partition elements in bounded phase space (the atomic potential well). The geometric constraints derived here are the same constraints governing atomic structure.

