\section{Virtual Chromatography}
\label{sec:virtual-chromatography}

\subsection{Chromatographic Oscillation}

Chromatography involves oscillatory partitioning between mobile and stationary phases.

\begin{definition}[Chromatographic Oscillator]
A chromatographic oscillator is the repeated equilibration between mobile phase (M) and stationary phase (S):
\begin{equation}
    \text{M} \rightleftharpoons \text{S} \rightleftharpoons \text{M} \rightleftharpoons \cdots
\end{equation}
with effective frequency $\nu_{\text{chrom}} = v / L$ where $v$ is flow velocity and $L$ is the characteristic plate height.
\end{definition}

\begin{proposition}[Chromatographic Quality Factor]
The plate count $N$ is the quality factor of the chromatographic oscillator:
\begin{equation}
    Q_{\text{chrom}} = N = \frac{L_{\text{column}}}{H}
\end{equation}
where $H$ is the height equivalent to a theoretical plate.
\end{proposition}

\subsection{Partition Coordinates from Retention}

\begin{definition}[Retention-Based Partition Depth]
The partition depth extractable from retention time $t_R$ is
\begin{equation}
    n_{\text{chrom}} = \left\lfloor \sqrt{N} \cdot \frac{t_R - t_0}{t_R} \right\rfloor
\end{equation}
where $t_0$ is the void time.
\end{definition}

\begin{proposition}[Chromatographic Resolution to Partition Depth]
Two compounds with retention times $t_{R1}$ and $t_{R2}$ have distinct partition depths if
\begin{equation}
    |t_{R1} - t_{R2}| > \frac{t_{R1}}{\sqrt{N}}
\end{equation}
\end{proposition}

\subsection{Multi-Dimensional Chromatography}

\begin{definition}[Orthogonal Chromatographic Dimensions]
Chromatographic dimensions are orthogonal if they separate by independent physicochemical properties:
\begin{itemize}
    \item Dimension 1: Hydrophobicity (reversed-phase)
    \item Dimension 2: Polarity (HILIC)
    \item Dimension 3: Size (SEC)
    \item Dimension 4: Charge (IEX)
\end{itemize}
\end{definition}

\begin{theorem}[Chromatographic Partition Mapping]
Orthogonal chromatographic dimensions map to partition coordinates:
\begin{align}
    \text{RP retention} &\to n \quad \text{(partition depth)} \\
    \text{HILIC retention} &\to l \quad \text{(angular complexity)} \\
    \text{SEC retention} &\to m \quad \text{(orientation)} \\
    \text{IEX retention} &\to s \quad \text{(chirality proxy)}
\end{align}
\end{theorem}

\begin{proof}
Each chromatographic dimension implements an independent oscillator with distinct selectivity. The mapping to partition coordinates follows from the oscillator-coordinate correspondence (Section~\ref{sec:hardware-categorical-mechanics}).
\end{proof}

\subsection{Virtual Chromatographic Instruments}

\subsubsection{Virtual Retention Index}

\begin{definition}[Virtual Retention Index]
The virtual retention index $I_V$ is a platform-independent retention descriptor:
\begin{equation}
    I_V = \frac{n \cdot 100 + l \cdot 10 + m + (s + 0.5)}{1.1}
\end{equation}
\end{definition}

The virtual retention index enables comparison of retention data across different chromatographic platforms (HPLC, UHPLC, GC) by converting to partition coordinates.

\subsubsection{Virtual Selectivity}

\begin{definition}[Virtual Selectivity]
The virtual selectivity between compounds A and B is
\begin{equation}
    \alpha_V = \frac{\|(n_A, l_A, m_A, s_A) - (n_B, l_B, m_B, s_B)\|}{n_{\max}}
\end{equation}
\end{definition}

\begin{proposition}[Selectivity Optimization]
Optimal chromatographic separation maximizes virtual selectivity:
\begin{equation}
    \text{Conditions}^* = \arg\max_{\text{Conditions}} \alpha_V(\text{A}, \text{B})
\end{equation}
\end{proposition}

\subsection{LC-MS Integration}

\begin{theorem}[LC-MS Partition Completeness]
Hyphenated LC-MS with $\geq 2$ chromatographic dimensions and MS/MS provides partition-complete measurement.
\end{theorem}

\begin{proof}
Two chromatographic dimensions provide $n$ and $l$. MS/MS provides $n$ (mass) and $l$ (fragmentation). The combination overcounts $n$ and $l$ but enables validation. Adding IMS provides $m$; chiral chromatography provides $s$.
\end{proof}

\subsection{Virtual Chromatogram}

\begin{definition}[Virtual Chromatogram]
A virtual chromatogram is the projection of partition coordinates onto a 1D time axis:
\begin{equation}
    C_V(t) = \sum_i \delta(t - t_i(n_i, l_i))
\end{equation}
where $t_i(n, l) = t_0 + (n + \alpha l) \cdot \Delta t$ is the virtual retention time.
\end{definition}

The virtual chromatogram is platform-independent: different chromatographic systems measuring the same analytes produce identical virtual chromatograms after coordinate transformation.

\subsection{Method Translation}

\begin{algorithm}
\caption{Chromatographic Method Translation}
\label{alg:method-translation}
\begin{algorithmic}[1]
\Require Method $M_1$ on platform $P_1$, target platform $P_2$
\Ensure Translated method $M_2$
\State Characterize $P_1$: oscillation hierarchy $\mathcal{H}_1$, quality factors $\{Q_i\}$
\State Characterize $P_2$: oscillation hierarchy $\mathcal{H}_2$, quality factors $\{Q_j\}$
\State Extract partition coordinates from $M_1$ on $P_1$: $(n, l, m, s)$
\State Compute required conditions on $P_2$ to achieve same coordinates
\State \Return $M_2$
\end{algorithmic}
\end{algorithm}

This algorithm enables systematic method transfer between chromatographic platforms---a common pain point in analytical chemistry.

