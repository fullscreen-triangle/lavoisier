\section{Platform Independence as Categorical Invariance}
\label{sec:platform-independence}

\subsection{The Platform Dependence Problem}

Mass spectrometry exhibits apparent platform dependence: different instruments yield different spectra for the same compound. This manifests as:
\begin{itemize}
    \item Different relative intensities of fragment peaks
    \item Different fragmentation patterns under nominally identical conditions
    \item Poor spectral library matching across instrument platforms
\end{itemize}

\begin{definition}[Platform Dependence]
A measurement protocol exhibits platform dependence if the output $O$ depends on both the target system $T$ and the measurement apparatus $A$:
\begin{equation}
    O = f(T, A)
\end{equation}
rather than $O = f(T)$ alone.
\end{definition}

\subsection{Categorical Invariants}

\begin{definition}[Categorical Invariant]
A quantity $I$ is a categorical invariant if it depends only on the categorical state, not on the measurement protocol:
\begin{equation}
    I(T, A_1) = I(T, A_2) \quad \forall A_1, A_2
\end{equation}
\end{definition}

\begin{theorem}[Partition Coordinates are Invariant]
\label{thm:invariance}
The partition coordinates $(n, l, m, s)$ of a categorical state are platform-independent: any measurement apparatus extracting these coordinates from the same target yields identical values.
\end{theorem}

\begin{proof}
Partition coordinates are defined by geometric constraints (Section~\ref{sec:geometric-partitioning}) that depend only on the bounded phase space $\Omega$ of the target, not on the measurement apparatus. The apparatus determines only the extraction protocol (Section~\ref{sec:hardware-mapping}), not the coordinates themselves.

Suppose apparatus $A_1$ measures $(n_1, l_1, m_1, s_1)$ and $A_2$ measures $(n_2, l_2, m_2, s_2)$ for the same target in the same categorical state. By the completeness theorem (Theorem~\ref{thm:completeness}), identical coordinates imply identical categorical states. If $n_1 \neq n_2$ (or similarly for other coordinates), the apparatuses would be detecting different categorical states of the same target---a contradiction, since the target is in a single well-defined state.
\end{proof}

\subsection{Apparent Dependence from Projection}

The apparent platform dependence of mass spectra arises from different projections of the same partition coordinates.

\begin{definition}[Partition Projection]
A partition projection is a map $\pi: (n, l, m, s) \to \mathbb{R}^k$ that extracts $k$ observable quantities from the coordinates. Different apparatuses implement different projections.
\end{definition}

\begin{example}[Mass-Only Projection]
A unit-resolution quadrupole implements the projection
\begin{equation}
    \pi_{\text{quad}}(n, l, m, s) = m(n)
\end{equation}
mapping to mass alone, discarding $(l, m, s)$ information.
\end{example}

\begin{example}[Mass-Intensity Projection]
A high-resolution mass spectrometer with intensity information implements
\begin{equation}
    \pi_{\text{HR}}(n, l, m, s) = (m(n), I(n, l))
\end{equation}
where $I(n, l)$ is the intensity determined by the transition rate (Section~\ref{sec:categorical-dynamics}).
\end{example}

\begin{proposition}[Projection Reconciliation]
Given two projections $\pi_1$ and $\pi_2$, the underlying partition coordinates can be recovered if the combined projection $(\pi_1, \pi_2)$ is invertible.
\end{proposition}

This proposition underlies multi-platform data fusion: combining mass spectra from different instruments provides more complete partition coordinate determination than any single instrument alone.

\subsection{Standardization Protocol}

\begin{algorithm}
\caption{Platform-Independent Coordinate Extraction}
\label{alg:standardization}
\begin{algorithmic}[1]
\Require Raw spectrum $S$, hardware characterization $H$, extraction protocol $P$
\Ensure Platform-independent partition coordinates $(n, l, m, s)$
\State Identify oscillation hierarchy from $H$: $\{\nu_1, \nu_2, \ldots\}$
\State Decompose $S$ into frequency-resolved components $\{S_i\}$
\State Apply timing analysis (Algorithm~\ref{alg:timing}) to extract raw coordinates
\State Correct for apparatus-specific factors (efficiency, transmission, detection)
\State Normalize to canonical coordinate system
\State \Return $(n, l, m, s)$
\end{algorithmic}
\end{algorithm}

\subsection{Experimental Validation}

Platform independence is testable. For a reference compound with known partition coordinates:

\begin{enumerate}
    \item Measure on platform $A_1$, extract $(n, l, m, s)_{A_1}$
    \item Measure on platform $A_2$, extract $(n, l, m, s)_{A_2}$
    \item Compare: $\|(n, l, m, s)_{A_1} - (n, l, m, s)_{A_2}\| < \epsilon$
\end{enumerate}

\begin{proposition}[Convergence Criterion]
Platforms $A_1$ and $A_2$ are correctly calibrated if and only if their partition coordinate extractions agree within resolution limits for all test compounds.
\end{proposition}

This provides a principled calibration protocol: rather than matching raw spectral features, match the underlying partition coordinates.

\subsection{Information-Theoretic Interpretation}

\begin{theorem}[Information Preservation]
The mutual information between target $T$ and measurement $O$ is maximized when the measurement extracts all partition coordinates:
\begin{equation}
    I(T; O) \leq H(n, l, m, s)
\end{equation}
with equality when $O = (n, l, m, s)$.
\end{theorem}

\begin{proof}
The partition coordinates constitute a sufficient statistic for the categorical state (Theorem~\ref{thm:completeness}). Any measurement retaining less information has strictly lower mutual information.
\end{proof}

Platform-independent measurements are thus optimal in the information-theoretic sense: they extract the maximum information about the target system.

