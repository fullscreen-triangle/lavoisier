\section{Oscillatory Measurement Theorem}
\label{sec:oscillatory-theorem}

\subsection{Hardware Oscillator Fundamentals}

\begin{definition}[Hardware Oscillator]
A hardware oscillator is a physical system with periodic dynamics characterized by frequency $\nu$ and quality factor $Q$. The minimum distinguishable time interval is
\begin{equation}
    \delta t = \frac{1}{Q \nu}
\end{equation}
\end{definition}

All mass spectrometers employ hardware oscillators:
\begin{itemize}
    \item Quadrupole: RF field oscillation at $\nu \sim 1$ MHz
    \item Ion trap: Secular motion at $\nu \sim 100$ kHz
    \item Orbitrap: Axial oscillation at $\nu \sim 100$ kHz
    \item Time-of-flight: Detection timing at $\delta t \sim 1$ ns resolution
\end{itemize}

\begin{theorem}[Oscillatory Measurement]
\label{thm:oscillatory}
Any measurement that extracts information from a bounded system must employ an oscillatory process. The information gained is bounded by
\begin{equation}
    I \leq Q \cdot \ln 2
\end{equation}
bits per oscillation cycle.
\end{theorem}

\begin{proof}
Information extraction requires distinguishing system states. In a bounded system, states differ by minimum action $h$. Distinguishing states separated by $\Delta E$ requires observation time $\Delta t \geq h/\Delta E$. This time-energy uncertainty mandates periodic sampling---oscillation. The quality factor $Q$ counts distinguishable cycles, bounding information at $Q \ln 2$ bits.
\end{proof}

\subsection{Oscillation Hierarchy}

\begin{definition}[Oscillation Hierarchy]
A mass spectrometer implements an oscillation hierarchy $\{\nu_1 > \nu_2 > \cdots > \nu_k\}$ where each frequency corresponds to a distinct physical process:
\begin{align}
    \nu_1 &: \text{RF field frequency (ion confinement)} \\
    \nu_2 &: \text{Secular frequency (ion motion)} \\
    \nu_3 &: \text{Detection frequency (signal digitization)}
\end{align}
\end{definition}

\begin{proposition}[Frequency-Coordinate Correspondence]
Each level in the oscillation hierarchy extracts one partition coordinate:
\begin{align}
    \nu_1 &\to n \quad \text{(partition depth from confinement)} \\
    \nu_2 &\to l \quad \text{(angular complexity from secular motion)} \\
    \nu_3 &\to m \quad \text{(orientation from phase detection)}
\end{align}
The chirality coordinate $s$ requires comparison between two measurements (e.g., left/right circular polarization).
\end{proposition}

\subsection{Timing Analysis Protocol}

\begin{algorithm}
\caption{Partition Coordinate Extraction from Oscillator Timing}
\label{alg:timing}
\begin{algorithmic}[1]
\Require Oscillation hierarchy $\{\nu_i\}$, quality factors $\{Q_i\}$, raw timing data $T$
\Ensure Partition coordinates $(n, l, m, s)$
\State Decompose $T$ into frequency components $\{T_i\}$ by bandpass filtering
\State $n \gets \lfloor Q_1 \cdot T_1 \cdot \nu_1 / \tau_{\text{ref}} \rfloor$ \Comment{Partition depth from confinement cycles}
\State $l \gets \lfloor Q_2 \cdot T_2 \cdot \nu_2 / \tau_{\text{ref}} \rfloor \mod n$ \Comment{Angular complexity from secular cycles}
\State $m \gets \text{sign}(\phi_2) \cdot \lfloor |Q_2 \cdot \phi_2| / \pi \rfloor$ \Comment{Orientation from phase}
\State $s \gets \frac{1}{2}\text{sign}(\phi_1 - \phi_1')$ \Comment{Chirality from polarization comparison}
\State \Return $(n, l, m, s)$
\end{algorithmic}
\end{algorithm}

\subsection{Resolution Limits}

\begin{theorem}[Resolution-Frequency Relation]
The mass resolution $R = m/\Delta m$ achievable by an oscillator-based analyzer is
\begin{equation}
    R = Q \cdot \nu \cdot \tau
\end{equation}
where $\tau$ is the measurement time.
\end{theorem}

\begin{proof}
Mass measurement requires timing ion motion to precision $\delta t$. The mass-to-time relationship $m \propto t^2$ (for TOF) or $m \propto 1/\nu^2$ (for trap) gives $\delta m/m = 2\delta t/t$ or $\delta m/m = 2\delta\nu/\nu$. With $\delta t = 1/(Q\nu)$ and $t = \tau$, we obtain $R = Q\nu\tau$.
\end{proof}

This theorem connects hardware specifications (quality factor, frequency) to partition coordinate resolution. Higher resolution corresponds to finer partition of phase space---more distinguishable categorical states.

\begin{corollary}[Partition Depth from Resolution]
The maximum partition depth resolvable by a mass analyzer is
\begin{equation}
    n_{\max} = \lfloor \sqrt{R} \rfloor
\end{equation}
\end{corollary}

\begin{proof}
The capacity at depth $n$ is $C(n) = 2n^2$. Resolving all states at depth $n$ requires $R \geq 2n^2$, giving $n \leq \sqrt{R/2}$.
\end{proof}

For an Orbitrap with $R = 100,000$, the maximum partition depth is $n_{\max} \approx 223$---far exceeding the fragmentation depths relevant for metabolomics.

