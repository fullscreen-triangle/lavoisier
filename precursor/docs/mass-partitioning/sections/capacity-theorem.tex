\section{Capacity Theorem}
\label{sec:capacity-theorem}

\subsection{State Enumeration at Fixed Depth}

\begin{theorem}[Fundamental Capacity Formula]
\label{thm:capacity}
The number of distinct categorical states at partition depth $n$ is
\begin{equation}
    C(n) = 2n^2
\end{equation}
\end{theorem}

\begin{proof}
At fixed $n$, enumerate states by summing over allowed values of $(l, m, s)$:
\begin{align}
    C(n) &= \sum_{l=0}^{n-1} \sum_{m=-l}^{l} \sum_{s \in \{-1/2, +1/2\}} 1 \\
    &= 2 \sum_{l=0}^{n-1} (2l + 1) \\
    &= 2 \sum_{l=0}^{n-1} (2l + 1)
\end{align}
The inner sum evaluates to:
\begin{equation}
    \sum_{l=0}^{n-1} (2l + 1) = 2 \cdot \frac{(n-1)n}{2} + n = n^2 - n + n = n^2
\end{equation}
Therefore $C(n) = 2n^2$.
\end{proof}

\begin{corollary}[Subshell Capacity]
The number of states with fixed $(n, l)$ is $2(2l+1)$.
\end{corollary}

\begin{proof}
Sum over $m$ and $s$: $\sum_{m=-l}^{l} \sum_s 1 = 2(2l+1)$.
\end{proof}

\subsection{Cumulative Capacity}

\begin{theorem}[Total Capacity]
The total number of categorical states with partition depth up to $N$ is
\begin{equation}
    C_{\text{total}}(N) = \sum_{n=1}^{N} 2n^2 = \frac{N(N+1)(2N+1)}{3}
\end{equation}
\end{theorem}

\begin{proof}
Direct summation using $\sum_{n=1}^N n^2 = N(N+1)(2N+1)/6$.
\end{proof}

For molecular fragmentation, this establishes the maximum structural information content at each fragmentation depth:

\begin{center}
\begin{tabular}{ccc}
\toprule
Depth $n$ & $C(n)$ & Cumulative \\
\midrule
1 & 2 & 2 \\
2 & 8 & 10 \\
3 & 18 & 28 \\
4 & 32 & 60 \\
5 & 50 & 110 \\
\bottomrule
\end{tabular}
\end{center}

The rapid growth of cumulative capacity explains why MS$^3$ typically suffices for structural elucidation: with 28 distinct categorical states accessible, most molecular topologies are uniquely determined.

\subsection{Capacity Density}

\begin{definition}[Capacity Density]
The capacity density at depth $n$ is
\begin{equation}
    \rho(n) = \frac{C(n)}{V(n)}
\end{equation}
where $V(n) \propto n^3$ is the phase space volume at depth $n$.
\end{definition}

\begin{proposition}
The capacity density decreases with depth:
\begin{equation}
    \rho(n) \propto \frac{2n^2}{n^3} = \frac{2}{n}
\end{equation}
\end{proposition}

This decreasing density has physical significance: deeper fragmentation yields diminishing returns in categorical information per unit energy invested. The energy required to reach depth $n$ scales as $E \propto n^2$ (see Section~\ref{sec:categorical-dynamics}), while information gained scales as $\ln C(n) \sim 2\ln n$. The information-to-energy ratio thus decreases as $\ln n / n^2 \to 0$.

\subsection{Geometric Interpretation}

The capacity formula admits a geometric interpretation. Consider the unit sphere $S^2$ at each radial shell $n$. The angular complexity $l$ partitions the sphere into zones of angular width $\pi/n$. Within each zone, the orientation $m$ selects one of $2l+1$ meridional sectors. The chirality $s$ doubles the count by distinguishing hemispheres under parity.

\begin{proposition}[Area Correspondence]
The number of states at $(n, l)$ equals the number of distinct solid angle elements of area $4\pi/(2n^2)$ that fit on the unit sphere with angular complexity exactly $l$.
\end{proposition}

This geometric picture connects partition capacity to the surface area of bounded regions, providing a bridge to thermodynamic considerations in Section~\ref{sec:entropy-equivalence}.

