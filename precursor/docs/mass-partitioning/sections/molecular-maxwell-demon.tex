\section{Thermodynamic Consistency: The Molecular Maxwell Demon}
\label{sec:molecular-maxwell-demon}

\subsection{The Measurement Problem}

Partition coordinate extraction raises a thermodynamic question: does measurement create or merely reveal categorical states? If measurement creates states, the entropy cost must be accounted for.

\begin{definition}[Maxwell Demon]
A Maxwell demon is a hypothetical agent that sorts molecules by velocity, apparently decreasing entropy without energy expenditure.
\end{definition}

The resolution of the Maxwell demon paradox (Landauer, Bennett) established that information processing has irreducible thermodynamic cost. We apply this insight to partition coordinate measurement.

\subsection{Measurement as Categorical Completion}

\begin{theorem}[Measurement Creates Categories]
\label{thm:measurement-creates}
A partition coordinate measurement $(n, l, m, s)$ establishes a categorical state that did not exist prior to measurement. The entropy cost is
\begin{equation}
    \Delta S_{\text{measure}} \geq k_B \ln 2 \cdot \log_2 C(n)
\end{equation}
where $C(n) = 2n^2$ is the capacity at depth $n$.
\end{theorem}

\begin{proof}
Prior to measurement, the system is in a superposition of categorical states. Measurement projects onto a specific $(n, l, m, s)$. This projection requires distinguishing one state from $C(n) - 1$ alternatives, with minimum entropy cost $k_B \ln C(n) = k_B \ln(2n^2)$ bits.
\end{proof}

\begin{corollary}[Measurement Entropy Scales with Depth]
The entropy cost of measurement increases with partition depth:
\begin{equation}
    \Delta S(n) = k_B(2\ln n + \ln 2) \approx 2k_B \ln n
\end{equation}
\end{corollary}

Deep fragmentation measurements (high $n$) are thermodynamically more expensive than shallow measurements.

\subsection{Heat and Entropy Decoupling}

\begin{theorem}[Heat-Entropy Decoupling]
\label{thm:decoupling}
At the single-molecule level, heat transfer $Q$ and entropy change $\Delta S$ are decoupled:
\begin{equation}
    \Delta S \neq \frac{Q}{T}
\end{equation}
The classical relation holds only in the thermodynamic limit of many molecules.
\end{theorem}

\begin{proof}
For a single categorical transition, $\Delta S = k_B \ln(g_f/g_i)$ where $g_i, g_f$ are the degeneracies of initial and final states. The heat transfer $Q = \Delta E$ depends on the energy difference. These quantities are independent: a transition to a higher-degeneracy state ($\Delta S > 0$) can occur with $Q < 0$ (exothermic) or $Q > 0$ (endothermic).

Averaging over many transitions recovers $\langle \Delta S \rangle = \langle Q \rangle / T$.
\end{proof}

This decoupling has practical implications: individual fragmentation events can decrease entropy locally while increasing it globally.

\subsection{The Biological Maxwell Demon}

Living systems perform categorical operations on molecules: synthesis, degradation, sorting. These operations appear to violate thermodynamic constraints.

\begin{definition}[Biological Maxwell Demon (BMD)]
A biological Maxwell demon is a molecular machine (enzyme, transporter, ribosome) that performs categorical operations on substrates.
\end{definition}

\begin{theorem}[BMD Thermodynamic Consistency]
\label{thm:bmd-consistency}
Biological Maxwell demons satisfy thermodynamic constraints through phase-lock network coupling:
\begin{equation}
    \Delta S_{\text{local}} + \Delta S_{\text{network}} \geq 0
\end{equation}
Local entropy decrease is compensated by network entropy increase.
\end{theorem}

\begin{proof}
The BMD couples to a phase-lock network (metabolic pathway, signaling cascade). The categorical operation on the substrate is coupled to categorical transitions in the network. Total entropy is conserved; local operations are not isolated.
\end{proof}

\subsection{Partition Lag and Irreversibility}

\begin{definition}[Partition Lag]
Partition lag $\tau_{\text{lag}}$ is the time required to establish a categorical distinction:
\begin{equation}
    \tau_{\text{lag}} = \frac{h}{\Delta E}
\end{equation}
where $\Delta E$ is the energy separation between categories.
\end{definition}

\begin{theorem}[Irreversibility from Partition Lag]
\label{thm:irreversibility}
Categorical transitions are irreversible if the partition lag exceeds the observation time:
\begin{equation}
    \tau_{\text{lag}} > \tau_{\text{obs}} \implies \text{transition is operationally irreversible}
\end{equation}
\end{theorem}

\begin{proof}
Reversing a categorical transition requires distinguishing the reverse pathway from the forward pathway. This requires time $\tau_{\text{lag}}$ to establish the distinction. If observation ceases before $\tau_{\text{lag}}$, the reverse pathway is indistinguishable from non-occurrence.
\end{proof}

\subsection{Entropy Production in Fragmentation}

\begin{proposition}[Fragmentation Entropy]
The entropy produced by fragmentation from depth $n_i$ to $n_f$ is
\begin{equation}
    \Delta S_{\text{frag}} = k_B \ln \frac{C(n_f)}{C(n_i)} = k_B \ln \frac{n_f^2}{n_i^2} = 2k_B \ln \frac{n_f}{n_i}
\end{equation}
\end{proposition}

For fragmentation from $n_i = 1$ (precursor) to $n_f = 3$ (secondary fragment):
\begin{equation}
    \Delta S = 2k_B \ln 3 \approx 2.2 k_B
\end{equation}

This entropy production is irreversible: the reverse process (fragment reassembly) would require the same entropy input, supplied by the measurement apparatus.

\subsection{Thermodynamic Efficiency of Mass Spectrometry}

\begin{definition}[Categorical Efficiency]
The categorical efficiency of a mass spectrometer is
\begin{equation}
    \eta = \frac{I_{\text{extracted}}}{W_{\text{dissipated}} / k_B T}
\end{equation}
where $I_{\text{extracted}}$ is the information gained and $W_{\text{dissipated}}$ is the work dissipated.
\end{definition}

\begin{proposition}[Landauer Bound]
The categorical efficiency is bounded:
\begin{equation}
    \eta \leq 1
\end{equation}
with equality only for reversible measurements (zero dissipation beyond the Landauer minimum).
\end{proposition}

Current mass spectrometers operate far below the Landauer bound ($\eta \ll 1$), suggesting significant room for efficiency improvements.

