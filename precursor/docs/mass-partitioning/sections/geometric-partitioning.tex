\section{Geometric Partitioning in Bounded Phase Space}
\label{sec:geometric-partitioning}

\subsection{Bounded Phase Space Structure}

\begin{definition}[Bounded Phase Space]
A bounded phase space is a compact region $\Omega \subset \mathbb{R}^{2d}$ with smooth boundary $\partial\Omega$ and finite symplectic volume
\begin{equation}
    V = \int_\Omega \omega^d < \infty
\end{equation}
where $\omega = \sum_{i=1}^d dp_i \wedge dq_i$ is the canonical symplectic form.
\end{definition}

The boundedness condition imposes constraints on trajectories. Any trajectory $\gamma(t) \subset \Omega$ must satisfy $\gamma(t) \in \Omega$ for all $t$, requiring either periodic orbits or reflection at $\partial\Omega$.

\begin{definition}[Partition Depth]
For a bounded phase space $\Omega$ with characteristic length $L$ and minimum action quantum $h$, the partition depth is
\begin{equation}
    n = \left\lfloor \frac{L \cdot p_{\max}}{h} \right\rfloor
\end{equation}
where $p_{\max}$ is the maximum momentum compatible with confinement to $\Omega$.
\end{definition}

The partition depth counts the number of distinguishable radial shells that fit within the bounded region. For a molecular ion in an electromagnetic trap, $n$ corresponds to the number of distinguishable fragmentation generations---the precursor ($n=0$), primary fragments ($n=1$), secondary fragments ($n=2$), and so forth.

\subsection{Angular Complexity and Orientation}

Within each radial shell, angular structure provides additional categorical distinctions.

\begin{definition}[Angular Complexity]
The angular complexity $l$ at partition depth $n$ counts the number of independent angular modes with non-zero amplitude, constrained by
\begin{equation}
    0 \leq l < n
\end{equation}
\end{definition}

The constraint $l < n$ follows from the requirement that angular motion fit within the radial extent. A trajectory with $l = n$ would require angular wavelength equal to radial confinement, violating the boundary condition at $\partial\Omega$.

\begin{definition}[Orientation Parameter]
The orientation parameter $m$ specifies the projection of angular structure onto a reference axis, constrained by
\begin{equation}
    -l \leq m \leq l, \quad m \in \mathbb{Z}
\end{equation}
\end{definition}

For molecular fragmentation, $l$ corresponds to the complexity of the fragmentation pathway---the number of distinct bond-breaking events required to reach a given fragment. The orientation $m$ distinguishes between pathways of equal complexity that produce different fragments.

\subsection{Chirality Parameter}

\begin{definition}[Chirality Parameter]
The chirality parameter $s$ takes values
\begin{equation}
    s \in \left\{ -\frac{1}{2}, +\frac{1}{2} \right\}
\end{equation}
encoding the handedness of the categorical state under parity transformation.
\end{definition}

The half-integer nature of $s$ emerges from the requirement that the wave function acquire a sign under $2\pi$ rotation of the reference frame. This is not a quantum mechanical postulate but a geometric consequence of the double-cover structure of $SO(3)$ by $SU(2)$.

For molecular systems, $s$ encodes stereochemical configuration at chiral centers. The binary nature of chirality (R/S, D/L, +/-) corresponds to the two values $s = \pm 1/2$.

\subsection{Partition Coordinate System}

\begin{definition}[Partition Coordinates]
The partition coordinates of a categorical state in bounded phase space are the 4-tuple
\begin{equation}
    (n, l, m, s)
\end{equation}
where $n \in \mathbb{Z}^+$, $l \in \{0, 1, \ldots, n-1\}$, $m \in \{-l, -l+1, \ldots, l\}$, and $s \in \{-1/2, +1/2\}$.
\end{definition}

\begin{theorem}[Completeness]
\label{thm:completeness}
The partition coordinates $(n, l, m, s)$ provide a complete specification of categorical states in bounded phase space. Any two states with identical coordinates are categorically indistinguishable.
\end{theorem}

\begin{proof}
Suppose two states $\psi_1, \psi_2$ share coordinates $(n, l, m, s)$. The coordinate $n$ fixes the radial extent, $l$ fixes the angular complexity, $m$ fixes the orientation, and $s$ fixes the chirality. These four parameters exhaust the degrees of freedom in a bounded region with spherical symmetry. Any additional distinction would require a fifth independent parameter, but the symplectic structure of phase space in $d=3$ spatial dimensions admits only four independent constraints compatible with the boundary $\partial\Omega$.
\end{proof}

\begin{theorem}[Discreteness]
\label{thm:discreteness}
The set of partition coordinates is discrete: there exists $\epsilon > 0$ such that any two distinct coordinate assignments differ by at least $\epsilon$ in at least one coordinate.
\end{theorem}

\begin{proof}
The constraints $n \in \mathbb{Z}^+$, $l \in \mathbb{Z}_{\geq 0}$, $m \in \mathbb{Z}$, and $s \in \{-1/2, +1/2\}$ define a discrete lattice in the parameter space. The minimum separation is $1$ for $n$, $l$, $m$ and $1$ for $s$.
\end{proof}

This discrete structure has immediate consequences for mass spectrometry. Fragment masses do not vary continuously but occupy discrete positions determined by the partition coordinates of the fragmenting molecular ion. The mass spectrum is thus a projection of the partition coordinate lattice onto the mass axis.

