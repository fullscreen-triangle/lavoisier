\section{Vibrational Modality: Molecular Bond Characterization}
\label{sec:vibrational}

\subsection{Vibrational Spectroscopy Fundamentals}

Molecular vibrations occur at characteristic frequencies determined by bond strengths and reduced masses \cite{Herzberg1945}. Harmonic approximation gives:
\begin{equation}
\omega_{\text{vib}} = \sqrt{\frac{k}{\mu}}
\end{equation}
where $k$ is force constant and $\mu = m_1 m_2/(m_1 + m_2)$ is reduced mass.

Common biological bonds have frequencies:
\begin{align}
\omega_{C-H} &\sim 2900\text{ cm}^{-1} \quad (8.7 \times 10^{13}\text{ Hz}) \\
\omega_{C=O} &\sim 1650\text{ cm}^{-1} \quad (4.9 \times 10^{13}\text{ Hz}) \\
\omega_{C-N} &\sim 1200\text{ cm}^{-1} \quad (3.6 \times 10^{13}\text{ Hz}) \\
\omega_{O-H} &\sim 3300\text{ cm}^{-1} \quad (9.9 \times 10^{13}\text{ Hz})
\end{align}

Raman spectroscopy measures inelastic light scattering with frequency shift $\Delta\omega = \omega_{\text{incident}} - \omega_{\text{scattered}}$ equal to vibrational frequency \cite{Raman1928}.

\subsection{Vibrational Partition Coordinates}

Vibrational quantum states labeled by quantum number $v = 0, 1, 2, \ldots$ with energies:
\begin{equation}
E_v = \hbar\omega_{\text{vib}}\left(v + \frac{1}{2}\right)
\end{equation}

At temperature $T$, Boltzmann distribution gives populations:
\begin{equation}
P_v = \frac{e^{-E_v/k_B T}}{\sum_v e^{-E_v/k_B T}}
\end{equation}

For $k_B T \ll \hbar\omega_{\text{vib}}$, most molecules in ground state $v = 0$. For biological systems at $T = 310$ K:
\begin{equation}
k_B T \approx 215\text{ cm}^{-1}
\end{equation}

Thus vibrations with $\omega > 500$ cm$^{-1}$ are predominantly ground state, while lower frequency modes have thermal population.

\subsection{Phase Determination from Vibrational Signatures}

Vibrational frequencies depend on molecular environment. Solid phase exhibits narrow lines:
\begin{equation}
\Gamma_{\text{solid}} \sim 1-10\text{ cm}^{-1}
\end{equation}
due to restricted motion. Liquid phase shows broader lines:
\begin{equation}
\Gamma_{\text{liquid}} \sim 10-100\text{ cm}^{-1}
\end{equation}
from rapid reorientation. Gas phase exhibits rotational structure with spacing:
\begin{equation}
\Delta\omega_{\text{rot}} \sim 1-10\text{ cm}^{-1}
\end{equation}

Measuring linewidth $\Gamma$ and structure determines phase:
\begin{align}
\Gamma < 10\text{ cm}^{-1} &\Rightarrow \text{solid phase} \\
10 < \Gamma < 100\text{ cm}^{-1} &\Rightarrow \text{liquid phase} \\
\text{Rotational structure} &\Rightarrow \text{gas phase}
\end{align}

\subsection{Exclusion Through Vibrational Mismatch}

\begin{theorem}[Vibrational Exclusion]
\label{thm:vibrational_exclusion}
Candidate structures predicting vibrational frequencies $\{\omega_i^{\text{pred}}\}$ differing from measured $\{\omega_i^{\text{meas}}\}$ by more than line width $\Gamma$ are excluded:
\begin{equation}
|\omega_i^{\text{pred}} - \omega_i^{\text{meas}}| > \Gamma \quad \Rightarrow \quad \text{excluded}
\end{equation}
\end{theorem}

\begin{proof}
Vibrational frequency is determined by force constants through $\omega = \sqrt{k/\mu}$. Different molecular structures have different force constants, thus different frequencies. Measurement precision is limited by linewidth $\Gamma$; frequencies differing by more than $\Gamma$ are distinguishable.
\end{proof}

For high-resolution Raman with $\Gamma \sim 1$ cm$^{-1}$ over range $500-3500$ cm$^{-1}$, number of resolvable features:
\begin{equation}
N_{\text{features}} \sim \frac{3000}{1} = 3000
\end{equation}

Not all features are independent; typical molecule has $\sim 100$ normal modes. Accounting for mode coupling and selection rules, effective independent features $\sim 30$, giving exclusion factor:
\begin{equation}
\epsilon_{\text{vibrational}} \sim (0.1)^{30} \sim 10^{-30}
\end{equation}

Conservatively, considering measurement noise and peak overlap:
\begin{equation}
\epsilon_{\text{vibrational}} \sim 10^{-15}
\end{equation}

\subsection{Temporal Resolution from Vibrational Dynamics}

Vibrational periods $\tau_{\text{vib}} = 2\pi/\omega_{\text{vib}}$ range from:
\begin{equation}
\tau_{\text{vib}} \sim 10-100\text{ fs}
\end{equation}

Time-resolved vibrational spectroscopy with pump-probe techniques \cite{Zewail2000} achieves temporal resolution:
\begin{equation}
\delta t \sim \tau_{\text{vib}} \sim 10-100\text{ fs}
\end{equation}

This enables tracking molecular dynamics orders of magnitude faster than optical frame rates limited by detector readout ($\sim$ ns-$\mu$s).

