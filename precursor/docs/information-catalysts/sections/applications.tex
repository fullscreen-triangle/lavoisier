\section{Applications}
\label{sec:applications}

\subsection{Structural Classification}

\begin{definition}[Terminator Projection]
The terminator projection of configuration $\rho$ is the vector:
\begin{equation}
    \mathbf{T}(\rho) = (w_1, w_2, \ldots, w_K)
\end{equation}
where $w_j$ is the probability that a cascade starting from $\rho$ terminates at $\rho_j^*$.
\end{definition}

\begin{theorem}[Classification by Terminator Projection]
\label{thm:classification}
Two configurations $\rho_1, \rho_2$ belong to the same structural class if and only if:
\begin{equation}
    \|\mathbf{T}(\rho_1) - \mathbf{T}(\rho_2)\| < \epsilon
\end{equation}
for sufficiently small $\epsilon > 0$.
\end{theorem}

\begin{proof}
The terminator projection encodes the structural information of $\rho$ in compressed form: which cascade endpoints are reachable and with what probability. Configurations with identical projections have identical reachable terminators, implying they share the same structural features that determine cascade pathways.
\end{proof}

\begin{corollary}[Reduced-Dimensional Classification]
Classification in terminator projection space has dimension $K$ (number of terminators), which is $O(n^2/\log n)$ compared to full configuration space dimension $O(n^2)$.
\end{corollary}

\subsection{Database Construction}

\begin{definition}[Terminator-Anchored Database]
A terminator-anchored database $\mathcal{D}$ is a collection of entries $(S, \mathbf{T}_S)$ where $S$ is a structural identifier and $\mathbf{T}_S$ is the terminator projection.
\end{definition}

\begin{proposition}[Database Query Efficiency]
Query by terminator projection has complexity $O(K \cdot |\mathcal{D}|)$ compared to $O(n^2 \cdot |\mathcal{D}|)$ for full configuration matching.
\end{proposition}

\begin{algorithm}
\caption{Terminator-Anchored Database Query}
\label{alg:database-query}
\begin{algorithmic}[1]
\Require Query observation $\mathcal{O}_q$, database $\mathcal{D}$, threshold $\tau$
\Ensure Matched structures $\{S_i\}$ with scores $\{s_i\}$
\State Extract terminators from $\mathcal{O}_q$: $\{\rho_j^*\}$
\State Compute query projection: $\mathbf{T}_q = (f_1, f_2, \ldots, f_K)$ where $f_j = f_{\text{obs}}(\rho_j^*)$
\State Normalize: $\hat{\mathbf{T}}_q = \mathbf{T}_q / \|\mathbf{T}_q\|$
\For{each entry $(S, \mathbf{T}_S) \in \mathcal{D}$}
    \State Compute similarity: $s = \hat{\mathbf{T}}_q \cdot \hat{\mathbf{T}}_S$
    \If{$s > \tau$}
        \State Add $(S, s)$ to matches
    \EndIf
\EndFor
\State Sort matches by score
\State \Return top matches
\end{algorithmic}
\end{algorithm}

\subsection{De Novo Structure Inference}

\begin{definition}[Inverse Terminator Problem]
Given observed terminator distribution $\{(\rho_j^*, f_j)\}$, find parent configurations $\{\rho_0\}$ consistent with the observations.
\end{definition}

\begin{theorem}[Structural Constraints from Terminators]
\label{thm:structural-constraints}
Each observed terminator $\rho_j^*$ with frequency $f_j$ implies:
\begin{enumerate}
    \item Parent contains structural motif $M_j$ that cascades to $\rho_j^*$
    \item The motif has multiplicity $\propto f_j / g(\rho_j^*)$
    \item Motifs must be spatially compatible in the parent structure
\end{enumerate}
\end{theorem}

\begin{proof}
(1) By definition of cascade, $\rho_j^*$ is reachable only from parents containing appropriate structural features. (2) The observed frequency is $f_j = g(\rho_j^*) \cdot P(M_j)$, where $P(M_j)$ is the probability of the motif. (3) All motifs must fit within a single connected parent structure.
\end{proof}

\begin{algorithm}
\caption{De Novo Inference from Terminators}
\label{alg:denovo}
\begin{algorithmic}[1]
\Require Observed terminators $\{(\rho_j^*, f_j)\}$, structural motif library $\mathcal{M}$
\Ensure Candidate parent structures $\{\rho_0\}$
\State \textbf{Step 1: Motif Identification}
\For{each terminator $\rho_j^*$}
    \State Look up compatible motifs: $M_j \gets \mathcal{M}[\rho_j^*]$
\EndFor
\State \textbf{Step 2: Motif Combination}
\State Generate candidate parents by combining motifs
\For{each combination $C = (M_{j_1}, M_{j_2}, \ldots)$}
    \If{motifs spatially compatible}
        \State Construct parent $\rho_0$ from $C$
        \State Score: $s = \prod_j P(f_j | \rho_0)$
    \EndIf
\EndFor
\State \textbf{Step 3: Ranking}
\State Sort candidates by score
\State \Return top candidates
\end{algorithmic}
\end{algorithm}

\subsection{Quality Assessment}

\begin{definition}[Terminator Completeness]
The completeness of observation $\mathcal{O}$ relative to reference structure $S$ is:
\begin{equation}
    C(\mathcal{O}; S) = \frac{|\{\rho_j^* \in \mathcal{O} : \rho_j^* \in \mathcal{T}(S)\}|}{|\mathcal{T}(S)|}
\end{equation}
where $\mathcal{T}(S)$ is the set of expected terminators for structure $S$.
\end{definition}

\begin{proposition}[Quality from Completeness]
Low completeness $C < C_{\text{threshold}}$ indicates:
\begin{enumerate}
    \item Incomplete cascade (insufficient activation)
    \item Incorrect structural assignment
    \item Mixture or contamination
\end{enumerate}
\end{proposition}

\begin{definition}[Terminator Consistency]
The consistency of observation $\mathcal{O}$ is:
\begin{equation}
    K(\mathcal{O}) = 1 - \frac{|\{\rho^* \in \mathcal{O} : \rho^* \text{ not in any } \mathcal{T}(S)\}|}{|\mathcal{O}|}
\end{equation}
for matched structure $S$.
\end{definition}

\begin{proposition}[Quality from Consistency]
Low consistency $K < K_{\text{threshold}}$ indicates:
\begin{enumerate}
    \item Unexpected cascade pathways (novel chemistry)
    \item Incorrect structural assignment
    \item Artifact or noise
\end{enumerate}
\end{proposition}

\subsection{Mixture Analysis}

\begin{theorem}[Mixture Decomposition]
\label{thm:mixture}
For a mixture of $M$ components with mole fractions $\{x_m\}$, the observed terminator distribution is:
\begin{equation}
    f_{\text{obs}}(\rho_j^*) = \sum_{m=1}^M x_m \cdot f_m(\rho_j^*)
\end{equation}
where $f_m(\rho_j^*)$ is the terminator frequency for pure component $m$.
\end{theorem}

\begin{proof}
Linearity of terminator observation: each component contributes independently to the observed terminator pool, weighted by its mole fraction.
\end{proof}

\begin{corollary}[Mixture Deconvolution]
Given reference terminator distributions $\{f_m\}$ for candidate components, the mole fractions can be estimated by non-negative least squares:
\begin{equation}
    \hat{\mathbf{x}} = \arg\min_{\mathbf{x} \geq 0, \sum x_m = 1} \| \mathbf{f}_{\text{obs}} - \mathbf{F} \mathbf{x} \|^2
\end{equation}
where $\mathbf{F}$ is the matrix with columns $\mathbf{f}_m$.
\end{corollary}

In mass spectrometry, these applications translate to:
\begin{itemize}
    \item Structural classification by fragment ion fingerprinting
    \item Spectral library construction anchored on diagnostic ions
    \item De novo structure elucidation from fragmentation patterns
    \item Spectral quality assessment by expected fragment coverage
    \item Mixture deconvolution by fragment intensity fitting
\end{itemize}

