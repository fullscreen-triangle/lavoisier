\section{Cascade Dynamics}
\label{sec:partition-cascade-dynamics}

\subsection{Master Equation}

\begin{definition}[Cascade State]
A cascade state is a pair $(n, \rho)$ where $n \in \mathbb{Z}_{\geq 0}$ is the partition depth and $\rho \in \Cspace$ is the current configuration.
\end{definition}

\begin{definition}[Cascade Propagator]
The cascade propagator $G(n', \rho'; n, \rho; t)$ is the probability of being in state $(n', \rho')$ at time $t$ given initial state $(n, \rho)$ at time $0$.
\end{definition}

\begin{theorem}[Master Equation]
\label{thm:master-equation}
The cascade propagator satisfies:
\begin{equation}
    \frac{\partial G}{\partial t} = \sum_{\Pi} \left[ r_\Pi(\rho) G(n, \rho'; n-1, \rho_{\text{parent}}; t) - r_\Pi(\rho') G(n', \rho'; n, \rho; t) \right]
\end{equation}
where the sum is over all partition operators $\Pi$ that connect $(n-1, \rho_{\text{parent}})$ to $(n, \rho)$.
\end{theorem}

\begin{proof}
Standard derivation for continuous-time Markov processes on state spaces with birth-death dynamics. The first term represents gain from partitions producing state $(n', \rho')$; the second represents loss from partitions removing population from $(n', \rho')$.
\end{proof}

\subsection{Moment Equations}

\begin{definition}[Partition Depth Moments]
The $k$-th moment of the partition depth distribution at time $t$ is:
\begin{equation}
    \langle n^k \rangle(t) = \sum_n \int_\Cspace n^k P(n, \rho; t) \, d\rho
\end{equation}
where $P(n, \rho; t)$ is the probability density over cascade states.
\end{definition}

\begin{theorem}[Mean Depth Evolution]
\label{thm:mean-depth}
The mean partition depth evolves as:
\begin{equation}
    \frac{d\langle n \rangle}{dt} = \langle r(\rho) \rangle
\end{equation}
where $\langle r(\rho) \rangle$ is the configuration-averaged partition rate.
\end{theorem}

\begin{proof}
\begin{align}
    \frac{d\langle n \rangle}{dt} &= \sum_n \int n \frac{\partial P}{\partial t} \, d\rho \\
    &= \sum_n \int n \left[ r(\rho') P(n-1, \rho') - r(\rho) P(n, \rho) \right] d\rho \\
    &= \sum_n \int (n+1) r(\rho) P(n, \rho) \, d\rho - \sum_n \int n r(\rho) P(n, \rho) \, d\rho \\
    &= \sum_n \int r(\rho) P(n, \rho) \, d\rho = \langle r(\rho) \rangle
\end{align}
\end{proof}

\begin{theorem}[Variance Evolution]
\label{thm:variance}
The variance $\sigma_n^2 = \langle n^2 \rangle - \langle n \rangle^2$ evolves as:
\begin{equation}
    \frac{d\sigma_n^2}{dt} = \langle r(\rho) \rangle + 2 \text{Cov}(n, r(\rho))
\end{equation}
\end{theorem}

\begin{proof}
Differentiate $\langle n^2 \rangle$ using the master equation and subtract $2\langle n \rangle d\langle n \rangle/dt$.
\end{proof}

\begin{corollary}[Autocatalytic Variance Enhancement]
For autocatalytic dynamics where $r(\rho)$ increases with partition depth $n$, the covariance $\text{Cov}(n, r) > 0$, leading to enhanced variance:
\begin{equation}
    \sigma_n^2 > \langle n \rangle \quad \text{(overdispersion)}
\end{equation}
In contrast, simple kinetics has $\sigma_n^2 = \langle n \rangle$ (Poisson).
\end{corollary}

\subsection{Lag-Exponential-Saturation Profile}

\begin{theorem}[Three-Phase Kinetics]
\label{thm:three-phase}
Autocatalytic cascade dynamics exhibit three distinct phases:
\begin{enumerate}
    \item \textbf{Lag phase} ($t < t_{\text{lag}}$): Mean depth grows linearly, $\langle n \rangle \approx r_1^{(0)} t$
    \item \textbf{Exponential phase} ($t_{\text{lag}} < t < t_{\text{sat}}$): Mean depth grows exponentially, $\langle n \rangle \propto e^{\bar{\beta} r_1^{(0)} t}$
    \item \textbf{Saturation phase} ($t > t_{\text{sat}}$): Mean depth approaches limiting value, $\langle n \rangle \to n_{\max}$
\end{enumerate}
\end{theorem}

\begin{proof}
\textbf{Lag phase:} At early times, few partitions have occurred, so the autocatalytic enhancement $e^{n\bar{\beta}}$ is negligible. The dynamics reduce to simple first-order kinetics with $d\langle n \rangle/dt \approx r_1^{(0)}$.

\textbf{Exponential phase:} Once sufficient partitions have occurred, the autocatalytic enhancement becomes significant. With $r_n \approx r_1^{(0)} e^{n\bar{\beta}}$:
\begin{equation}
    \frac{d\langle n \rangle}{dt} \approx r_1^{(0)} e^{\langle n \rangle \bar{\beta}}
\end{equation}
This has solution $\langle n \rangle = \bar{\beta}^{-1} \ln(1 + \bar{\beta} r_1^{(0)} t)$, which approximates $e^{\bar{\beta} r_1^{(0)} t}$ for moderate $t$.

\textbf{Saturation phase:} The cascade terminates when all pathways reach terminators. The maximum depth $n_{\max}$ is set by the terminator distribution.
\end{proof}

\begin{definition}[Characteristic Times]
\begin{align}
    t_{\text{lag}} &= \frac{1}{\bar{\beta} r_1^{(0)}} \quad \text{(time to onset of autocatalysis)} \\
    t_{\text{sat}} &= \frac{n_{\max}}{\bar{\beta} r_1^{(0)}} \quad \text{(time to terminator saturation)}
\end{align}
\end{definition}

\subsection{Terminator Accumulation}

\begin{theorem}[Terminator Population Dynamics]
\label{thm:terminator-accumulation}
The population of terminator $\rho_i^*$ at time $t$ is:
\begin{equation}
    P_i^*(t) = \int_0^t J_i^*(t') \, dt'
\end{equation}
where $J_i^*(t)$ is the flux into terminator $i$:
\begin{equation}
    J_i^*(t) = \sum_{\rho \to \rho_i^*} r_{\text{term}}(\rho) P(\tau(\rho_i^*) - 1, \rho; t)
\end{equation}
\end{theorem}

\begin{proof}
The terminator population increases by the flux of configurations partitioning into the terminator and does not decrease (terminators are stable). The flux is the sum over all configurations $\rho$ that can partition into $\rho_i^*$, weighted by their population and partition rate.
\end{proof}

\begin{corollary}[Steady-State Terminator Distribution]
At long times ($t \gg t_{\text{sat}}$), the terminator population ratio is:
\begin{equation}
    \frac{P_i^*}{P_j^*} = \frac{g(\rho_i^*)}{g(\rho_j^*)}
\end{equation}
where $g(\rho_i^*)$ is the pathway degeneracy (number of distinct cascade pathways leading to $\rho_i^*$).
\end{corollary}

The three-phase kinetic profile and overdispersion provide experimental signatures distinguishing autocatalytic cascades from simple sequential dissociation. In mass spectrometry, these signatures manifest in collision energy-dependent fragmentation patterns.

