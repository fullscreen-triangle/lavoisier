\section{Charge Partition Spaces}
\label{sec:charge-partitioning}

\subsection{Configuration Space}

\begin{definition}[Charge Configuration]
A charge configuration on a bounded region $\Omega \subset \mathbb{R}^3$ is a signed measure $\rho: \Omega \to \mathbb{R}$ satisfying:
\begin{equation}
    Q = \int_\Omega \rho(\mathbf{r}) \, d^3r
\end{equation}
where $Q$ is the total charge.
\end{definition}

\begin{definition}[Configuration Space]
The configuration space $\Cspace(Q)$ is the set of all charge configurations with total charge $Q$:
\begin{equation}
    \Cspace(Q) = \left\{ \rho: \Omega \to \mathbb{R} \,\middle|\, \int_\Omega \rho \, d^3r = Q \right\}
\end{equation}
\end{definition}

The configuration space admits a metric structure induced by the electrostatic energy functional.

\begin{definition}[Electrostatic Distance]
The electrostatic distance between configurations $\rho_1, \rho_2 \in \Cspace(Q)$ is:
\begin{equation}
    d_E(\rho_1, \rho_2) = \left( \int_\Omega \int_\Omega \frac{(\rho_1(\mathbf{r}) - \rho_2(\mathbf{r}))(\rho_1(\mathbf{r}') - \rho_2(\mathbf{r}'))}{|\mathbf{r} - \mathbf{r}'|} \, d^3r \, d^3r' \right)^{1/2}
\end{equation}
\end{definition}

\begin{proposition}[$\Cspace(Q)$ is a Metric Space]
The pair $(\Cspace(Q), d_E)$ is a complete metric space.
\end{proposition}

\begin{proof}
Positive definiteness: $d_E(\rho_1, \rho_2) \geq 0$ with equality iff $\rho_1 = \rho_2$ almost everywhere (by uniqueness of the Coulomb potential). Symmetry: immediate from the definition. Triangle inequality: follows from the Minkowski inequality for the $L^2$ norm of the difference charge distribution weighted by the Coulomb kernel. Completeness: $\Cspace(Q)$ is a closed subset of $L^2(\Omega)$, which is complete.
\end{proof}

\subsection{Partition Operators}

\begin{definition}[Partition Operator]
A partition operator $\Pi: \Cspace(Q) \to \Cspace(Q_1) \times \Cspace(Q_2)$ maps a configuration to a pair of daughter configurations:
\begin{equation}
    \Pi(\rho) = (\rho_1, \rho_2)
\end{equation}
subject to the conservation constraint:
\begin{equation}
    \rho_1 + \rho_2 = \rho, \quad Q_1 + Q_2 = Q
\end{equation}
\end{definition}

\begin{definition}[Partition Axis]
The partition axis $\mathbf{a} \in S^2$ specifies the direction along which charge is separated. For a partition with axis $\mathbf{a}$:
\begin{equation}
    \rho_1(\mathbf{r}) = \rho(\mathbf{r}) \cdot \Theta(\mathbf{a} \cdot \mathbf{r} - c), \quad \rho_2(\mathbf{r}) = \rho(\mathbf{r}) \cdot \Theta(c - \mathbf{a} \cdot \mathbf{r})
\end{equation}
where $\Theta$ is the Heaviside function and $c$ is the partition plane offset.
\end{definition}

\begin{definition}[Charge Separation]
The charge separation induced by partition $\Pi$ is:
\begin{equation}
    \Delta Q = |Q_1 - Q_2|
\end{equation}
The dipole moment created is:
\begin{equation}
    \mathbf{p} = Q_1 \mathbf{r}_1 - Q_2 \mathbf{r}_2
\end{equation}
where $\mathbf{r}_1, \mathbf{r}_2$ are the centroids of the daughter distributions.
\end{definition}

\subsection{Partition Energy}

\begin{proposition}[Partition Energy Cost]
The electrostatic energy cost of partition $\Pi$ is:
\begin{equation}
    \Delta E_\Pi = \frac{Q_1 Q_2}{|\mathbf{r}_1 - \mathbf{r}_2|} - E_{\text{int}}(\rho)
\end{equation}
where $E_{\text{int}}(\rho)$ is the internal electrostatic energy of the parent configuration.
\end{proposition}

\begin{proof}
Before partition, the electrostatic energy is:
\begin{equation}
    E_{\text{before}} = \frac{1}{2} \int \int \frac{\rho(\mathbf{r}) \rho(\mathbf{r}')}{|\mathbf{r} - \mathbf{r}'|} \, d^3r \, d^3r' = E_{\text{int}}(\rho)
\end{equation}

After partition, the energy is the sum of self-energies and interaction energy:
\begin{equation}
    E_{\text{after}} = E_{\text{int}}(\rho_1) + E_{\text{int}}(\rho_2) + \frac{Q_1 Q_2}{|\mathbf{r}_1 - \mathbf{r}_2|}
\end{equation}

For a sharp partition, $E_{\text{int}}(\rho_1) + E_{\text{int}}(\rho_2) \approx E_{\text{int}}(\rho)$, yielding the stated result.
\end{proof}

\begin{corollary}[Partition Favorability]
Partition $\Pi$ is energetically favorable if:
\begin{equation}
    \frac{Q_1 Q_2}{|\mathbf{r}_1 - \mathbf{r}_2|} < E_{\text{int}}(\rho)
\end{equation}
Partitions that create large spatial separation $|\mathbf{r}_1 - \mathbf{r}_2|$ are favored.
\end{corollary}

\subsection{Sequential Partitions}

\begin{definition}[Partition Sequence]
A partition sequence of depth $n$ is an ordered sequence of partition operators:
\begin{equation}
    \mathcal{S}_n = (\Pi_1, \Pi_2, \ldots, \Pi_n)
\end{equation}
The $n$-fold composition $\mathcal{S}_n(\rho_0)$ generates a tree of $2^n$ terminal configurations.
\end{definition}

\begin{definition}[Partition Tree]
The partition tree $\mathcal{T}(\rho_0, \mathcal{S}_n)$ is the directed graph where:
\begin{itemize}
    \item Nodes are configurations reached during the sequence
    \item Edges are partition operations
    \item The root is $\rho_0$
    \item Leaves are terminal configurations at depth $n$
\end{itemize}
\end{definition}

\begin{proposition}[Leaf Count]
The partition tree has exactly $2^n$ leaves for depth $n$, but the number of distinct configurations may be smaller due to convergent pathways.
\end{proposition}

In molecular ion fragmentation, the partition operator corresponds to bond cleavage. The parent charge configuration is the precursor ion; daughter configurations are fragment ions and neutrals. The partition axis corresponds to the breaking bond. The charge separation $\Delta Q$ is typically $\pm e$ for singly charged systems.

