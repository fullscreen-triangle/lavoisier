% Figure Captions for Single-Ion Observatory Paper
% =================================================
% All figures focus on target ion m/z = 299.0555
% Horizontal format: 1 row x 4 charts per panel

\begin{figure*}[!htbp]
    \centering
    \includegraphics[width=\textwidth]{figures/panel_1_bounded_phase_space.png}
    \caption{\textbf{Bounded Phase Space: Finite State Structure in Ion Traps.}
    (\textbf{A}) Bounded phase space in $(x, p)$ coordinates showing circular boundary (black) with concentric energy shells (purple gradient). The target ion at $m/z = 299.0555$ (gold star) occupies a specific region within the bounded domain. The finite boundary ensures countable categorical states.
    (\textbf{B}) Discrete partition cells tiling the phase space without gaps or overlaps. Each cell corresponds to a distinguishable categorical state labeled by quantum numbers. The target cell at $n = 5$ is highlighted, demonstrating that continuous phase space admits discrete labeling.
    (\textbf{C}) Energy level structure showing $E_n \propto n^2$ scaling with degeneracy $C(n) = 2n^2$ at each level. The quadratic growth of accessible states with principal quantum number is characteristic of bounded systems. Points on each level represent distinct $(l, m, s)$ configurations.
    (\textbf{D}) Capacity scaling showing both shell capacity $C(n) = 2n^2$ (bars) and cumulative state count $\sum C(n)$ (green bars) as function of principal number. The target ion level $n = 5$ with $C(5) = 50$ states is marked. Theoretical $2n^2$ curve (red dashed) confirms capacity formula.}
    \label{fig:bounded_phase_space}
\end{figure*}

\begin{figure*}[!htbp]
    \centering
    \includegraphics[width=\textwidth]{figures/panel_2_partition_coordinates.png}
    \caption{\textbf{Partition Coordinates: Complete Categorical State Description.}
    (\textbf{A}) Three-dimensional partition state space $(n, l, m)$ showing all accessible states for $n = 1$ to $6$. Each point represents a valid $(n, l, m)$ configuration, color-coded by principal number. The target ion state $(n=5, l=4, m=-4)$ is marked with a gold star. The hierarchical structure is evident: higher $n$ admits more $(l, m)$ combinations.
    (\textbf{B}) Capacity formula verification showing measured capacity $C(n) = 2n^2$ for each shell (blue bars) versus theoretical prediction (red dashed line). The target ion shell $n = 5$ with $C(5) = 50$ states is highlighted. Perfect agreement confirms the partition capacity theorem.
    (\textbf{C}) Angular momentum states at fixed $n = 5$ showing all valid $(l, m)$ pairs. For each $l \in \{0, 1, 2, 3, 4\}$, the orientation $m$ ranges from $-l$ to $+l$. The target state $(l=4, m=-4)$ is marked. This structure generates exactly $n^2 = 25$ pairs, doubled by spin to give $C(5) = 50$.
    (\textbf{D}) Chirality distribution showing spin states $s = \pm 1/2$ for the ion population. The target ion chirality $s = -0.5$ (negative ion mode) is marked. The two spin states provide the factor of 2 in the capacity formula $C(n) = 2n^2$.}
    \label{fig:partition_coordinates}
\end{figure*}

\begin{figure*}[!htbp]
    \centering
    \includegraphics[width=\textwidth]{figures/panel_3_commutation_relations.png}
    \caption{\textbf{Commutation Relations: Simultaneous Measurement Without Backaction.}
    (\textbf{A}) Commutator matrix for partition coordinate operators $(\hat{n}, \hat{l}, \hat{m}, \hat{s})$ showing all zeros. Unlike conjugate variables $(x, p)$ with $[x, p] = i\hbar$, partition coordinates mutually commute: $[\hat{A}, \hat{B}] = 0$ for all pairs. This enables simultaneous eigenstate measurement.
    (\textbf{B}) Simultaneous eigenstate $|n, l, m, s\rangle = |5, 4, -4, -0.5\rangle$ for the target ion. All four eigenvalues are sharp and simultaneously measurable. No generalized uncertainty relation constrains joint determination of partition coordinates.
    (\textbf{C}) Measurement backaction $\Delta p / p$ as function of trap size $L$. The quantum backaction (blue curve) falls far below thermal fluctuations (red dashed) for typical trap dimensions. At $L = 1$ mm, backaction is $\sim 10^{-11}$, validating classical treatment of partition measurement.
    (\textbf{D}) Quantum non-demolition (QND) measurement sequence showing repeated measurements of $n$ and $l$ yielding identical values without state disturbance. The constancy confirms that partition coordinates are conserved observables: $[\hat{O}, \hat{H}] = 0$.}
    \label{fig:commutation_relations}
\end{figure*}

\begin{figure*}[!htbp]
    \centering
    \includegraphics[width=\textwidth]{figures/panel_4_triple_equivalence.png}
    \caption{\textbf{Triple Equivalence: Oscillatory, Categorical, and Partition Views.}
    (\textbf{A}) Oscillatory description showing harmonic motion $x(t) = A\cos(\omega t)$ with quantum levels $n = 1, 2, \ldots, 5$ superimposed. The ion at $m/z = 299.0555$ executes periodic motion at $\omega = 2\pi \times 1$ MHz with amplitude determined by trap parameters.
    (\textbf{B}) Categorical description showing $M = C(n) = 50$ distinguishable states arranged in cyclic sequence. Each point on the arc represents a categorical state the ion traverses during one oscillation period. The current state (gold star) demonstrates discrete state occupation.
    (\textbf{C}) Partition description showing binary tree structure with $n^M$ terminal states (leaves). The hierarchical partitioning of phase space generates the capacity formula through recursive subdivision. Depth $M$ determines total accessible states.
    (\textbf{D}) Equivalence diagram (Venn) showing three overlapping perspectives converging to the fundamental entropy formula $S = k_B M \ln n$. Oscillatory frequency, categorical state count, and partition depth are equivalent descriptions of the same bounded dynamics.}
    \label{fig:triple_equivalence}
\end{figure*}

\begin{figure*}[!htbp]
    \centering
    \includegraphics[width=\textwidth]{figures/panel_5_thermodynamics.png}
    \caption{\textbf{Single-Ion Thermodynamics: Categorical Temperature and Equations of State.}
    (\textbf{A}) Categorical temperature $T_{\text{cat}} = \hbar\omega / (2\pi k_B)$ as function of RF frequency. For the target ion at $\omega = 2\pi \times 1$ MHz, $T_{\text{cat}} \approx 7.6$ $\mu$K. This information temperature measures state exploration rate, not thermal energy.
    (\textbf{B}) Single-ion ideal gas law $PV = k_B T_{\text{cat}}$ showing categorical pressure versus trap volume. The ion satisfies thermodynamic relations through temporal (not ensemble) averaging over categorical states.
    (\textbf{C}) Maximum categorical entropy $S_{\max} = k_B \ln(2n^2)$ scaling logarithmically with principal number. The target ion at $n = 5$ has $S_{\max}/k_B = \ln 50 \approx 3.9$. The scaling $\ln 2 + 2\ln n$ confirms partition structure.
    (\textbf{D}) Bounded Maxwell-Boltzmann distribution showing natural truncation at $v_{\max}$ (gold vertical line). The categorical framework produces finite velocity distributions without ad hoc cutoffs, resolving classical divergences including the ultraviolet catastrophe.}
    \label{fig:thermodynamics}
\end{figure*}

\begin{figure*}[!htbp]
    \centering
    \includegraphics[width=\textwidth]{figures/panel_6_ternary_representation.png}
    \caption{\textbf{Ternary Representation: Address Encoding and Position-Trajectory Duality.}
    (\textbf{A}) Ternary address space showing base-3 encoding with depth $k = 4$. Each node branches into three children (trits 0, 1, 2), generating $3^k = 81$ terminal addresses. The highlighted path shows a specific ternary address encoding the target ion's partition coordinates.
    (\textbf{B}) Position-trajectory duality demonstrating that a ternary address simultaneously encodes both final position $x$ and the complete path through partition space. This dual encoding is fundamental to the partition coordinate framework.
    (\textbf{C}) Emergent continuity showing discrete ternary grids at depths $k = 1, 2, 3, 4$ converging to the continuous interval $[0, 1]$ as $k \to \infty$. The discrete-to-continuous transition is smooth, validating the framework's classical limit.
    (\textbf{D}) Ternary-to-partition mapping showing how decimal ternary addresses convert to partition coordinates $(n, l, m)$. The bijective mapping ensures unique state identification from ternary encoding.}
    \label{fig:ternary_representation}
\end{figure*}

\begin{figure*}[!htbp]
    \centering
    \includegraphics[width=\textwidth]{figures/panel_7_s_entropy_coordinates.png}
    \caption{\textbf{S-Entropy Coordinates: Universal Normalized Representation.}
    (\textbf{A}) Three-dimensional S-entropy space $(S_k, S_t, S_e) \in [0,1]^3$ showing ion population distribution. Each point represents an ion colored by intensity (plasma colormap). The target ion at $m/z = 299.0555$ (gold star) is located by its knowledge, temporal, and evolution coordinates.
    (\textbf{B}) Individual S-coordinate distributions showing $S_k$ (knowledge/mass), $S_t$ (temporal/retention), and $S_e$ (evolution/fragmentation). The target ion's $S_k$ position is marked, demonstrating localization in the universal coordinate system.
    (\textbf{C}) Information compression demonstrating the sufficiency theorem: raw spectra (megabytes) compress to S-entropy coordinates (24 bytes) with $\sim 10^5$-fold compression while preserving all identification-relevant information.
    (\textbf{D}) Molecular similarity metric showing correlation between Euclidean distance in S-space and chemical similarity. Close S-space neighbors are chemically related, validating the metric structure $d(A,B) = \sqrt{\Sigma(S_i^A - S_i^B)^2}$.}
    \label{fig:s_entropy_coordinates}
\end{figure*}

\begin{figure*}[!htbp]
    \centering
    \includegraphics[width=\textwidth]{figures/panel_8_hardware_implications.png}
    \caption{\textbf{Hardware Implications: Partition-Capable Detector Design.}
    (\textbf{A}) Quadrant detector geometry for simultaneous $(n, l, m, s)$ measurement. Four sensor quadrants (colored regions) positioned at $90°$ intervals enable single-pass determination of all partition coordinates. The central ion beam (gold) executes spiral trajectory through the detection volume.
    (\textbf{B}) Frequency resolution showing distinct peaks for different principal numbers $n = 3, 4, 5, 6$. The frequency separation enables spectral discrimination of partition states. Target ion frequency at 1 MHz (gold dashed line) corresponds to $n = 5$.
    (\textbf{C}) Resolution scaling showing frequency resolution $\Delta\omega / 2\pi$ (blue) improving with integration time $T$, while states resolved (green) increases. The trade-off determines optimal measurement duration for partition state discrimination.
    (\textbf{D}) Platform comparison showing mass resolution ($\log_{10} R$) across TOF, Orbitrap, FT-ICR, and Quadrupole analyzers. Checkmarks indicate platforms with sufficient resolution for partition coordinate measurement ($R > 10^5$).}
    \label{fig:hardware_implications}
\end{figure*}

\begin{figure*}[!htbp]
    \centering
    \includegraphics[width=\textwidth]{figures/panel_9_experimental_validation.png}
    \caption{\textbf{Experimental Validation: Cross-Platform Agreement.}
    (\textbf{A}) Mass accuracy across four analyzer platforms (TOF, Orbitrap, FT-ICR, Quadrupole) for the target ion $m/z = 299.0555$. All measurements fall within $\pm 2$ ppm (shaded region), confirming that different physical mechanisms yield identical mass through partition coordinate equivalence.
    (\textbf{B}) Partition coordinate consistency showing repeated $(n, l)$ measurements clustering tightly around true values $(5, 4)$. The $1\sigma$ confidence ellipse (red dashed) confirms reproducibility. Observer-invariant categorical state assignment achieves $R^2 = 1.000$.
    (\textbf{C}) Backaction suppression showing measured momentum disturbance $\Delta p / p$ (blue circles) three orders of magnitude below classical limit (red dashed). The categorical measurement framework achieves $\sim 10^{-6}$ backaction versus $10^{-3}$ classical prediction.
    (\textbf{D}) Observer invariance demonstrating that independent observers measure identical partition states. Perfect correlation along the diagonal confirms that categorical states are objective properties, not observer-dependent artifacts.}
    \label{fig:experimental_validation}
\end{figure*}

\begin{figure*}[!htbp]
    \centering
    \includegraphics[width=\textwidth]{figures/panel_10_bijective_validation.png}
    \caption{\textbf{Bijective Validation: Circular Consistency Without External Ground Truth.}
    (\textbf{A}) Bijective transformation from ion properties $(m/z, I, \text{RT})$ through S-entropy coordinates $(S_k, S_t, S_e)$ to thermodynamic droplet parameters $(v, r, \sigma, T)$. The target ion at $m/z = 299.0555$ (gold star) demonstrates smooth, invertible mapping that preserves information content.
    (\textbf{B}) Physics validation via dimensionless numbers: Weber (We), Reynolds (Re), Capillary (Ca), and Bond (Bo). Thresholds for droplet breakup (We $> 12$) and turbulent flow (Re $> 1000$) are marked. Ions producing physically invalid numbers are flagged as spurious signals.
    (\textbf{C}) Fragment-precursor subset constraint $I(\text{fragment}) \subset I(\text{precursor})$ demonstrating that valid DDA linkages (green circles) have fragment information below precursor. Invalid assignments (red crosses) violate the subset rule and indicate chimeric spectra or misassignment.
    (\textbf{D}) Circular validation cycle: Ion $\to$ S-Entropy $\to$ Droplet $\to$ Physics Check $\to$ Validated Ion. The closed loop requires no external spectral library or authentic standard. Internal consistency through all transformation stages confirms correct partition coordinate assignment.}
    \label{fig:bijective_validation}
\end{figure*}
