% Figure Captions for Complete Trajectory of a Charged Ion in Bounded Phase Space
% ================================================================================

\begin{figure*}[!htbp]
    \centering
    \includegraphics[width=\textwidth]{figures/ion_journey.png}
    \caption{\textbf{Complete Ion Trajectory Through Mass Spectrometer.}
    Overview visualization showing the full path of a single ion ($m/z = 299.0555$) from electrospray ionization through detection. The trajectory passes through seven distinct instrumental regions: ionization source, ion optics, chromatographic interface, mass analyzer, fragmentation cell, detector, and data acquisition. Color gradient indicates temporal progression (purple: early; yellow: late). At each stage, the ion occupies partition coordinates $(n, \ell, m, s)$ that determine its response to electromagnetic fields. The bounded geometry of each region partitions the accessible phase space into $C(n) = 2n^2$ states per energy level. Trajectory completion generates cumulative entropy $S = k_B M \ln 2$ where $M$ counts partition crossings, producing irreversible measurement.}
    \label{fig:ion_journey}
\end{figure*}

\begin{figure*}[!htbp]
    \centering
    \includegraphics[width=\textwidth]{figures/panel_1_ionization.png}
    \caption{\textbf{Ionization and Partition Coordinate Assignment.}
    (\textbf{A}) Three-dimensional visualization of the ion's partition state $(n, \ell, m, s) = (5, 4, -4, -0.5)$ showing occupation in quantum number space. The radial coordinate encodes principal quantum number $n$, angular position encodes $\ell$ and $m$, and color indicates spin state $s$. This representation demonstrates that ionization assigns discrete coordinates, not continuous variables.
    (\textbf{B}) Partition capacity $C(n) = 2n^2$ showing quadratic scaling with principal quantum number. Blue bars indicate discrete state counts: $C(1) = 2$, $C(2) = 8$, $C(3) = 18$, through $C(10) = 200$. Red dashed line confirms theoretical prediction. This capacity function governs maximum occupancy at each energy level.
    (\textbf{C}) Energy level structure with angular momentum splitting. Horizontal lines show energy eigenvalues $E_n$, with $\ell$-dependent fine structure visible as vertical spread within each manifold. The splitting arises from spin-orbit coupling and relativistic corrections.
    (\textbf{D}) Ionization onset showing signal intensity rise as ions enter the gas phase from electrospray droplets. The sigmoidal curve reflects the kinetics of desolvation and ion release. Threshold behavior marks the transition from solvated molecule to gas-phase ion.}
    \label{fig:ionization}
\end{figure*}

\begin{figure*}[!htbp]
    \centering
    \includegraphics[width=\textwidth]{figures/panel_2_chromatography.png}
    \caption{\textbf{Chromatographic Trajectory as Partition Traversal.}
    (\textbf{A}) Three-dimensional trajectory through (retention time, $m/z$, intensity) coordinate space. The ion traverses the chromatographic column, partitioning between mobile and stationary phases. Each partition crossing contributes to retention according to $t_R = t_0(1 + k')$ where $k'$ is the capacity factor. Color gradient indicates temporal progression through the separation.
    (\textbf{B}) Extracted ion chromatogram (EIC) for $m/z = 299.0555$ showing the characteristic Gaussian peak shape arising from partition lag statistics. Peak width reflects the distribution of partition crossing times. Area under the curve represents total ion current integrated over the elution window.
    (\textbf{C}) Cumulative partition crossings $M(t)$ showing linear accumulation rate during chromatographic transport. The constant slope confirms steady-state partition traversal. Each crossing generates entropy increment $\Delta S = k_B \ln 2$, contributing to the irreversibility of the separation process.
    (\textbf{D}) Framework equivalence demonstration: Van Deemter curve (classical kinetic theory, blue) and partition lag description (green) yield identical plate height $H$ as function of linear velocity $u$. The equation $H = A + B/u + Cu$ emerges from both frameworks, confirming that partition coordinates unify classical and statistical descriptions of chromatographic band broadening.}
    \label{fig:chromatography}
\end{figure*}

\begin{figure*}[!htbp]
    \centering
    \includegraphics[width=\textwidth]{figures/panel_3_mass_analysis.png}
    \caption{\textbf{Mass Analysis Across Four Analyzer Platforms.}
    (\textbf{A}) Three-dimensional ion trajectory in Orbitrap geometry showing characteristic axial oscillation with radial drift. The ion executes harmonic motion along the central electrode axis with frequency $\omega_z = \sqrt{qk/m}$ determined by field curvature $k$ and mass-to-charge ratio. Trajectory confinement demonstrates bounded phase space.
    (\textbf{B}) Time-of-flight spectrum showing flight time distribution centered at $t = 12.449~\mu$s corresponding to $m/z = 299.0555$. Peak width reflects initial kinetic energy spread and turn-around time in the reflectron. The relation $t = L\sqrt{m/(2qV)}$ converts flight time to mass.
    (\textbf{C}) Platform comparison showing mass measurement accuracy across TOF, Orbitrap, FT-ICR, and Quadrupole analyzers. Error bars indicate $\pm 2$~ppm precision. All four platforms, despite measuring different physical quantities (flight time, axial frequency, cyclotron frequency, RF stability), converge to identical $m/z$ within instrumental precision.
    (\textbf{D}) Frequency domain representation showing distinct oscillation frequencies for Orbitrap ($\omega_z$) and FT-ICR ($\omega_c$) that map to identical mass through their respective calibration relations. The frequency ratio encodes field parameters, not mass differences, confirming measurement equivalence.}
    \label{fig:mass_analysis}
\end{figure*}

\begin{figure*}[!htbp]
    \centering
    \includegraphics[width=\textwidth]{figures/panel_4_fragmentation.png}
    \caption{\textbf{Fragmentation as Partition Operations.}
    (\textbf{A}) Three-dimensional fragmentation tree showing precursor ion at $m/z = 299.0555$ and product ions from characteristic neutral losses. Vertical axis indicates $m/z$, horizontal axes show fragmentation coordinate space. Major pathways include loss of H$_2$O (18~Da, $\rightarrow$ 281), CO (28~Da, $\rightarrow$ 271), and CO$_2$ (44~Da, $\rightarrow$ 255). Tree structure reflects molecular connectivity.
    (\textbf{B}) Simulated MS2 spectrum showing relative intensities of fragment ions. Peak heights reflect transition probabilities governed by bond dissociation energies and selection rules. The fragment pattern provides structural fingerprint for compound identification.
    (\textbf{C}) Selection rule visualization: allowed transitions ($\Delta\ell = \pm 1$) shown in green, forbidden transitions ($|\Delta\ell| \geq 2$) shown in red. Dipole selection rules constrain which partition coordinate changes are permitted during electronic transitions accompanying fragmentation. The rules emerge from angular momentum conservation.
    (\textbf{D}) Bond dissociation energetics showing linear relationship between neutral loss mass and activation energy threshold. Lower activation energies produce more abundant fragments. The correlation enables prediction of fragmentation patterns from molecular structure without quantum mechanical calculation.}
    \label{fig:fragmentation}
\end{figure*}

\begin{figure*}[!htbp]
    \centering
    \includegraphics[width=\textwidth]{figures/panel_5_detection.png}
    \caption{\textbf{Detection and Trajectory Completion.}
    (\textbf{A}) Three-dimensional detector response surface showing spatial localization of ion impact. The surface represents signal amplitude as function of $(x, y)$ position on the detector face. Impact localization marks trajectory completion---the ion's phase space distribution collapses to a definite position upon measurement.
    (\textbf{B}) Cumulative entropy growth $S(M) = M k_B \ln 2$ as function of partition crossing count $M$. The linear relationship confirms extensive entropy scaling. Each partition crossing contributes exactly $k_B \ln 2$ (one bit) of irreversible entropy, the Landauer minimum for measurement. For typical trajectories with $M \sim 10^6$ crossings, total entropy exceeds $10^5 k_B$.
    (\textbf{C}) Reversal probability $P_{\text{reverse}}(M) \sim e^{-M}$ on semi-logarithmic scale showing exponential suppression with partition count. The probability of spontaneous trajectory reversal becomes negligible after $M \approx 25$ crossings (gray dashed line at $10^{-10}$). This exponential suppression explains macroscopic irreversibility without invoking thermodynamic assumptions.
    (\textbf{D}) Counting statistics histogram showing Poisson distribution of detected ion counts. Blue bars indicate observed frequencies; red circles show theoretical Poisson prediction $P(k) = \lambda^k e^{-\lambda}/k!$ with mean $\lambda$. Agreement confirms discrete detection events and validates shot noise as fundamental precision limit.}
    \label{fig:detection}
\end{figure*}

\begin{figure*}[!htbp]
    \centering
    \includegraphics[width=\textwidth]{figures/panel_6_transport.png}
    \caption{\textbf{Transport Coefficients from Partition Lag.}
    (\textbf{A}) Three-dimensional partition lag surface $\tau_p(T, \rho)$ showing mean time between partition crossings as function of temperature $T$ and density $\rho$. The surface demonstrates that partition lag decreases with temperature (faster crossings at higher energy) and increases with density (more collisions slow traversal). Partition lag is the fundamental quantity from which transport coefficients derive.
    (\textbf{B}) Viscosity comparison: classical kinetic theory prediction (blue) and partition lag calculation (green) as function of temperature. Both frameworks yield identical scaling $\mu \propto \sqrt{T}$ for dilute gases. The viscosity emerges from momentum transfer during partition traversal, $\mu = N^{-1} \sum_{ij} \tau_{p,ij} g_{ij}$ where $g_{ij}$ is the momentum transfer tensor.
    (\textbf{C}) Diffusivity following Stokes-Einstein relation $D = k_B T/(6\pi\mu r)$. The diffusion coefficient (blue) increases with temperature and decreases with particle radius. Partition description recovers the classical result through position displacement during traversal, $D = N^{-1} \sum_{ij} \tau_{p,ij} \Delta x_{ij}^2$.
    (\textbf{D}) Phase transition behavior showing discontinuous vanishing of partition lag $\tau_p$ at critical temperature $T_c$. Above $T_c$: normal transport with finite $\tau_p$ (resistive regime). At $T_c$: abrupt transition. Below $T_c$: $\tau_p = 0$ exactly (superconducting regime). The discontinuity reflects topological change in partition structure, producing exactly zero resistance rather than asymptotically small resistance.}
    \label{fig:transport}
\end{figure*}

\begin{figure*}[!htbp]
    \centering
    \includegraphics[width=\textwidth]{figures/panel_7_validation.png}
    \caption{\textbf{Bijective Validation Without External Ground Truth.}
    (\textbf{A}) Ion-to-S-Entropy transformation showing bijective mapping from ion properties $(m/z, I, \text{RT})$ through S-Entropy coordinates $(S_{\text{knowledge}}, S_{\text{time}}, S_{\text{entropy}})$ to thermodynamic droplet parameters $(v, r, \sigma, T)$. Color gradient (plasma colormap) indicates relative intensity. Star marker identifies target ion at $m/z = 299.0555$. The transformation is invertible: original ion properties can be recovered from droplet parameters.
    (\textbf{B}) Physics validation via dimensionless numbers. Weber number $\text{We} = \rho v^2 d/\sigma$ (inertial/surface tension, threshold $< 12$), Reynolds number $\text{Re} = \rho v d/\mu$ (inertial/viscous, threshold $< 1000$), Capillary number $\text{Ca} = \mu v/\sigma$ (viscous/surface tension, scaled $\times 100$), and Bond number $\text{Bo} = \rho g d^2/\sigma$ (gravitational/surface tension). Ions producing physically invalid dimensionless numbers are flagged as spurious signals.
    (\textbf{C}) Fragment-precursor subset validation demonstrating information conservation $I(\text{fragment}) \subset I(\text{precursor})$. Green circles indicate valid DDA linkages where fragment information content falls below precursor (below diagonal). Red crosses mark invalid assignments violating the subset constraint (above diagonal). Fragmentation cannot create information---fragments must be proper subsets of their parent ions.
    (\textbf{D}) Circular validation cycle schematic: Ion Properties $\rightarrow$ S-Entropy Coordinates $\rightarrow$ Droplet Parameters $\rightarrow$ Physics Validation $\rightarrow$ Ion Properties (validated). The closed loop requires no external spectral library or authentic standard. Internal consistency through all transformation stages confirms correct partition coordinate assignment. Violations localize reconstruction errors.}
    \label{fig:validation}
\end{figure*}
