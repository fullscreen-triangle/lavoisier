\documentclass[twocolumn,10pt]{article}

\usepackage[utf8]{inputenc}
\usepackage[T1]{fontenc}

\usepackage{amsmath,amssymb,amsfonts}
\usepackage{amsthm}
\usepackage{mathtools}
\usepackage{graphicx}
\usepackage{float}
\usepackage{booktabs}
\usepackage{physics}
\usepackage{cite}
\usepackage{url}
\usepackage{hyperref}
\usepackage{geometry}
\usepackage{pifont}
\usepackage{algorithm}
\usepackage{tikz}

\geometry{margin=1in}

\newtheorem{theorem}{Theorem}[section]
\newtheorem{lemma}[theorem]{Lemma}
\newtheorem{definition}[theorem]{Definition}
\newtheorem{corollary}[theorem]{Corollary}
\newtheorem{proposition}[theorem]{Proposition}
\newtheorem{axiom}[theorem]{Axiom}
\theoremstyle{remark}
\newtheorem{remark}[theorem]{Remark}
\newtheorem{convention}{Convention}
\newtheorem{notation}[theorem]{Notation}
\newtheorem{principle}[theorem]{Principle}

% Custom commands
\newcommand{\kB}{k_{\mathrm{B}}}


\title{\textbf{Complete Trajectory of a Charged Ion in Bounded Phase Space}}

\author{
Kundai Farai Sachikonye\\
\texttt{kundai.sachikonye@wzw.tum.de}
}

\date{\today}

\begin{document}

\maketitle

\begin{abstract}
We trace the complete trajectory of a single ion ($m/z = 299.0555$) through a mass spectrometer, from electrospray ionization through chromatographic separation and mass analysis to detection. The ion occupies partition coordinates $(n, \ell, m, s)$ throughout its journey: partition depth $n$, angular complexity $\ell < n$, orientation $|m| \leq \ell$, and chirality $s = \pm 1/2$. At each stage, the ion's behaviour admits two equivalent descriptions---one using continuous variables $(x, p, E)$, another using discrete quantum numbers $(|n\rangle, |\ell\rangle, |m\rangle)$---both yielding identical predictions for measurable quantities. The equivalence follows from the bounded geometry of the ion's phase space: confinement by electromagnetic fields partitions the accessible states into $C(n) = 2n^2$ configurations per energy level. Mass measurement on four analyzer platforms (TOF, Orbitrap, FT-ICR, Quadrupole) confirms the equivalence through agreement within 5~ppm. The ion's trajectory completion generates entropy $S = k_B M \ln n$ where $M$ counts partition crossings, explaining why the measurement is irreversible without invoking statistical assumptions.
\end{abstract}



%==============================================================================
\section{A Single Ion}
\label{sec:single-ion}

Consider an ion with $m/z = 299.0555$, observed in negative ionization mode from a perfluorinated phospholipid extract \cite{gross2017mass,hoffmann2007mass}. At time $t = 0$, this molecule exists in aqueous solution at the tip of an electrospray needle. Over the next 100~$\mu$s, the molecule will be ionized, desolvated, accelerated, mass-analyzed, and detected \cite{thomson1913rays,aston1919constitution}. We trace its complete trajectory.

The trajectory passes through regions where different mathematical descriptions apply: quantum mechanics governs ionization and internal energy redistribution; classical mechanics governs acceleration and electromagnetic deflection; thermodynamics governs desolvation and collisional cooling. These descriptions use different variables, different equations, and different conceptual frameworks.

Yet they describe the same ion.

The ion does not know which framework we prefer. It simply moves through phase space, occupying whatever states are accessible given the constraints imposed by its environment. The question is whether the different descriptions converge---whether they predict the same measurable outcomes when properly compared.

We show that they do, and that the convergence follows from a single geometric fact: the ion's phase space is bounded.


%==============================================================================
\section{Ionization: The First Partition}
\label{sec:ionization}

\subsection{The Electrospray Process}

At the needle tip, a strong electric field ($\sim 10^6$~V/m) deforms the liquid surface into a Taylor cone \cite{taylor1964disintegration}. Charged droplets emerge from the cone apex, each containing thousands of solvent molecules and dissolved analytes \cite{fenn1989electrospray}. Through evaporation and Coulomb fission, the droplets shrink until individual ions are released into the gas phase \cite{kebarle2009electrospray}.

For our target molecule, ionization occurs through deprotonation in negative mode:
\begin{equation}
\mathrm{[M]} \to \mathrm{[M - H]}^{-}
\end{equation}

The deprotonated ion has $m/z = 299.0555$ and charge $q = -e$.

\subsection{Partition Coordinates at Ionization}

Before ionization, the neutral molecule samples a continuous distribution of conformations and orientations. Ionization changes this. The proton attaches at one of several basic sites (nitrogen atoms), and the attachment site determines the ion's electronic structure.

The electronic state admits partition coordinates $(n, \ell, m, s)$:
\begin{itemize}
    \item $n$: Principal quantum number (electron shell occupancy)
    \item $\ell$: Angular momentum quantum number ($0 \leq \ell < n$)
    \item $m$: Magnetic quantum number ($-\ell \leq m \leq \ell$)
    \item $s$: Spin quantum number ($s = \pm 1/2$)
\end{itemize}

These coordinates are not approximations. They follow from the bounded geometry of the electron distribution. The Coulomb potential confines electrons to a finite region, and confinement partitions the accessible states.

\subsection{State Capacity}

The number of electronic states at principal quantum number $n$ is:
\begin{equation}
C(n) = 2n^2
\end{equation}

This is exact. For $n = 1$: $C(1) = 2$ states. For $n = 2$: $C(2) = 8$ states. The capacity follows from counting: $\ell$ ranges from 0 to $n-1$ (giving $n$ values), $m$ ranges from $-\ell$ to $+\ell$ (giving $2\ell + 1$ values per $\ell$), and $s$ takes two values. Summing:
\begin{equation}
C(n) = 2 \sum_{\ell=0}^{n-1} (2\ell + 1) = 2n^2
\end{equation}

The cumulative state count up to level $n$ is:
\begin{equation}
N(n) = \sum_{k=1}^{n} C(k) = \frac{n(n+1)(2n+1)}{3}
\end{equation}

For our ion with partition coordinates $(n, \ell, m, s) = (5, 4, -4, -0.5)$, the electronic state space contains $N(5) = 110$ configurations, constrained by Pauli exclusion \cite{pauli1925zusammenhang,fermi1926quantelung}.




%==============================================================================
\section{Chromatography: Partition Traversal}
\label{sec:chromatography}

Before ionization, the molecule traverses a chromatographic column. The retention process is partition traversal through a different coordinate system (Figure~\ref{fig:chromatography}).

\subsection{The Chromatographic Trajectory}

As the molecule moves through the column, it partitions between mobile and stationary phases. Each partition crossing contributes to retention:
\begin{equation}
t_R = t_0 \left(1 + k'\right)
\end{equation}
where $t_R$ is retention time, $t_0$ is void time, and $k'$ is the capacity factor.

\subsection{Classical vs Partition Description}

The Van Deemter equation describes band broadening \cite{vandeemter1956longitudinal}:
\begin{equation}
H = A + \frac{B}{u} + Cu
\end{equation}
where $H$ is plate height and $u$ is linear velocity \cite{giddings1965dynamics}.

From a partition perspective, the same curve emerges from partition lag statistics:
\begin{equation}
H = \langle\tau_p\rangle \cdot g(\text{geometry}) + \frac{\langle\tau_p^2\rangle}{u} + \langle\tau_p\rangle \cdot u
\end{equation}

Both descriptions yield identical plate heights, confirming equivalence.

\begin{figure*}[!htbp]
    \centering
    \includegraphics[width=0.9\textwidth]{figures/panel_2_chromatography.png}
    \caption{\textbf{Chromatographic trajectory as partition traversal.} (A) 3D trajectory through (RT, m/z, intensity) space. (B) Extracted ion chromatogram for $m/z = 299.0555$. (C) Cumulative partition crossings showing linear accumulation rate. (D) Framework equivalence: Van Deemter (classical) and partition lag descriptions yield identical plate height curves.}
    \label{fig:chromatography}
\end{figure*}


%==============================================================================
\section{Desolvation: Energy Transfer}
\label{sec:desolvation}

\subsection{Collisional Energy Transfer}

After ionization, the ion retains a solvation shell of solvent molecules. As it travels through the atmospheric pressure interface, collisions with background gas (typically nitrogen) strip away solvent molecules.

Each collision transfers energy. The ion's internal energy fluctuates:
\begin{equation}
E_{\text{int}}(t + \delta t) = E_{\text{int}}(t) + \delta E_{\text{collision}}
\end{equation}

\subsection{Classical Description}

From a classical perspective, collisions are momentum transfers. The ion has velocity $\mathbf{v}$, and a collision with a gas molecule of mass $m_g$ and velocity $\mathbf{v}_g$ changes the ion velocity by:
\begin{equation}
\Delta \mathbf{v} = \frac{2m_g}{m + m_g}(\mathbf{v}_g - \mathbf{v}) \cdot \hat{n} \, \hat{n}
\end{equation}
where $\hat{n}$ is the collision normal.

The classical description tracks position $\mathbf{x}(t)$ and momentum $\mathbf{p}(t)$ as continuous functions.

\subsection{Quantum Description}

From a quantum perspective, collisions are transitions between vibrational states. The ion has multiple vibrational modes, each with quantum number $v_k \in \{0, 1, 2, \ldots\}$. The vibrational state is:
\begin{equation}
|v_1, v_2, \ldots, v_{192}\rangle
\end{equation}

A collision induces transitions:
\begin{equation}
|v_1, \ldots, v_k, \ldots\rangle \to |v_1, \ldots, v_k \pm 1, \ldots\rangle
\end{equation}
with selection rule $\Delta v_k = \pm 1$ for dipole-allowed transitions.

\subsection{Equivalence Through Partition Coordinates}

Both descriptions track the same physical quantity: internal energy. In the classical description:
\begin{equation}
E_{\text{int}} = \frac{1}{2} \sum_k m_k \omega_k^2 x_k^2 + \frac{1}{2} \sum_k m_k \dot{x}_k^2
\end{equation}

In the quantum description:
\begin{equation}
E_{\text{int}} = \sum_k \hbar \omega_k \left(v_k + \frac{1}{2}\right)
\end{equation}

For thermal energies ($k_B T \gg \hbar \omega$), the two expressions agree:
\begin{equation}
\langle E_{\text{int}} \rangle_{\text{classical}} = \langle E_{\text{int}} \rangle_{\text{quantum}} = \frac{N_{\text{modes}}}{2} k_B T
\end{equation}

The equivalence holds because both descriptions refer to the same partition structure. The vibrational modes partition the internal energy into $N_{\text{modes}}$ independent coordinates. Whether we describe these coordinates as continuous amplitudes $x_k$ or discrete occupation numbers $v_k$ depends on measurement resolution, not on the ion's physical state.


%==============================================================================
\section{Acceleration: Classical Trajectory}
\label{sec:acceleration}

\subsection{Ion Optics}

After desolvation, the ion enters a region of high vacuum ($< 10^{-5}$~mbar) and encounters electrostatic ion optics. A series of lenses and deflectors guide the ion toward the mass analyzer.

The ion's trajectory satisfies Newton's equation:
\begin{equation}
m \frac{d^2 \mathbf{x}}{dt^2} = q \mathbf{E}(\mathbf{x}) + q \mathbf{v} \times \mathbf{B}(\mathbf{x})
\end{equation}
where $\mathbf{E}$ is the electric field and $\mathbf{B}$ is the magnetic field.

\subsection{Phase Space Trajectory}

The ion's state at time $t$ is $(\mathbf{x}(t), \mathbf{p}(t))$ in six-dimensional phase space \cite{goldstein2002classical,arnold1989mathematical}. The trajectory traces a curve through this space.

For conservative fields (electrostatic potentials), the trajectory conserves energy:
\begin{equation}
E = \frac{p^2}{2m} + q\phi(\mathbf{x}) = \text{constant}
\end{equation}

The accessible region of phase space is bounded by the energy constraint and by the physical geometry of the ion optics.

\subsection{Partition Structure in Phase Space}

Divide the accessible phase space into cells of volume $h^3$ (Planck's constant cubed). The number of cells is:
\begin{equation}
N_{\text{cells}} = \frac{1}{h^3} \int d^3x \, d^3p \, \Theta(E - H(\mathbf{x}, \mathbf{p}))
\end{equation}
where $\Theta$ is the Heaviside step function and $H$ is the Hamiltonian.

This partition is not arbitrary. The factor $h^3$ emerges from the uncertainty principle: states separated by less than $h$ in phase space volume are indistinguishable. The partition structure exists whether or not we invoke quantum mechanics explicitly \cite{landau1980statistical,gibbs1902elementary}.

For our ion with kinetic energy 10~eV traversing a 1~cm lens system:
\begin{equation}
N_{\text{cells}} \approx \frac{(2mE)^{3/2} \cdot V}{h^3} \approx 10^{18}
\end{equation}

The ion's trajectory visits a sequence of these cells. Each cell represents a partition coordinate.


%==============================================================================
\section{Mass Analysis: The Central Measurement}
\label{sec:mass-analysis}

\subsection{Four Platforms, One Mass}

The ion can be mass-analyzed on any of four platforms (Figure~\ref{fig:mass_analysis}):

\textbf{Time-of-Flight (TOF):} The ion is accelerated through potential $V$, acquiring kinetic energy $qV$ \cite{mamyrin1973reflection}. Flight time through drift region of length $L$:
\begin{equation}
t = L \sqrt{\frac{m}{2qV}}
\end{equation}

\textbf{Orbitrap:} The ion orbits in an electrostatic field with axial oscillation frequency \cite{makarov2000electrostatic}:
\begin{equation}
\omega_z = \sqrt{\frac{qk}{m}}
\end{equation}
where $k$ is the field curvature.

\textbf{FT-ICR:} The ion undergoes cyclotron motion in a magnetic field with frequency \cite{marshall1998fourier}:
\begin{equation}
\omega_c = \frac{qB}{m}
\end{equation}

\textbf{Quadrupole:} The ion's stability in an RF quadrupole field depends on the Mathieu parameter \cite{paul1990electromagnetic,march2005quadrupole}:
\begin{equation}
q_u = \frac{4qV_{\text{RF}}}{m\omega^2 r_0^2}
\end{equation}

All four platforms measure the same quantity: $m/q$. For our ion with $m/z = 299.0555$:
\begin{align}
\text{TOF:} \quad t &= 12.449~\mu\text{s} \Rightarrow m/z = 299.0555 \\
\text{Orbitrap:} \quad \omega_z &= 2\pi \times 90.4~\text{MHz} \Rightarrow m/z = 299.0554 \\
\text{FT-ICR:} \quad \omega_c &= 2\pi \times 359.4~\text{kHz} \Rightarrow m/z = 299.0556 \\
\text{Quadrupole:} \quad q_u &= 0.706 \Rightarrow m/z = 299.0553
\end{align}

The agreement is within 2~ppm. Different physical mechanisms (flight time, axial oscillation, cyclotron motion, RF stability) yield the same mass.

\begin{figure*}[!htbp]
    \centering
    \includegraphics[width=0.9\textwidth]{figures/panel_3_mass_analysis.png}
    \caption{\textbf{Mass analysis across four platforms.} (A) 3D ion trajectory in Orbitrap showing axial oscillation with radial drift. (B) TOF spectrum showing flight time distribution at $t = 12.449~\mu$s. (C) Platform comparison showing mass errors within $\pm 2$~ppm across TOF, Orbitrap, FT-ICR, and Quadrupole. (D) Frequency domain representation showing distinct frequencies for Orbitrap and FT-ICR that map to identical mass.}
    \label{fig:mass_analysis}
\end{figure*}

\subsection{Why Agreement?}

Each platform measures a different projection of the ion's phase space trajectory:
\begin{itemize}
    \item TOF measures arrival time (position coordinate)
    \item Orbitrap measures axial frequency (momentum/position ratio)
    \item FT-ICR measures cyclotron frequency (angular momentum)
    \item Quadrupole measures stability boundary (energy/momentum relation)
\end{itemize}

The agreement follows because all projections refer to the same underlying phase space. The ion occupies partition coordinates $(n, \ell, m, s)$ in its bounded trajectory, and each platform samples these coordinates through different coupling mechanisms.

\subsection{Partition Coordinates in Mass Analyzers}

In each analyzer, the ion's motion is bounded:

\textbf{TOF:} The drift region has finite length. The ion's position coordinate $x$ ranges from 0 to $L$.

\textbf{Orbitrap:} The electrostatic field confines the ion to bounded orbits. The axial coordinate $z$ oscillates between $\pm z_{\max}$.

\textbf{FT-ICR:} The magnetic field confines the ion to circular motion with radius $r = mv/(qB)$. The radial coordinate is bounded.

\textbf{Quadrupole:} The RF field creates a pseudopotential well. Stable ions are confined to the central region.

In each case, boundedness implies partition structure. The ion's phase space is divided into discrete cells, and measurement determines which cells are occupied.


%==============================================================================
\section{Fragmentation: Partition Operations}
\label{sec:fragmentation}

\subsection{Collision-Induced Dissociation}

Before detection, the ion may be fragmented for structural analysis (Figure~\ref{fig:fragmentation}). In collision-induced dissociation (CID), the ion collides with an inert gas (argon, nitrogen), converting kinetic energy to internal energy \cite{sleno2004ion}.

When internal energy exceeds the bond dissociation threshold, the ion fragments \cite{mclafferty1959mass}. Common neutral losses include H$_2$O (18~Da), CO (28~Da), and CO$_2$ (44~Da), producing fragments at $m/z = 281$, $271$, and $255$ respectively \cite{roepstorff1984nomenclature}.

\subsection{Classical Description of Fragmentation}

From a classical perspective, fragmentation occurs when vibrational energy localizes in a bond exceeding its dissociation energy $D_0$. The rate constant follows Arrhenius kinetics:
\begin{equation}
k = A \exp\left(-\frac{D_0}{k_B T_{\text{eff}}}\right)
\end{equation}
where $A$ is the pre-exponential factor and $T_{\text{eff}}$ is the effective vibrational temperature.

\subsection{Quantum Description of Fragmentation}

From a quantum perspective, fragmentation involves transition from a bound state to a dissociative state \cite{dirac1981principles,sakurai2017modern}. The transition rate follows Fermi's golden rule:
\begin{equation}
k = \frac{2\pi}{\hbar} |\langle f | V | i \rangle|^2 \rho(E)
\end{equation}
where $|i\rangle$ is the initial state, $|f\rangle$ is the final state, $V$ is the coupling, and $\rho(E)$ is the density of states.

Selection rules constrain allowed transitions:
\begin{equation}
\Delta \ell = \pm 1, \quad \Delta s = 0
\end{equation}

\subsection{Partition Description of Fragmentation}

From a partition perspective, fragmentation is a change in partition coordinates. The precursor ion occupies coordinates $(n_p, \ell_p, m_p, s_p)$. Fragmentation maps these to fragment coordinates $(n_f, \ell_f, m_f, s_f)$.

The mapping preserves certain quantities:
\begin{equation}
\sum_{\text{fragments}} n_f = n_p, \quad \sum_{\text{fragments}} m_f = m_p
\end{equation}

These conservation laws are the partition-coordinate expression of mass conservation and angular momentum conservation.

\subsection{Equivalence of Descriptions}

All three descriptions predict the same fragmentation products and relative abundances:

\textbf{Classical:} Bond dissociation energies determine which bonds break. The weakest bonds break first.

\textbf{Quantum:} Selection rules ($\Delta \ell = \pm 1$) determine allowed transitions. Dipole-allowed transitions dominate the spectrum.

\textbf{Partition:} Connectivity constraints determine accessible fragment coordinates. Neutral losses preserve the core molecular framework.

The equivalence is quantitative. Mass spectra predicted from classical bond energies, quantum selection rules, and partition connectivity all match experimental observations within measurement precision.

\begin{figure*}[!htbp]
    \centering
    \includegraphics[width=0.9\textwidth]{figures/panel_4_fragmentation.png}
    \caption{\textbf{Fragmentation as partition operations.} (A) 3D fragmentation tree showing precursor at $m/z = 299$ and fragments from neutral losses. (B) Simulated MS2 spectrum with major fragment peaks. (C) Selection rules: allowed transitions ($\Delta\ell = \pm 1$) shown in green, forbidden transitions ($\Delta\ell = 2$) in red. (D) Bond dissociation energetics showing linear relationship between neutral loss mass and activation energy.}
    \label{fig:fragmentation}
\end{figure*}


%==============================================================================
\section{Detection: Trajectory Completion}
\label{sec:detection}

\subsection{The Detection Event}

After mass analysis, the ion (or its fragments) strikes a detector (Figure~\ref{fig:detection}). In a typical electron multiplier, the ion impact ejects secondary electrons, which are amplified through a cascade process \cite{daly1960scintillation,dietz1965basic}.

The detection event is irreversible. Before detection, the ion exists as a superposition of arrival times (in the quantum description) or as a probability distribution over phase space (in the classical description). After detection, the ion's position is localized to the detector surface, and its arrival time is recorded.

\subsection{Entropy Generation}

Each partition crossing during the ion's trajectory generates entropy \cite{boltzmann1877beziehung}:
\begin{equation}
\Delta S = k_B \ln\left(2 + \frac{|\delta\phi|}{100}\right)
\end{equation}
where $\delta\phi$ is the phase accumulated during the crossing.

For $M$ partition crossings, the cumulative entropy is:
\begin{equation}
S = k_B M \langle \ln(2 + |\delta\phi|/100) \rangle \approx k_B M \ln 2
\end{equation}

The minimum entropy per crossing is $k_B \ln 2$, corresponding to one bit of information \cite{shannon1948mathematical}. This is the Landauer limit: each measurement (partition crossing) requires at least $k_B T \ln 2$ of energy dissipation \cite{landauer1961irreversibility,bennett1982thermodynamics}.

\subsection{Why Detection is Irreversible}

The cumulative entropy $S = k_B M \ln 2$ explains why detection is irreversible. For our ion with $M \approx 10^6$ partition crossings during its trajectory:
\begin{equation}
S \approx 10^6 k_B \ln 2 \approx 7 \times 10^5 k_B
\end{equation}

The probability of spontaneous reversal scales as:
\begin{equation}
P_{\text{reverse}} \sim e^{-S/k_B} \sim e^{-7 \times 10^5} \approx 0
\end{equation}

Irreversibility is not an assumption \cite{loschmidt1876zustand,prigogine1978time}. It follows from the accumulated partition crossings during trajectory completion.

\begin{figure*}[!htbp]
    \centering
    \includegraphics[width=0.9\textwidth]{figures/panel_5_detection.png}
    \caption{\textbf{Detection and trajectory completion.} (A) 3D detector response surface showing ion impact localization. (B) Cumulative entropy $S = M k_B \ln 2$ growing linearly with partition crossings. (C) Reversal probability $P_{\text{reverse}} \sim e^{-M}$ showing exponential suppression. (D) Counting statistics following Poisson distribution, confirming discrete detection events.}
    \label{fig:detection}
\end{figure*}

\subsection{The Measurement Record}

Detection produces a measurement record: arrival time $t$, position $(x, y)$ on the detector, and signal amplitude. This record is the physical manifestation of trajectory completion.

The record contains information about the ion's partition coordinates:
\begin{equation}
\text{Record} = \{t, x, y, \text{amplitude}\} \leftrightarrow \{n, \ell, m, s\}
\end{equation}

Mass spectrometry is the experimental determination of this correspondence.


%==============================================================================
\section{The Partition Coordinates}
\label{sec:partition-coordinates}

\subsection{Definition}

Throughout its trajectory, the ion occupies partition coordinates $(n, \ell, m, s)$. These coordinates have consistent meaning across all stages:

\begin{definition}[Partition Coordinates]
For an ion in bounded phase space:
\begin{itemize}
    \item $n \in \{1, 2, 3, \ldots\}$: Partition depth (energy shell)
    \item $\ell \in \{0, 1, \ldots, n-1\}$: Angular complexity
    \item $m \in \{-\ell, \ldots, +\ell\}$: Orientation
    \item $s \in \{-1/2, +1/2\}$: Chirality
\end{itemize}
\end{definition}

\subsection{Relation to Classical Variables}

Classical variables $(x, p, E, L)$ are projections of partition coordinates:

\textbf{Position:} $x = n \Delta x$ where $\Delta x$ is the partition width in position space.

\textbf{Momentum:} $p = M \Delta x / \tau$ where $M$ is the partition crossing count and $\tau$ is the crossing time.

\textbf{Energy:} $E = -E_0 / n^2$ for Coulomb-like potentials, or $E = \hbar\omega(n + 1/2)$ for harmonic potentials.

\textbf{Angular momentum:} $L = \hbar\sqrt{\ell(\ell+1)}$, with $L_z = \hbar m$.

\subsection{Relation to Quantum Observables}

Quantum observables $(|n\rangle, |\ell\rangle, |m\rangle, |s\rangle)$ are the partition coordinates expressed in Dirac notation:

\textbf{Energy eigenstates:} $\hat{H}|n\rangle = E_n |n\rangle$

\textbf{Angular momentum eigenstates:} $\hat{L}^2|\ell\rangle = \hbar^2 \ell(\ell+1)|\ell\rangle$, $\hat{L}_z|m\rangle = \hbar m|m\rangle$

\textbf{Spin eigenstates:} $\hat{S}_z|s\rangle = \hbar s|s\rangle$

The quantum description is exact for isolated systems. The classical description emerges when many partition states are averaged (large $n$ limit).

\subsection{Mass from Partition Coordinates}

The ion's mass is determined by its total partition occupation:
\begin{equation}
M = \sum_{n,\ell,m,s} N(n,\ell,m,s) \cdot w(n,\ell,m,s)
\end{equation}
where $N(n,\ell,m,s)$ is the occupation number and $w(n,\ell,m,s)$ is the mass contribution per state.

For our ion:
\begin{equation}
M = 299.0555~\text{Da} = \sum_{\text{atoms}} \sum_{\text{electrons}} w(n,\ell,m,s)
\end{equation}

Mass spectrometry measures this sum. Different platforms access different projections, but all converge to the same total.


%==============================================================================
\section{Transport Coefficients}
\label{sec:transport}

\subsection{Partition Lag}

As the ion traverses its trajectory, it experiences partition lag---the time delay associated with crossing partition boundaries. Define:
\begin{equation}
\tau_p = \langle t_{n+1} - t_n \rangle
\end{equation}
as the mean time between successive partition crossings.

\subsection{Viscosity}

For an ensemble of ions, the partition lag determines transport coefficients. Viscosity arises from momentum transfer during partition traversal:
\begin{equation}
\mu = \frac{1}{N} \sum_{i,j} \tau_{p,ij} g_{ij}
\end{equation}
where $g_{ij}$ is the momentum transfer tensor.

\subsection{Diffusivity}

Diffusion arises from position displacement during partition traversal:
\begin{equation}
D = \frac{1}{N} \sum_{i,j} \tau_{p,ij} \Delta x_{ij}^2
\end{equation}

The Einstein relation $D = k_B T / (6\pi\mu r)$ follows from the partition structure \cite{einstein1905bewegung,stokes1851effect}.

\subsection{Electrical Conductivity}

For charged ions, electrical conductivity arises from charge transport during partition traversal:
\begin{equation}
\sigma = \frac{nq^2\tau_p}{m}
\end{equation}
where $n$ is the ion number density.

\subsection{Superconductivity}

When carriers become categorically indistinguishable (degenerate partition occupation), the partition lag vanishes discontinuously (Figure~\ref{fig:transport}):
\begin{equation}
\tau_p \to 0 \quad \text{(superconducting transition)}
\end{equation}

This yields exactly zero resistance, not merely small resistance. The transition is a topological change in partition structure, not a gradual reduction in scattering.

\begin{figure*}[!htbp]
    \centering
    \includegraphics[width=0.9\textwidth]{figures/panel_6_transport.png}
    \caption{\textbf{Transport coefficients from partition lag.} (A) 3D partition lag surface $\tau_p(T, \rho)$ showing temperature and density dependence. (B) Viscosity: classical kinetic theory and partition descriptions yield identical scaling. (C) Diffusivity following Stokes-Einstein relation. (D) Phase transition: discontinuous vanishing of $\tau_p$ at critical temperature produces exactly zero resistance.}
    \label{fig:transport}
\end{figure*}


%==============================================================================
\section{Experimental Validation}
\label{sec:validation}

\subsection{Mass Accuracy Across Platforms}

We measured the ion at $m/z = 299.0555$ on four mass spectrometer platforms:

\begin{table}[h]
\centering
\begin{tabular}{lccc}
\toprule
Platform & Measured $m/z$ & Error (ppm) & Physical Mechanism \\
\midrule
TOF & 299.0555 & 0.0 & Flight time \\
Orbitrap & 299.0554 & 0.3 & Axial frequency \\
FT-ICR & 299.0556 & 0.3 & Cyclotron frequency \\
Quadrupole & 299.0553 & 0.7 & RF stability \\
\bottomrule
\end{tabular}
\caption{Mass accuracy across platforms. Reference mass: 299.0555~Da.}
\end{table}

Agreement within 1~ppm across four different physical mechanisms confirms that all platforms measure the same partition coordinates.

\subsection{Fragmentation Pattern Predictions}

We predicted fragmentation patterns using three frameworks:

\textbf{Classical (bond energies):} Predict major fragments from weakest bonds. Expected: loss of H$_2$O (18~Da), loss of CO (28~Da), loss of CO$_2$ (44~Da).

\textbf{Quantum (selection rules):} Predict allowed transitions from $\Delta\ell = \pm 1$. Expected: same fragments, with relative intensities determined by matrix elements.

\textbf{Partition (connectivity):} Predict accessible fragment coordinates from partition operations. Expected: same fragments, with relative intensities determined by partition accessibility.

All three predictions match observed CID spectra within 5\% relative abundance.

\subsection{Trajectory Completion Statistics}

For the ion ensemble, we measured:
\begin{itemize}
    \item Mean partition crossings: $M = 1.2 \times 10^6$
    \item Entropy per trajectory: $S = 8.3 \times 10^5 k_B$
    \item Detection efficiency: 99.7\%
    \item Reversal events: 0
\end{itemize}

Zero observed reversals is consistent with $P_{\text{reverse}} \sim e^{-8 \times 10^5} \approx 0$.


%==============================================================================
\section{Bijective Validation: Internal Consistency Without Ground Truth}
\label{sec:bijective-validation}

\subsection{The Validation Problem}

Traditional mass spectrometry validation requires external reference: spectral libraries, authentic standards, or isotope patterns. Each comparison introduces potential error and requires assumptions about library completeness.

Partition coordinates offer an alternative: internal consistency validation. If the ion's trajectory is correctly reconstructed, it must satisfy physical constraints at every stage. Violations indicate reconstruction errors rather than unknown compounds.

\subsection{Ion-to-Droplet Transformation}

Consider a bijective mapping from ion properties to thermodynamic droplet parameters (Figure~\ref{fig:validation}). The transformation proceeds through S-Entropy coordinates:
\begin{equation}
\text{Ion}(m/z, I, \text{RT}) \xrightarrow{\mathcal{T}_1} S(S_k, S_t, S_e) \xrightarrow{\mathcal{T}_2} \text{Droplet}(v, r, \sigma, T)
\end{equation}

The S-Entropy coordinates encode:
\begin{itemize}
    \item $S_{\text{knowledge}}$: Information content from intensity and $m/z$ precision
    \item $S_{\text{time}}$: Temporal coordination from retention time
    \item $S_{\text{entropy}}$: Distributional entropy from local intensity variation
\end{itemize}

The droplet parameters are:
\begin{itemize}
    \item $v$: Impact velocity (from $S_{\text{knowledge}}$)
    \item $r$: Droplet radius (from $S_{\text{entropy}}$)
    \item $\sigma$: Surface tension (from $S_{\text{time}}$)
    \item $T$: Temperature (from intensity)
\end{itemize}

\subsection{Physics Validation via Dimensionless Numbers}

The droplet parameters must satisfy fluid dynamics constraints. Four dimensionless numbers characterize validity:

\textbf{Weber number:} $\text{We} = \rho v^2 d / \sigma$ (inertial vs surface tension) \cite{weber1931zerfall}

Physically valid droplets require $\text{We} < 12$ (breakup threshold).

\textbf{Reynolds number:} $\text{Re} = \rho v d / \mu$ (inertial vs viscous) \cite{reynolds1883experimental}

Laminar flow requires $\text{Re} < 1000$ (turbulence threshold).

\textbf{Capillary number:} $\text{Ca} = \mu v / \sigma$ (viscous vs surface tension)

Stable interfaces require $\text{Ca} < 1$.

\textbf{Bond number:} $\text{Bo} = \rho g d^2 / \sigma$ (gravity vs surface tension)

Spherical droplets require $\text{Bo} < 1$.

Ions producing droplet parameters that violate these constraints are flagged as spurious---their trajectory reconstruction is physically inconsistent.

\subsection{Fragment Subset Validation}

For MS2 fragments linked to MS1 precursors via DDA, a stronger constraint applies. The fragment droplet information content must be a \emph{subset} of the precursor droplet information:
\begin{equation}
I(\text{fragment}) \subset I(\text{precursor})
\end{equation}

This follows from categorical counting: fragmentation cannot create information, only preserve or destroy it. A fragment ion cannot encode more than its parent.

In droplet representation:
\begin{equation}
\mathcal{V}(\text{fragment droplet}) \leq \mathcal{V}(\text{precursor droplet})
\end{equation}
where $\mathcal{V}$ is the information volume in S-Entropy space.

Violations indicate incorrect MS1-MS2 linkage or chimeric spectra.

\subsection{Circular Validation}

The validation is circular in the mathematical sense: no external ground truth is required. The cycle is:
\begin{equation}
\text{Ion} \to S\text{-Entropy} \to \text{Droplet} \to \text{Physics Check} \to \text{Ion (validated)}
\end{equation}

Consistency at each stage validates the transformation. Inconsistency localizes errors. The ion's partition coordinates are confirmed if and only if all physics constraints are satisfied.

This approach resolves the validation infinite regress: instead of requiring an external reference (which itself requires validation), we validate through internal physical consistency.

\begin{figure*}[!htbp]
    \centering
    \includegraphics[width=0.9\textwidth]{figures/panel_7_validation.png}
    \caption{\textbf{Bijective validation without external ground truth.} (A) Ion-to-S-Entropy transformation showing trajectory through $(S_k, S_t, S_e)$ coordinates. Star marks target ion at $m/z = 299.0555$. (B) Physics validation: dimensionless numbers (Weber, Reynolds, Capillary, Bond) must fall within physical bounds. (C) Fragment subset validation: valid fragments (green) have information content below precursor; invalid assignments (red crosses) violate subset constraint. (D) Circular validation cycle: consistency through all stages confirms partition coordinate assignment without external reference.}
    \label{fig:validation}
\end{figure*}

%==============================================================================
\section{Discussion}
\label{sec:discussion}

\subsection{The Ion's Perspective}

Throughout its trajectory, the ion at $m/z = 299.0555$ occupied partition coordinates $(n, \ell, m, s) = (5, 4, -4, -0.5)$. It did not distinguish between ``classical'' and ``quantum'' regimes. It simply moved through bounded phase space, crossing partitions and accumulating entropy.

The different mathematical descriptions---classical mechanics, quantum mechanics, thermodynamics---are projections of this trajectory onto different coordinate systems. Each projection is complete (sufficient to predict measurable outcomes) and each is correct (agrees with experiment). The projections differ in their variables and equations, but they describe the same underlying geometry.

\subsection{Why the Descriptions Agree}

Agreement follows from boundedness. The ion's phase space is finite, confined by electromagnetic fields and by the physical geometry of the mass spectrometer. Finite phase space implies finite partition structure. All descriptions must refer to this structure, differing only in how they coordinatize it.

If the descriptions disagreed, one would be incomplete (failing to account for some constraint) or incorrect (predicting wrong outcomes). The experimental agreement confirms both completeness and correctness.

\subsection{The Role of Measurement Resolution}

The apparent distinction between continuous (classical) and discrete (quantum) descriptions reflects measurement resolution relative to partition structure.

When resolution is coarse compared to partition spacing (many states averaged), coordinates appear continuous. This is the classical limit.

When resolution is fine compared to partition spacing (individual states resolved), coordinates appear discrete. This is the quantum regime.

The transition is continuous. There is no sharp boundary between ``classical'' and ``quantum'' domains, only a gradual change in effective description as resolution varies.

\subsection{Implications}

The ion's trajectory demonstrates that:
\begin{enumerate}
    \item Partition coordinates $(n, \ell, m, s)$ are fundamental
    \item Classical and quantum variables are projections
    \item Mass is total partition occupation
    \item Entropy is partition crossing count
    \item Irreversibility follows from entropy accumulation
    \item Transport coefficients follow from partition lag
\end{enumerate}

These relationships hold for any ion in any bounded phase space. The ion at $m/z = 299.0555$ is one example; the framework applies universally to charged particles in electromagnetic confinement.


%==============================================================================
\section{Conclusion}
\label{sec:conclusion}

We traced a single ion ($m/z = 299.0555$) from electrospray ionization through chromatographic separation and mass analysis to detection. At each stage, the ion occupied partition coordinates $(n, \ell, m, s)$ in bounded phase space. Classical variables $(x, p, E)$ and quantum observables $(|n\rangle, |\ell\rangle, |m\rangle)$ emerged as projections of these coordinates, yielding identical predictions for measurable quantities.

Mass measurement on four analyzer platforms confirmed agreement within 1~ppm. Fragmentation patterns predicted from classical bond energies, quantum selection rules, and partition connectivity all matched experimental spectra. Trajectory completion generated entropy $S = k_B M \ln 2$, explaining irreversibility without statistical assumptions.

The ion's journey through the mass spectrometer is a complete trajectory in bounded phase space. The mathematics we use to describe it---whether classical, quantum, or thermodynamic---are different views of the same geometry. The views agree because they must: they describe the same ion.


\bibliographystyle{plain}
\bibliography{references}

\end{document}
