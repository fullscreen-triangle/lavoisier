\begin{figure*}[htbp]
    \centering
    \includegraphics[width=\textwidth]{figures/categorical_partition_panel.png}
    \caption{\textbf{Categorical structure and partition geometry.} 
    Continuous observables discretize into categorical states via finite observer resolution, generating quantum numbers $(n, l, m, s)$ with $2n^2$ shell capacity.
    %
    \textbf{(Row 1, Left)} Continuous $\to$ categorical: oscillating signal (blue/yellow) discretizes into finite observer bins. Finite resolution transforms continuous variable into categorical states.
    %
    \textbf{(Row 1, Center-Left)} Completion order (Hasse diagram): directed acyclic graph shows hierarchical ordering of 8 categorical states. Arrows indicate completion dependencies, forming partially ordered set (poset).
    %
    \textbf{(Row 1, Center-Right)} Temporal emergence: sigmoid curve shows categories completed over time, reaching 95\% by $t = 10$. Red dashed lines mark discrete completion events. Irreversible monotonic growth.
    %
    \textbf{(Row 1, Right)} Categorical irreversibility: completion function $\mu(C,t)$ increases monotonically (blue staircase) from 0 to 9 states. Red arrow indicates irreversible time direction.
    %
    \textbf{(Row 2, Left)} Partition coordinates $(n, l, m)$: 3D scatter shows quantum state distribution. Colors indicate depth $n$ (purple: $n=1$, blue: $n=2$, green: $n=3$, yellow: $n=4$). States organized in shells.
    %
    \textbf{(Row 2, Center-Left)} Shell capacity theorem: $N(n) = 2n^2$. Blue bars show shell capacity (2, 8, 18, 32, 50, 72, 98, 128, 162, 200, 242, 280), orange cumulative curve. Perfect quadratic scaling.
    %
    \textbf{(Row 2, Center-Right)} Energy ordering rule: $(n + \alpha l)$ with $\alpha = 1$ generates Madelung rule (1s, 2s, 2p, 3s, 3p, 4s, 3d, ...). Horizontal bars show orbital filling sequence matching periodic table.
    %
    \textbf{(Row 2, Right)} Selection rules: $\Delta l = \pm 1$ allowed transitions. Diagram shows allowed paths (yellow arrows) between angular momentum levels (s, p, d, f). Energy increases vertically.
    %
    \textbf{(Row 3, Left)} Spherical harmonic $Y_2^0(\theta, \phi)$: 3D visualization shows $l=2$, $m=0$ angular distribution. Blue (positive) and red (negative) lobes demonstrate spatial anisotropy.
    %
    \textbf{(Row 3, Center-Left)} Angular momentum states: $l = 0, 1, 2$ with $m \in \{-l, ..., +l\}$. Grid shows probability densities for all $(l, m)$ combinations. Red/blue patterns indicate phase structure.
    %
    \textbf{(Row 3, Center-Right)} Chirality $s = \pm 1/2$: spin-up (blue, right-handed) and spin-down (red, left-handed) phase trajectories. Circular paths with opposite orientations demonstrate intrinsic angular momentum.
    %
    \textbf{(Row 3, Right)} State degeneracy: $g(n) = 2n^2$. Bars show total states per shell ($n=1$: 2, $n=2$: 8, $n=3$: 18, $n=4$: 32). Green shading indicates cumulative capacity.
    %
    Validation: Shell capacity $N(n) = 2n^2$, Madelung rule $(n + l)$ ordering, $\Delta l = \pm 1$ selection rules, $g(n) = 2n^2$ degeneracy.}
    \label{fig:categorical_partition}
    \end{figure*}

    \begin{figure}[htbp]
        \centering
        \includegraphics[width=\textwidth]{figures/panel_force_field_mapping.png}
        \caption{Comprehensive force field mapping demonstrating emergence of all fundamental interactions from partition coordinate geometry, spanning 40 orders of magnitude in coupling strength.
        \textbf{(A) Coulomb field (mode occupation asymmetry):} Electric field lines around point charges showing $1/r^2$ force law. Red and blue dots represent positive and negative charges, with field lines (black arrows) indicating force direction. Asymmetric mode occupation creates attractive/repulsive patterns characteristic of electromagnetic interactions.
        \textbf{(B) Yukawa potentials (mediator mass effect):} Exponentially screened potentials $V(r) \propto e^{-mr}/r$ for different mediator masses. Coulomb (m=0, blue): unscreened $1/r$ potential. Light mediator (m=0.5, green): moderate screening. Medium (m=1, orange) and heavy (m=2, red): strong screening at short range. Demonstrates how partition coordinate mass parameters generate different interaction ranges.
        \textbf{(C) Force hierarchy (40 orders of magnitude):} Logarithmic scale showing relative coupling strengths: Strong ($\alpha \approx 1$, red), Electromagnetic ($\alpha \approx 7 \times 10^{-3}$, blue), Weak ($\alpha \approx 10^{-6}$, orange), Gravity ($\alpha \approx 10^{-39}$, purple). All forces emerge from same partition geometry with different categorical parameters, explaining the hierarchy problem through geometric scaling.
        \textbf{(D) Resonance enhancement (mode coupling):} Response amplitude vs. driving frequency showing resonant peaks. Multiple curves ($\gamma = 0.01$ to $0.2$) demonstrate damping effects. Peak enhancement reaches $100 \times$ at resonance, showing how partition coordinate coupling generates strong interactions through frequency matching.
        \textbf{(E) 3D potential well (mode attraction):} Three-dimensional surface showing attractive potential with minimum at origin. Yellow surface indicates binding region, blue indicates repulsive barrier. Contour lines show equipotential surfaces characteristic of bound state formation in partition coordinate space.
        \textbf{(F) Mode overlap (coupling strength):} Radial wavefunctions for 1s (blue), 2s (orange), and 2p (green) states showing spatial overlap. Coupling strength proportional to overlap integral determines transition rates and interaction strengths between partition coordinate levels.
        \textbf{(G) Gravitational field (universal mode coupling):} Vector field showing universal attractive interaction. Purple arrows indicate field direction toward mass center. Demonstrates how gravity emerges as universal coupling between all partition coordinates, explaining equivalence principle through geometric universality.
        \textbf{(H) Scattering cross-section (resonance detection):} Energy-dependent cross-section showing resonant peaks (orange dashed) above smooth background (blue dotted). Total cross-section (blue solid) exhibits characteristic resonance structure enabling experimental detection of partition coordinate energy levels through scattering experiments.}
        \label{fig:force_field_mapping}
        \end{figure}
    

        \begin{figure}[htbp]
            \centering
            \includegraphics[width=\textwidth]{panel_05_selection_rules.png}
            \caption{\textbf{Selection rules emerge as geometric constraints on allowed trajectories.} 
            (\textbf{A}) Allowed versus forbidden transitions in energy-position space. Blue circles represent s-states ($$\ell = 0$$), green circles represent p-states ($$\ell = 1$$), red circles represent d-states ($$\ell = 2$$). Solid green lines show allowed transitions satisfying $$\Delta \ell = \pm 1$$ with transition rates $$> 10^6$$ s$$^{-1}$$. Dashed red lines show forbidden transitions ($$\Delta \ell \neq \pm 1$$) with suppressed rates $$< 10^{-2}$$ s$$^{-1}$$. Labels indicate measured transition rates. 
            (\textbf{B}) Angular momentum conservation diagram in $$L_x$$-$$L_y$$ plane. Blue arrow shows initial angular momentum $$\mathbf{L}_i$$, green arrow shows photon angular momentum $$\mathbf{L}_\gamma$$, red arrow shows final angular momentum $$\mathbf{L}_f = \mathbf{L}_i + \mathbf{L}_\gamma$$. Yellow shaded region indicates allowed final states satisfying $$|\mathbf{L}_f| = \sqrt{\ell(\ell+1)}\hbar$$ with $$\ell = 1$$. Black circles show measured transitions ($$N = 30$$), all falling within allowed region. 
            (\textbf{C}) Transition probability matrix $$P(\ell_i \rightarrow \ell_f)$$ for initial states $$\ell_i = 0$$ to 5 and final states $$\ell_f = 0$$ to 5. Yellow diagonal bands ($$P \sim 0.85$$-$$0.96$$) correspond to $$\Delta \ell = \pm 1$$ transitions. Black off-diagonal elements ($$P \sim 0$$) correspond to forbidden transitions. Matrix structure demonstrates geometric origin of selection rules. 
            (\textbf{D}) Three-dimensional angular momentum trajectory on the $$|\mathbf{L}| = \sqrt{2}\hbar$$ sphere (yellow surface, corresponding to $$\ell = 1$$). Blue curve shows measured trajectory from initial state (green sphere, $$\ell = 0$$) to final state (red square, $$\ell = 1$$). Trajectory remains confined to allowed surface, demonstrating angular momentum conservation throughout transition. Axes in units of $$\hbar$$.}
            \label{fig:selection}
            \end{figure}

            \begin{figure}[htbp]
                \centering
                \includegraphics[width=\textwidth]{figure1_ternary_encoding.png}
                \caption{Ternary encoding system for categorical state representation in 3D S-entropy coordinate space with hierarchical partition refinement.
                \textbf{(A) 3D entropy coordinate space:} Scattered points in $(S_k, S_t, S_e)$ unit cube showing categorical state distribution. Red and blue spheres indicate distinct categorical regions with connecting trajectories demonstrating state transitions.
                \textbf{(B) Hierarchical partition refinement:} Ternary subdivision at levels k=1 through k=4, showing exponential growth from $3^3=27$ to $3^{12}=531,441$ cells. Red boxes highlight selected partitions demonstrating recursive refinement structure.
                \textbf{(C) Ternary address encoding:} Tree structure showing hierarchical address assignment with example address "0210_3" (red highlight) mapping to specific categorical state. Bottom bar shows ternary digit sequence with color coding.
                \textbf{(D) Convergence to continuum:} Cell volume $V(k) = 3^{-3k}$ decreasing exponentially with trit number k. Blue curve crosses machine precision (red dashed) at k≈12, reaching continuum limit (gray region) for categorical state resolution.}
                \label{fig:ternary_encoding}
                \end{figure}
                

                \begin{figure}[htbp]
                    \centering
                    \includegraphics[width=\textwidth]{figures/spatial_matter_panel.png}
                    \caption{\textbf{Spatial structure, matter emergence, and energy relationships from partition coordinates.}
                    (\textbf{Top Left}) Three-dimensional spherical coordinates from angular quantum numbers $(\ell, m)$. Cyan wireframe sphere shows $\ell = 2$ surface with coordinate grid. Demonstrates how angular momentum quantum numbers generate spatial dimensionality.
                    (\textbf{Top Center}) Radial extension scaling as $r \propto n^2$ (Bohr radius scaling). Concentric circles show radial probability distributions for $n = 1$ to $7$, with colors progressing from center (red) to outer shells (purple). Central annotation indicates $n = 1$ to $n = 7$ progression.
                    (\textbf{Top Right}) Dimensionality from partition constraints showing quantum number limits. Bar chart displays spatial dimensions 1-4, with orange bar highlighting unique structure at $D = 3$ (our observed spatial dimensionality). Demonstrates geometric origin of three-dimensional space.
                    (\textbf{Top Far Right}) Locality principle showing exponential decay. Blue line demonstrates overlap integral decay with 1\% threshold (red dashed line) at separation $|n_1 - n_2| \sim 2$. Shows how partition separation generates spatial locality.
                    (\textbf{Middle Left}) Mode occupation statistics showing 6/100 occupied states (5\%). Blue bars indicate occupied modes (matter), white bars show unoccupied modes (vacuum). Demonstrates sparse occupation characteristic of matter in mostly empty space.
                    (\textbf{Middle Center}) Exclusion principle illustrated through electron configurations. Orbital diagrams for 1s, 2s, 2p, 3s levels showing spin-up/spin-down occupancy (red arrows). Maximum 2 electrons per orbital enforces Pauli exclusion through partition coordinate uniqueness.
                    (\textbf{Middle Right}) Mass-frequency identity $m = \hbar\omega/c^2$. Blue line shows linear relationship between mass and oscillation frequency $\omega$, with annotations for muon and electron masses. Demonstrates how partition oscillation frequency generates particle mass.
                    (\textbf{Middle Far Right}) Cosmic mode occupation showing dark energy (68\%), dark matter (27\%), and ordinary matter (5\%). Pie chart demonstrates that most partition coordinates remain unoccupied, explaining dark sector dominance.
                    (\textbf{Bottom Left}) Wave-particle duality showing mode versus occupation. Blue oscillatory curve represents wave function (mode structure), red dashed line shows particle occupation probability. Red circle indicates measurement collapse to particle state.
                    (\textbf{Bottom Center}) Energy conservation with $dE/dt = 0$. Oscillatory curves show kinetic energy (blue), potential energy (red), and constant total energy (black). Demonstrates energy conservation through partition coordinate dynamics.
                    (\textbf{Bottom Center-Right}) Mode occupation statistics comparing Fermi-Dirac (blue, $s = \pm 1/2$) and Bose-Einstein (red, $s = 0, 1, ...$) distributions. Shows occupation probability versus energy, with Fermi surface at $\mu = 2.0$.
                    (\textbf{Bottom Right}) Vacuum energy from unoccupied mode contributions. Purple curve shows energy density versus frequency $\omega$, demonstrating how unoccupied partition coordinates contribute to vacuum energy density.}
                    \label{fig:spatial_matter_energy}
                \end{figure}
                

                
\begin{figure}[htbp]
    \centering
    \includegraphics[width=\textwidth]{figures/panel_03_multimodal_synthesis.png}
    \caption{Multi-modal measurement synthesis achieving $10^5 \times$ enhancement through five independent spectroscopic modalities with uncorrelated noise combination.
    \textbf{Top left:} Individual modality SNR enhancement showing 10× improvement across frequency (Doppler), phase (optical path), amplitude (absorption), polarization (Faraday), and temporal (impulse) measurements.
    \textbf{Top right:} Combined SNR enhancement vs. number of modalities. Red curve (1000 meas/modality) achieves target $10^5$ enhancement (red dashed) with 5 modalities through $\sqrt{n_{total}}$ scaling.
    \textbf{Bottom left:} Error reduction following $1/\sqrt{n \cdot n_{mod}}$ law. Five independent modalities (red) achieve $\sigma = 0.045$ vs. single modality $\sigma = 0.10$ at 100 measurements.
    \textbf{Bottom right:} 3D measurement distribution showing variance minimization in (frequency shift, phase delay, variance) space. Target zero variance (star) approached through multi-modal combination with uncorrelated noise sources.}
    \label{fig:multimodal_synthesis}
    \end{figure}
    
    
    \begin{figure}[htbp]
    \centering
    \includegraphics[width=\textwidth]{figures/panel_04_harmonic_coincidence.png}
    \caption{Harmonic coincidence network achieving $10^3 \times$ enhancement through frequency space triangulation with K=12 harmonic constraints.
    \textbf{Top left:} Harmonic frequency detection showing 45 coincidences (blue dots) with linear harmonic progression. Red stars indicate triangulation points for frequency space mapping.
    \textbf{Top right:} Network topology with 15 nodes, 45 edges providing $\sqrt{45} = 6.7 \times$ enhancement. Numbered nodes show connectivity pattern for harmonic coincidence detection.
    \textbf{Bottom left:} Uncertainty reduction through triangulation. Blue curve: triangulation-only scaling $1/\sqrt{K}$. Red curve: combined with beat frequencies achieving $10^{-3}$ total uncertainty at K=12 constraints.
    \textbf{Bottom right:} 3D frequency space network showing 30 oscillators with 40 connections in $(f_1, f_2, f_3)$ coordinates. Color gradient indicates node degree (2-10), demonstrating distributed harmonic relationships enabling network enhancement $F_{graph} = 59,428$ in full implementation.}
    \label{fig:harmonic_coincidence}
    \end{figure}
    
    
    \begin{figure}[htbp]
    \centering
    \includegraphics[width=\textwidth]{figure1_ternary_encoding.png}
    \caption{Ternary encoding system for categorical state representation in 3D S-entropy coordinate space with hierarchical partition refinement.
    \textbf{(A) 3D entropy coordinate space:} Scattered points in $(S_k, S_t, S_e)$ unit cube showing categorical state distribution. Red and blue spheres indicate distinct categorical regions with connecting trajectories demonstrating state transitions.
    \textbf{(B) Hierarchical partition refinement:} Ternary subdivision at levels k=1 through k=4, showing exponential growth from $3^3=27$ to $3^{12}=531,441$ cells. Red boxes highlight selected partitions demonstrating recursive refinement structure.
    \textbf{(C) Ternary address encoding:} Tree structure showing hierarchical address assignment with example address "0210_3" (red highlight) mapping to specific categorical state. Bottom bar shows ternary digit sequence with color coding.
    \textbf{(D) Convergence to continuum:} Cell volume $V(k) = 3^{-3k}$ decreasing exponentially with trit number k. Blue curve crosses machine precision (red dashed) at k≈12, reaching continuum limit (gray region) for categorical state resolution.}
    \label{fig:ternary_encoding}
    \end{figure}
    
    \begin{figure}[htbp]
    \centering
    \includegraphics[width=\textwidth]{figures/topology_categories_panel.png}
    \caption{Topological structure of categorical spaces showing partial ordering, dimensional relationships, and completion dynamics in hierarchical categorical measurement systems.
    \textbf{(A) Partial order:} Completion precedence structure showing hierarchical dependencies between categorical states with directed connectivity indicating measurement ordering constraints.
    \textbf{(B) Tri-dimensional S-space:} Three-dimensional coordinate system $(S_k, S_t, S_e)$ with yellow point indicating specific categorical state location within unit cube geometry.
    \textbf{(C) $3^k$ branching structure:} Hierarchical tree showing exponential branching with root C and ternary subdivision creating multi-colored terminal nodes representing categorical state endpoints.
    \textbf{(D) Scale ambiguity:} Identical triangular structures at Level n and Level n+1 demonstrating scale-invariant topology with ambiguity parameter $\Psi_n$ indicating measurement uncertainty.
    \textbf{(E) Completion trajectory:} Fraction completed $\gamma(t)$ approaching unity asymptotically (green curve) with completion target (red dashed line) showing bounded convergence dynamics.
    \textbf{(F) Asymptotic slowing:} Completion rate $\dot{C}(t) \to 0$ (red curve) with completion time $T$ (dotted line) demonstrating deceleration in categorical state enumeration approaching completeness.}
    \label{fig:topology_categories}
    \end{figure}
    
    \begin{figure}[htbp]
    \centering
    \includegraphics[width=\textwidth]{autocatalysis_dynamics_panel.png}
    \caption{Autocatalytic dynamics in virtual instruments demonstrating burden-dependent resistance reduction, coupling enhancement, and information generation rate amplification across multiple spectroscopic configurations.
    \textbf{Top row:} Resistance formula $R = 1/(1 + B)$ showing inverse relationship with categorical burden B (left). Effective coupling enhancement from 0.1 to 0.2 with burden accumulation, doubling base coupling strength (center). Multiple burden trajectories converging to unity over 100 measurement cycles (right).
    \textbf{Middle row:} Autocatalytic phase space showing rate vs burden trajectories with flow field vectors (left). SNR enhancement comparison between incoherent $N^{0.5}$ and coherent $N^{0.7}$ scaling, reaching 15× advantage at N=50 (center). 2D correlation heat map from autocatalysis showing enhanced signal correlation in $(\omega_1, \omega_2)$ frequency space (right).
    \textbf{Bottom row:} Virtual instrument configurations (XPS, UV-Vis, Zeeman, NMR) with identical hardware but different software, showing frequency ranges from $10^6$ to $10^{17}$ Hz (left). Categorical burden persistence across all four configurations maintaining 0.8-1.0 levels throughout measurement cycles (center). Information generation rate enhancement from 1.0 to 2.0× through accumulated burden, demonstrating autocatalytic amplification (right).}
    \label{fig:autocatalytic_dynamics}
    \end{figure}
    
    
    \begin{figure}[htbp]
        \centering
        \includegraphics[width=\textwidth]{information_catalysis_validation.png}
        \caption{Comprehensive validation of information catalysis through categorical apertures, demonstrating enhanced signal averaging, cross-coordinate reduction, and zero-cost information processing compared to Maxwell's demon paradigm.
        \textbf{Top row:} Signal averaging enhancement showing autocatalytic $\alpha = 0.441$ vs standard $\alpha = 0.224$ scaling exponents, validating theoretical predictions (left). Alpha enhancement bar chart confirming 0.441 > 0.5 theoretical minimum (center). Burden accumulation over 50 measurement cycles reaching saturation (right).
        \textbf{Middle row:} Partition completion information showing exponential approach to 2.5 bits with cumulative information overlay (left). Cross-coordinate distance reduction from 5.87 (independent) to 4.50 (sequential), validating 1.37 unit improvement (center). 2D spectroscopy enhancement achieving 1.39× information gain (15.60 vs 11.24 bits) (right).
        \textbf{Bottom sections:} Maxwell's demon vs categorical aperture comparison highlighting key difference: demon acquires information (Shannon bits) with $kT \ln(2)$ cost per bit, while aperture generates information through partition completion at zero cost. Resonance aperture profile with FWHM = 2$\Gamma$ (center). Validation summary confirming all theoretical predictions with autocatalytic enhancement factor of 2.0× at full burden, frequency regime separation across four decades, and zero information processing cost validation.}
        \label{fig:information_catalysis_validation}
        \end{figure}